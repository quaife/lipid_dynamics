\documentclass[11pt]{article}
\usepackage[utf8]{inputenc}
\usepackage{fullpage}
\pagestyle{empty}
%\linespread{1.05}
\begin{document}
%\section*{Summary}


\subsection*{Overview}
\vspace{-0.1in}
This proposal aims to advance mathematical modeling, analysis, and
numerical simulations of collective dynamics and self-assembly arising
from amphiphilic attractive interactions between finite-sized bipolar
particles in a suspension. Experimentalists study amphiphilic
self-assembly using particles with a biphasic
material label.
These amphiphilic particles experience a non-additive, long-range
attraction that is a ubiquitous source of nonspecific interactions in
biology and soft matter. Motivated by its broad applicability, the PIs
recently developed a mathematical model for phase-field-mediated
interactions between particles. This model was used to show, for the
first time, self-assembly of bipolar particles into vesicle-like
structures using moving domain elliptic partial differential equations
(PDEs).

The PIs propose to analyze the elliptic PDEs, develop fast and accurate
numerical algorithms, and apply their hydrophobic attraction potential
model to more general self-organization dynamics. They aim to generalize
the linear model to a wider class of hydrophobic interactions, and
numerically implement these interactions using integral equation
formulations. Finally, they will develop a coarse-grained model based on
kinetic theory to investigate the rheological properties of the assembly
of particles with tunable hydrophobicity. The unifying aspects of the
project will establish a platform for efficiently simulating the
collective dynamics of large numbers of particles, and focus on
rigorous, mathematical analysis of the underlying model equations in
complex geometries. The project supports undergraduate and graduate
student collaborators.

\vspace{-0.1in}
\subsection*{Intellectual Merit}
\vspace{-0.1in}
This collaborative project focuses on modeling, simulation, and analysis
of self-assembly and collective dynamics of a suspension of granules
with non-uniform hydrophobicity on their boundaries. The main ingredient
is a nonlocal interaction through the solution of moving domain elliptic
PDEs that encompasses long-range (non-additive) amphiphilic and
short-range steric interactions. The PIs have validated this
coarse-grained model against well-studied vesicle hydrodynamics. The
technical research tasks include quantifying collective properties of
ensembles of granules with tunable hydrophobicity, analyzing the
mathematical model in distinguished limits to make connections with
continuum models, designing efficient and high-order numerical
algorithms for large-scale two- and three-dimensional simulations with
confinement, and developing a kinetic theory to quantify both the polar
and nematic characteristics in the collective dynamics.

%The purpose of this research is to reach interesting physical phenomena
%with less computational cost than molecular dynamics, and account for
%more general features that continuum theory misses. The main ingredient
%is defining a nonlocal interaction through the solution of an elliptic
%boundary value problem that has the phenomenological characteristics of
%long-range hydrophobic attraction. This minimal model, while intuitive,
%is quite a general description of amphiphiles in solvent and gives rise
%to rich phenomena from Janus particle aggregates to correctly predicting
%elastic properties of bilayer. The technical research tasks include
%quantifying collective properties of amphiphilic ensembles, improving on
%mathematical models, efficient, high-order numerical algorithms for
%large-scale simulations, and incorporating external fields through
%electric charge.

\vspace{-0.1in}
\subsection*{Broader Impacts}
\vspace{-0.1in}
This project aims to advance the mathematical modeling of collective
dynamics of amphiphilic granules. The framework uses a new PDE-based
formulation that accounts for important and complex systems in soft
matter. These complex systems include optimal shape design in
metamaterials and fusion and fission of amphiphilic bilayer membranes.
The simulations will be performed with computational tools designed to
solve the governing PDEs both efficiently and with high-order accuracy.
The models describing granular systems could be transformative in
biomedicine and material science. The research draws from expertise in
scientific computing, physics of fluids, and mathematics. The
mathematical component incorporates variational techniques and offers
insight into fundamentals of self-organization and collective dynamics.
The project brings socially consequential research into the classroom
and offers undergraduates the opportunity to train alongside faculty and
graduate students. With its unique combination of mathematical modeling,
analysis, and scientific computing, the project highlights the potential
advancement of STEM from Applied Mathematics.

\end{document}

