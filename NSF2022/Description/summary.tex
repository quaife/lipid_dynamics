\documentclass[10pt]{article}
\usepackage[utf8]{inputenc}
\usepackage{fullpage}
\pagestyle{empty}
\linespread{1.05}
\begin{document}
\section*{Summary}


\subsection*{Overview}
\vspace{-0.1in}
This proposal aims to advance mathematical modeling, analysis and numerical simulations of
collective dynamics and self-assembly arising from attractive
interactions between dipolar/nematic particles in a suspension. 
%
%One example is the organization of lipids and proteins in vesicles, which is crucial
%to design effective drug carriers. 
%
Physicists model such a self-assembly
using particles possessing a biphasic material label on either half,
with one more hydrophobic than the other. Hydrophobic surfaces
experience a nonadditive long-range attraction called the hydrophobic
force. This force is a major source of nonspecific interactions in
biology and soft matter. Despite its ubiquity, theoretical developments
explaining its basic underlying principles have come relatively late.
Motivated by its broad applicability, the PIs recently developed a novel
and intuitive mathematical model, called the hydrophobic attraction
potential model, based on the linear response of water to surface
perturbations. This model was used to show, for the first time,
self-assembly of such bipolar particles into vesicle-like structures
using moving domain elliptic partial differential equations.


%This proposal aims to advance mathematical modeling and analysis of
%collective dynamics of amphiphilic self-assembly. Amphiphilic particles
%such as lipid molecules possess both hydrophobic and hydrophilic
%surfaces and self-assemble into meso/macroscopic structures such as
%micelles and bilayers of lipids. Experimentalists model self-assembly
%using Janus particles—spherical particles possessing a biphasic material
%label on either hemisphere. Janus particles have been widely used in
%soft matter physics as a model amphiphilic colloid and can be designed
%for catalytic activity or stimuli-responsive smart materials.

%Hydrophobic surfaces experience a nonadditive long-range attraction
%called the hydrophobic force. This force is a major source of
%nonspecific interactions in biology and soft matter. Despite its
%ubiquity, theoretical developments explaining its basic underlying
%principles have come relatively late. Motivated by its broad
%applicability, the PIs recently developed a novel and intuitive
%mathematical model, called the hydrophobic attraction potential model,
%based on the linear response of water to surface perturbations. This
%model was used to show, for the first time, self-assembly of Janus
%particles into vesicle-like structures using a partial differential
%equation-based particle interaction.

The PIs propose to apply their hydrophobic attraction potential model to
more general self-organization dynamics. They aim to generalize the
linear-response model to more general hydrophobic interactions, and
numerically implement these interactions using integral equation
formulations. Finally, they will develop a coarse-grained model based on 
the kinetic theory to investigate the rheological properties of the assembly of 
particles with tunable hydrophobicity. The unifying
aspects of the project are establishing a platform for efficiently
simulating the collective dynamics at large scales, and the focusing on
rigorous, mathematical analysis of the underlying model equations and
complicated geometries. The project supports undergraduate and graduate
student collaborators.

\subsection*{Intellectual Merit}
\vspace{-0.1in}
The purpose of this research is to reach interesting physical phenomena
with less computational cost than molecular dynamics, and account for
more general features that continuum theory misses. The main ingredient
is defining a nonlocal interaction through the solution of an elliptic
boundary value problem that has the phenomenological characteristics of
long-range hydrophobic attraction. This minimal model, while intuitive,
is quite a general description of amphiphiles in solvent and gives rise
to rich phenomena from Janus particle aggregates to correctly predicting
elastic properties of bilayer. The technical research tasks include
quantifying collective properties of amphiphilic ensembles, improving on
mathematical models, efficient, high-order numerical algorithms for
large-scale simulations, and incorporating external fields through
electric charge.

\subsection*{Broader Impacts}
\vspace{-0.1in}
This project aims to advance the mathematical modeling of collective
dynamics of amphiphilic granules. The framework uses a new PDE-based
formulation that accounts for important and complex systems in soft
matter. These complex systems include optimal shape design in
metamaterials and fusion and fission of amphiphilic bilayer membranes.
The simulations will be performed with computational tools designed to
solve the governing PDEs both efficiently and with high-order accuracy.
The models describing granular systems could be transformative in
biomedicine and material science. The research draws from expertise in
scientific computing, physics of fluids, and mathematics. The
mathematical component incorporates leading variational techniques and
offers insight into fundamentals of self-organization and collective
dynamics. The project brings socially consequential research into the
classroom and offers undergraduates the opportunity to train alongside
faculty and graduate students. With its unique combination of
mathematical modeling, analysis, and scientific computing, the project
highlights the potential advancement of STEM from applied mathematics.


\end{document}

