\section{Proposed Research}
\label{sec:proposed-work}

\subsection{Specific Aim 1: Membrane vesicles as self-organized granules}
\label{sec:specific_aim1}

We introduced \eqref{eq:RBT}--\eqref{eqn:stressbalance} to overcome the challenges faced by
traditional Helfrich-type models of elastic membranes.
We have compared two-dimensional configurations of amphiphilic granules
to membrane configurations found in continuum mechanics.
The simulations results in~\cite{Fu2018_SIAM,FuQuRyYo22}
showed that the granular bilayers 
basically replicated the elastic energies for bending, tilt, and
stretching.
The hydrodynamics of the Janus particle vesicle is similar to the behaviour of an inextensible,
elastic vesicle membrane with permeability.
The model has reached sufficient maturity to warrant 
rigorous studies in analysis.

A natural questions concerns the well-posedness
\eqref{eq:RBT}--\eqref{eqn:stressbalance}
and its generalizations. A recent and strongly developing direction in
the area of nematics liquid crystals concerns the interaction of
Landau de Gennes functionals like \eqref{eq:energy_law} with colloids
\cite{doi:10.1098/rsta.2020.0432}.
In this spirit, a first step to understand the effect of the flow
in our model is to obtain some solid well-posedness results.

\begin{quotation}
  \textbf{Problem 1.} Consider a collection the closed,
  disjoint granules $U_i$ with smooth boundary,
with no-slip, rigid body motion boundary conditions.
Assume a either a single or double well potential
and a smooth phase field boundary condition.  
Study the well-posedness of the coupled system
\eqref{eq:RBT}--\eqref{eqn:stressbalance}
for short times.  For global in time solutions, determine
whether lubrication forces ensure positive distance
between granules.  Describe the behaviour in the limit
zero screening length limit.  
\end{quotation}

This problem is attractive because combines the diffusive interface
effects involving a free boundary in aqueous part of the region \cite{}.
The well-posedness of Allen-Cahn-type equations is well-established in the
literature.  What makes this challening, though, is that
the moving domain means the function space for the solution,
a Sobolov space, is also changing with time.
On ther other hand, the problem \eqref{eq:RBT}--\eqref{eqn:stressbalance}
is an autonmous ordinary differential equation for rigid body transformations.
The right-hand side of this equation is defined by a system of elliptic
PDE.  On the other hand, several methods have been developed for
analysing the smooth domain dependence of solutions to elliptic PDE,
especially in the study of eigenvalue problems, for example.
It is conceivable that moving domain estimates can be applied to the
present problem to exhibit locally Lipschitz dependence on
the rigid deformations. 

Generalizations of Problem 1 include modifying the 
system to account for various time scales.
We may consider, for example, modifying \eqref{eqn:phase}
to include transport of the phase field by the background fluid;
\begin{equation*}
\phi_t + \uu \cdot \nabla = -\lambda \frac{\delta E}{\delta \phi} = \lambda(\epsilon^2 \Delta \phi - W'(\phi))
\end{equation*}
The modified systems enjoys the same energy law as \eqref{eq:energy_law} 
except that there will be an addtional dissipation term coming from the
Euler-Lagrange derivative $\frac{\delta E}{\delta \phi}$.
(DEFINE $\lambda$)

The second problem we consider concerns the approximation
of equilibrium elastic energy of curves and surfaces.
To understand the approximation mathematically, we take the following:
\begin{quotation}
  \textbf{Problem 2.} Place a collection of
  amphiphilic granules
with equally spaced centers along a curve $\mathcal{C}$.
For simplicity, assume that the directors are normal to the curve
and that the granules are all disks with radius proportional to $\epsilon$. 
Study the sharp interface limit of the energy $E$ 
relative to that of an array of granules placed along a flat curve. 
In this limit, the number of granules $N_b$ goes to infinity
to ensure equal spacing.
Study how the bending the bending $k_B$ and stretching $k_A$ coefficients
depend on the granual size relative to their spacing.
Generalize the problem to account for aspect ratio of elliptical granules
and nonzero tilt. 
\end{quotation}
Given the relatively complicated geometry consising
of a plane with a number of disjoint disks removed, 
we can start by approaching the problem in a simplified
setting such as when the reference curve is a circle.
There is a large body work in this direction for nematics
studying either the case of a single coloidal particle and
the qualitative structure of defects around it
\cite{Alama2015MinimizersOT, Alama2021SaturnRD, PhysRevE.96.042702}
many particles and the homogenization effects
\cite{Canevari2019DesignOE,doi:10.1137/18M1163919,doi:10.1137/18M1163919,BERLYAND200597,doi:10.1137/130910348}.
A closely related area that consider similar analysis
dimension reduction for
Landau-de Gennes models on curved nematic confined to thin films
\cite{Golovaty2017DimensionRF, Golovaty2015DimensionRF,doi:10.1142/S0218202516500470, FoFrLe07}.

We conclude this section by highlighting another well-developed
area connected to our approach.  Surface-director models 
considers the energy of a nematic liquid crystal on a surface to study how the
surface geometry effects the director field and vice versa. Researchers
have formulated finite element methods for [][] and studied the existence
of minimizers []-[].
Liquid crystals are an extremely well-studied are in mathematics [].


On the other hand, in biophysics, tilt is defined by the tilt vector
  $\TT = \dd/\dd\cdot \NN - \NN$ so that the tilt energy density
  behaves like $\tan^2 \theta$ rather than $\sin^2 \theta$.
  These two energies densities are, however, asymptotically equal
  for small deformations.
Hamm and Kozlov [] and later Terzi and Deserno [], Pinigen et al. []
showed that the full Helfrich elastic energy of a \emph{monolayer} takes the form
\begin{equation}
\label{eq:Helfrich}
  \begin{aligned}
 &\int_{\Sigma}
  %\label{eq:Pinigan}
\tfrac{1}{2}k_{b}[(\nabla \cdot \mathbf{d} + k_{0})^{2} -  k^{2}_{0}]  
+ \tfrac{1}{2}k_{\theta}\mathbf{T}^{2} + \tfrac{1}{2}k_a[(\alpha - \alpha_0)^2 - \alpha_0^2] \\
&+ k_{c}\textbf{T} \cdot \nabla \nabla \cdot \mathbf{d}  + \tfrac{1}{2}k_{g}(\nabla \nabla \cdot \mathbf{d})^{2}
 + A\alpha \nabla \cdot \mathbf{d}
+ B \mathbf{T} \cdot \nabla \alpha \\
&- \tfrac{1}{2}k_c |\nabla \alpha|^2 + C \nabla \alpha \cdot \nabla \nabla \cdot \mathbf{d}\,dS
\end{aligned}
\end{equation}
where $\nabla$ and $\nabla \cdot$ are the surface gradient and surface divergence,
respectively, and $k_B$, $k_{\theta}$, $k_a$, $k_c$, $k_g$, $A$, $B$, $C$ are
appropriate elastic moduli.  An important and consequential distinction between
\eqref{eq:Canham-Helfrich} is the presence of the terms $\mathbf{T}$ and $\alpha$.
These are, respectively, the \emph{tilt} vector $\mathbf{T}$ measuring
the difference between the lipid director $\mathbf{d}$ and the surface normal
$\mathbf{N}$; the \emph{area per lipid} $\alpha$ area density at rest $\alpha_0$.
The two models \eqref{eq:Canham-Helfrich} and \eqref{eq:Helfrich} agree
when $\mathbf{T} = 0$ and $\alpha = \alpha_0$ everywhere.

(MAKE CONNECTION WITH SPECIFIC AIM 2)
(PLAN B)
(SHORTER PROBLEM STATEMENT; ADDRESS IN PARAGRAPH)

Our future goals include extending the current framework to a three-dimensional JP vesicle system. This will require additional algorithmic implementation including a fast summation method such as the fast multipole method. Other research directions include fluctuating hydrodynamics for Brownian suspensions (Bao et al. 2018), critical to understanding membrane diffusion, and an analysis of the relationship between particle shape on effective properties. Finally, physically more accurate boundary conditions for the HAP model will allow us to draw comparisons between computational and laboratory experiments.
