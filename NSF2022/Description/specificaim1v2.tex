%\documentclass[11pt]{article}
%
%%\usepackage[margin=1.01in]{geometry}
%%\usepackage{mathpazo}
%\usepackage[utf8]{inputenc}
%\usepackage{NSFProposal}
%
%%\addtolength{\oddsidemargin}{-0.6in}
%%\addtolength{\evensidemargin}{-0.75in}
%%\addtolength{\textwidth}{1.2in}
%%\addtolength{\topmargin}{-0.75in}
%%\addtolength{\textheight}{1.4in}
%%
%%\usepackage[margin=1.02in]{geometry}
%%\def\baselinestretch{1}
%%\special{papersize=8.5in,11in}
%
%%\usepackage[left=1.1in, right=1.1in]{geometry}
%
%%
%\usepackage{url}
%\usepackage{amsfonts,amsmath,amssymb,amsbsy}
%\usepackage{latexsym}
%
%\usepackage{bm}
%\usepackage{comment}
%%\usepackage{upgreek}
%\usepackage{graphics,graphicx}
%\usepackage{psfrag}
%\usepackage{epsfig}
%\usepackage{wrapfig}
%\usepackage{mathrsfs}
%
%\usepackage{pgfgantt}
%\usepackage{subfigure}
%
%\usepackage[square,numbers,comma,sort&compress]{natbib}
%%\usepackage{hyperref}
%
%\usepackage{commath}
%\usepackage{array}
%%\newcommand{\todo}[1]{%
%%  {\bfseries\color{red} XXX TODO: #1}
%%}
%\usepackage[font=small,labelfont=bf,labelsep=period]{caption}
%%\usepackage[margin=30pt,font=small,labelfont=bf,
%%  labelsep=period,skip=10pt]{caption}
%
%\usepackage{todonotes}
%\usepackage{pgfplots,tikz,color,soul,array,enumitem}
%
%
%\newcounter{midequation}
%
%\newtheorem{definition}{Definition}
%\newtheorem{lemma}{Lemma}
%\newtheorem{example}{Example}
%\newtheorem{theorem}{Theorem}
%\newtheorem{proposition}{Proposition}
%\newtheorem{remark}{Remark}
%
%
%
%\newcommand{\diff}{\mathrm{d}}
%\newcommand{\intd}{\,\mathrm{d}}
%\newcommand{\kBT}{$\mathrm{k}_{\mathrm{B}}\mathrm{T}$}
%\newcommand{\KB}{k_{\mathrm{B}}}
%\newcommand{\KG}{k_{\mathrm{G}}}
%\newcommand{\KT}{k_{\mathrm{T}}}
%\newcommand{\KTH}{k_{\theta}}
%\newcommand{\KA}{k_{\mathrm{A}}}
%\newcommand{\eg}{{\it e.g.~}}
%\newcommand{\ie}{{\it i.e.~}}
%\DeclareMathOperator{\Div}{div}
%\DeclareMathOperator{\Curl}{curl}
%
%% Bryan's Marcros
%\renewcommand{\aa}{\mathbf{a}}
%\newcommand{\bd}{\partial}
%\newcommand{\dd}{\mathbf{d}}
%\newcommand{\DD}{\mathcal{D}}
%\newcommand{\eeta}{\boldsymbol{\eta}}
%\newcommand{\FF}{\mathbf{F}}
%\renewcommand{\gg}{\mathbf{g}}
%\newcommand{\GG}{\mathbf{G}}
%\newcommand{\JJ}{\mathbf{J}}
%\newcommand{\nn}{\mathbf{n}}
%\newcommand{\NN}{\mathbf{N}}
%\newcommand{\nnu}{\boldsymbol{\nu}}
%\newcommand{\ttau}{\boldsymbol{\tau}}
%\newcommand{\ssigma}{\boldsymbol{\sigma}}
%\newcommand{\rr}{\mathbf{r}}
%\newcommand{\RR}{\mathbb{R}}
%\renewcommand{\SS}{\mathbf{S}}
%\newcommand{\TT}{\mathbf{T}}
%\newcommand{\xx}{\mathbf{x}}
%\newcommand{\XX}{\mathbf{X}}
%\newcommand{\uu}{\mathbf{u}}
%\newcommand{\vv}{\mathbf{v}}
%\newcommand{\yy}{\mathbf{y}}
%\newcommand{\pderiv}[2]{\frac{\partial #1}{\partial #2}}
%\newcommand{\jump}[1]{[\![ #1 ]\!]}
%
%\begin{document}
%
%
%\newpage
%\pagenumbering{arabic}
%%\documentclass{article}
\usepackage[utf8]{inputenc}
\usepackage{fullpage}
\pagestyle{empty}

\begin{document}
\begin{center}
{\bf \Large Summary}\\ Collaborative Research
\begin{tabular}{ccc}\\ 
Fordham University & Florida State University& New Jersey Institute of Technology\\
Rolf Ryham &            Bryan Quaife &              Yuan-Nan Young
\end{tabular}

\end{center}

\section*{Overview}
Numerical simulation has become an essential tool in nearly all areas of science and engineering, especially biophysics. This proposal specifically aims to advance mathematical modeling and analysis in membrane biophysics through the study of
collective dynamics of amphiphilic self-assembly. Self-assembly is a ubiquitous process in biology and is a major source of 
nonspecific interactions in soft matter. In the case of membranes, lipids self-assemble into bilayer and micelle morphologies. Over the past decades, macroscopic continuum mechanics has done a good job of modeling bilayer membranes at large length scales but can miss details at small length scales brought about by inclusions or topological changes that are crucial to the dynamics. Molecular dynamics models account for these granular details but are computationally costly when accurately accounting for nonlocal interactions. To design better membrane models, it is imperative to quantify the collective physical properties and processes of amphiphiles, and the proposed mathematics establishes a platform for efficiently simulating the collective dynamics at large scales in a manner that allows for direct comparison with preexisting theory and experiment.

\section*{Intellectual Merit}
The purpose of this research is to reach interesting physical phenomena with 
less computational cost than molecular dynamics, and account for more general
features that continuum theory misses. The main ingredient is defining a 
nonlocal interaction through the solution of an elliptic boundary value problem
that has the phenomenological characteristics of long-range hydrophobic
attraction. It turns out that this minimal model gives rise to rich phenomena
for Janus particle aggregates and correctly predicts elastic properties of bilayer. 
The technical research tasks include quantifying collective properties of 
amphiphilic ensembles, mathematical analysis of continuum elastic energies, 
efficient, high-order numerical algorithms for large-scale simulations, and 
incorporating external fields through electric charge. Lastly, the proposal 
extends the results using three-dimensional boundary integral formulations.

\section*{Broader Impacts}
This project aims to advance the mathematical modeling of collective 
dynamics of amphiphilic particles. The simulations use a new, yet intuitive,
approach that can account for important and complex systems that are out of 
reach in computational material science. These complex systems include 
fusion and fission of amphiphilic bilayer membranes and optimal shape design
in metamaterials. The development of three-dimensional 
models describing colloidal systems could be transformative in biomedicine
and material science. The research draws from expertise in scientific 
computing, physics of fluids, and mathematics. The mathematical component 
incorporates leading techniques from geometric analysis and gives deep insight 
into fundamental material science. The project offers undergraduates 
in a socially impactful manner the opportunity to do research and train 
with graduate and postdoctoral personnel. It incorporates research in the 
classroom, and with its combination of mathematical modeling, analysis, 
and scientific computing, the project highlights the importance of 
mathematics and computation to all areas of science and engineering.




\end{document}

%\pagestyle{empty}

\section{Specific Aim 1: Membrane vesicles as self-organized granules}
\label{sec:specific_aim1}
\subsection{Prior work related to the proposal.}

\subsubsection{Vesicles}
\begin{wrapfigure}[12]{r}{0.3\textwidth}
  \vspace{-5pt}
\centerline{\includegraphics[width=0.3\textwidth]{figures/SA1Figures/LiposomePore.pdf}}
  \vspace{-5pt}
\caption{\label{fig:LiposomePore}Pore formation in a
stressed liposome~\cite{Kaetal03, BrdGSa00}.}
\end{wrapfigure}
PI Ryham has made fundamental contributions to interface modeling in
membrane biology.  His first works include
phase field formulations for the elastic bending energy \cite{0951-7715-18-3-016,Du05}
and 
topological indicators \cite{DuEuler} of vesicle membranes
and fluid-vesicle interactions \cite{QiangDu09}.
A decade ago, Ryham focused on the
problem of \emph{lipidic pore dynamics}.  
This problem came from 
experiments in the late 1990's 
showing that liposomes swollen by osmotic stress can form a single pore that
widens to allow the release of fluid (see Figure \ref{fig:LiposomePore}). 
Once pressure is released,
the widened pore closes according to a linear
followed by a square root law \cite{BrdGSa00}.  Researchers have
used the rates of pore closure to infer properties of the lipid membrane \cite{PoDi10}. 


In \cite{BrdGSa00}, 
physicists Olivier Sandre and Fran\c{c}oise Brochard-Wyart
and Nobel laureate Pierre-Gilles de Gennes
(PGG) formulated
rate equations for lipid pore dynamics.  These equations couple
the liposome stretching tension, pore edge tension, and membrane
dissipation.  A coefficient $C$ accounting for dissipation due to water
movement, however, was left unspecified because the Stokes flow
problem for the fluid surrounding a spherical cap with a moving hole
was unknown at the time \cite{Ra73}.  PI Ryham approached this problem using
a two-phase field method model \cite{RyCoEi12} and in a much-cited work
(coauthored with an undergraduate math major) by
fitting a dissipation coefficient $C \approx 8$ to data.
data \cite{RYHAM20112929}.  Finally, in the sole-author 
JFM paper \cite{Ryham2017OnTV},
Ryham solved the Stokes flow problem for
a moving pore completely yielding a value 
$C = \frac{4}{\pi}\csc \alpha\cot^2\frac{\alpha}{2}(\alpha - \sin \alpha + \alpha^2 \tan \frac{\alpha}{2})
~\sim 2\pi + 5(\alpha - \pi)$ in the pore angle $\alpha$.
Olivier Sandre wrote about the publication of the PI’s JFM paper:

\begin{quotation}“Maybe you can tell them this story: In my thesis
paper with Brochard \& PGG we got a factor of 8 wrong in determining
the viscosity of lipid membranes until a math teacher from the
Bronx corrected the model prefactor!”  Sandre, Olivier @Olive\_Free.
Twitter, January 3, 2018, https://twitter.com/Olive\_Free/status/948640705362169857
\end{quotation}

\begin{wrapfigure}[14]{l}{0.4\textwidth}
  \vspace{-5pt}
\centerline{\includegraphics[width=0.4\textwidth]{figures/SA1Figures/FusionMicroscopy.pdf}}
  \vspace{-5pt}
\caption{\label{fig:FusionMicroscopy}
Fusion of an influenza virus-like particle with a liposome resolved
by electron tomography
~\cite{Chetal16}.}
\end{wrapfigure}

\noindent\textbf{Membrane Fusion.} Ryham has contributed
physically relevant, state-of-the-art methods in the
area of \emph{membrane fusion} (see Figure \ref{fig:FusionMicroscopy}).
Membranes fuse as a part of many biological
processes such as synaptic transmission, intracellular trafficking,
fertilization, and viral infection, and  a major challenge
to determine the forces needed to merge two lipid bilayers
~\cite{chernomordik2008mechanics, kozlov1982possible, Kuzmin19062001, markin2002membrane, qian2012novel}.  
Studying fusion is such a difficult problem because bilayers
consist of two monolayer surfaces coupled by a pair of surface vector fields
that model the elongated lipids. 


In his much noticed Biophysical Journal
and Nature papers~\cite{RyKlYaCo16,Chetal16}, Ryham and collaborators 
calculated, for the first time, a least energy path for membrane fusion.  
One mathematical innovation of this paper was to combine
the underlying surface-director model  with the simplified string method 
introduced by Weinan E, Weiqing Ren, and Eric Vanden-Eijnden
~\cite{doi:10.1063/1.2720838}
to calculate the saddle point separating two energy basins. 
The results~\cite{RyKlYaCo16} yielded energy barriers for stages of fusion
consistent with predictions from prominent MD and experimental groups
\cite{SmRiMu19,2017PNAS..114.1238F}.
The PI's papers ~\cite{RyWaCo13} and ~\cite{RyKlYaCo16} 
were also coauthored with undergraduate
mathematics majors.

\begin{wrapfigure}[10]{r}{0.4\textwidth}
  \vspace{-5pt}
\centerline{\includegraphics[width=0.4\textwidth]{figures/SA1Figures/Hemifusion.pdf}}
  \vspace{-5pt}
\caption{\label{fig:Hemifusion}
Axisymmetric hemifusion stage of two planar bilayers~\cite{RyKlYaCo16}.
Bilayer energy depends not only on surface curvature, but lipid orientations as well.}
\end{wrapfigure}
Phase field methods have been used to
study fusion in two dimensions~\cite{C9SM01983A} (MORE REFS), 
but these models do not resolve lipid orientations that play a crucial
role in processes like fusion.

\subsection{Vesicles as self-organized granules}
Interface problems coming from biology are a challenging
area of applied mathematics.  Broadly speaking, these problems
involve a fluid structure interaction where the modeler must
solve for the velocity in the aqueous phase, track the interface,
and couple the velocity boundary condition to the interface through
stress balance boundary conditions.  The aqueous phase can
be a low Reynolds number fluid, a porous medium, or viscoelastic fluid,
and the interfacial forces can come from surface tension, elasticity,
or electrostatics.
One typically enforces constant volume constraints through a no-slip
boundary condition for the velocity; if the interface is area incompressible
then one can additionally require that the velocity surface divergence
be zero.  Incorporating all these details can lead to models with substantial
complexity.

\subsubsection{Mathematical innovations.}
We developed the model \eqref{eqn:stokes}-\eqref{eqn:hydro_stress} 
as a robust and flexible approach
that cuts across many of the aforementioned challenging aspects to vesicle modeling.
\begin{enumerate}
\item The granules are not fixed to a stencil which
affords greater flexibility in terms of vesicle morphology 
and topological changes--sufficiently large forces will pull the granules apart 
as physically required,
\item An elliptic PDE \eqref{eqn:phase} provides the attraction that binds granules together,
opening the possibility for mathematical analysis, 
\item The model includes important details like lipid orientations and inter-monolayer
slip not present in diffusive interface models, and
\item With the help of boundary integral equations
and fast summation methods, the dynamics can be solved in 
near-linear complexity in $N_b$, which is equal to or better than 
that of related sharp interface methods. 
\end{enumerate}
The main insight that leads a collection of granules to form bilayers
is to define a so-called ``amphiphilic'' boundary condition for \eqref{}
It takes the form:
\begin{equation}
\label{eq:amphiphilic_BC}
h_i(\mathbf{x}) = \tfrac{1}{2}\left(\dd_i \cdot (\xx - \aa_i)/c_i + 1\right)
\end{equation}
where $\dd_i$ is the director, $\aa_i$ is the center, and $c_i$ the
``radius'' of the $i$th granule.  Inspired
by the amphiphilic structure of lipids, 
\eqref{eq:amphiphilic_BC} 
takes values close to $1$ on in the direction $\dd_i$,
representing a hydrophobic tail, and values close to $0$ 
in the direction $\dd_i$ representing the hydrophilic head.
%\begin{figure}
%Wrapped figure: a picture of a lipid and the granule geometry.
%\end{figure}

\begin{wrapfigure}[13]{r}{0.5\textwidth}
  \vspace{-5pt}
\centerline{\includegraphics[width=0.5\textwidth]{figures/SA1Figures/AmphiphilicAssembly.pdf}}
  \vspace{-5pt}
\caption{\label{fig:amphiphilic_assembly}
Self-organization under amphiphilic boundary condition \eqref{eq:amphiphilic_BC}.
Granules first match hydrophobic tails (red) (a), then align length-wise (b). 
The behavior is the same in three dimensions 
as simulated by Kohl, Corona, and Veerapaneni
\cite{koh-cor-che-vee2021}.}
\end{wrapfigure}
The basic self-organization proceeds in two steps.
(a) Supensions of granules first minimize energy by matching hydrophobic
hydrophobic tails with one or more apposing granules
(Figure \ref{fig:amphiphilic_assembly}a, red regions.)
Once matched, there is still some residual energy in the
exposed sides.  In the minimimal configuration, pairs of
granules line up side by side, forming a bilayer,
and sequestering the hydrophobic
phase to the smallest possible region
(Figure \ref{fig:amphiphilic_assembly}b).
For numerical stability and physical realism,
we include a repulsion to prevent the granules from colliding.

\subsubsection{Do granule-based vesicle behave like continuum vesicles?}
A vesicle is a ring-shaped (or sphere-shaped in $\mathbb{R}^3$) 
and we analyze its energy as follows:
\begin{enumerate}
\item[Bending]
  The vesicle has bending energy because the directors are not
  parallel.  In the outer layer, the directors have negative
  splay $\nabla \cdot \dd < 0$ and in the inner layer
  the directors have positive splay $\nabla \cdot \dd > 0$.
  Due to minimality, the energy depends locally quadratically
  on the splay and has a bending modulus $k_B$.
  When $\dd$ is everywhere parallel to the unit normal $\NN,$ then
  \[
  \nabla \dd = -2H
  \]
  where $H$ is the mean curvature. 
\item[Stretch]
  If the vesicle is stretched, then the distance between
  neighboring granules is increased from rest.  The energy
  for this increase is Hookean and for small deformations
  equals
  \begin{equation}
    \label{eq:stretch}
    \tfrac{1}{2}k_A(L - L_0)^2/L_0
  \end{equation}
  where $k_A$ is a stretching modulus, $L$ and $L_0$ are stretched
  vesicle length and length at rest, respectively
  ($\tfrac{1}{2}k_A(A - A_0)^2/A_0$ in three dimensions).
  If the stretching is too great, the vesicle ruptures
  and the free eneds either rejoin, repairing the vesicle,
  of move apart forming a flat bilayer.
\item[Tilt]
  The directors in Figure \ref{} are normal to the vesicle surface.
  However, there are situations where the directors are not normal
  to the surface and this introduces the tilt angle $\theta$
  given by $\cos \theta = \dd \cdot \NN$.
  The energy for tilt is also Hooken and behaves like
  $k_{\theta}|\dd \times \NN|^2$ for the tilt modulus $k_{\theta}$.
\end{enumerate}

With regard to vesicle membranes and bilayers, two prominent 
classes of models researchers consider are the
Canham-Helfrich model and the more general Helfrich model.
In the Canham-Helfrich model, the membrane is assumed to have
negligible thickness and the lipids that form the membrane bilayer
are normal to the interface $\Sigma$.  The elastic energy takes the form
\begin{equation}
\label{eq:Canham-Helfrich}
  \int_{\Sigma} k_B(H - k_0)^2 + k_{\theta} \sigma\, dS 
\end{equation}
where $H$ is the mean curvature, $k_0$ is a spontaneous curvature,
$G$ is the Gaussian curvature, and $k_B$, $k_G$, $\sigma$, and $\gamma$ are
moduli for bending, saddle splay, surface tension, and line tension, respectively. 
If $\Sigma$ is a closed surface, then the integral
of Gaussian curvature is constant due to the Gauss-Bonnet Theorem.

This is our first step to showing the energy of the granule-based
vesicles have the same elastic energy as bilayers in the limit $N_b \to \infty$.
Once we have simulations for large systems sizes, we can provide
evidence that the two formulations are the same.
So and so has made simulations with thousands of granules in three-D
and this is now a technical/not theoretical limitation. 

\subsubsection{Do granule-based vesicles share hydrodynamic behavior of continuum vesicles?}

\begin{figure}R
  eview bend/stretch/tilt experiments
\end{figure}
\subsection{Vesicles in background flows}
\begin{itemize}
\item pull from JFM paper
\item we will consider background flows ...
\item results, rupture, critical shear rate
\item extensional flows
\item Taylor green flows; reorganization; do not know critical flow rate yet
\end{itemize}

\begin{remark}
  Mathematical problems: well-posedness of the system;
  Extensions to problems with transport;
  Calc-variational problem; place granules on fixed curve; study PDE system in
  SL -> 0; radius -> 0; Nb -> infty limit
\end{remark}
\begin{remark}
  In biophysics, tilt is defined by the tilt vector
  $\TT = \dd/\dd\cdot \NN - \NN$ so that the tilt energy density
  behaves like $\tan^2 \theta$ rather than $\sin^2 \theta$.
  These two energies densities are, however, asymptotically equal
  for small deformations.
Hamm and Kozlov [] and later Terzi and Deserno [], Pinigen et al. []
showed that the full Helfrich elastic energy of a \emph{monolayer} takes the form
\begin{equation}
\label{eq:Helfrich}
  \begin{aligned}
 &\int_{\Sigma}
  %\label{eq:Pinigan}
\tfrac{1}{2}k_{b}[(\nabla \cdot \mathbf{d} + k_{0})^{2} -  k^{2}_{0}]  
+ \tfrac{1}{2}k_{\theta}\mathbf{T}^{2} + \tfrac{1}{2}k_a[(\alpha - \alpha_0)^2 - \alpha_0^2] \\
&+ k_{c}\textbf{T} \cdot \nabla \nabla \cdot \mathbf{d}  + \tfrac{1}{2}k_{g}(\nabla \nabla \cdot \mathbf{d})^{2}
 + A\alpha \nabla \cdot \mathbf{d}
+ B \mathbf{T} \cdot \nabla \alpha \\
&- \tfrac{1}{2}k_c |\nabla \alpha|^2 + C \nabla \alpha \cdot \nabla \nabla \cdot \mathbf{d}\,dS
\end{aligned}
\end{equation}
where $\nabla$ and $\nabla \cdot$ are the surface gradient and surface divergence,
respectively, and $k_B$, $k_{\theta}$, $k_a$, $k_c$, $k_g$, $A$, $B$, $C$ are
appropriate elastic moduli.  An important and consequential distinction between
\eqref{eq:Canham-Helfrich} is the presence of the terms $\mathbf{T}$ and $\alpha$.
These are, respectively, the \emph{tilt} vector $\mathbf{T}$ measuring
the difference between the lipid director $\mathbf{d}$ and the surface normal
$\mathbf{N}$; the \emph{area per lipid} $\alpha$ area density at rest $\alpha_0$.
The two models \eqref{eq:Canham-Helfrich} and \eqref{eq:Helfrich} agree
when $\mathbf{T} = 0$ and $\alpha = \alpha_0$ everywhere.
\end{remark}



(Researchers Take from former background grant)

In biological processes like fusion and fission, however, the length scales of the
deformation (tens of nanometers) are comparable to membrane thickness (about 5 nm)
and so the detailed structure of bilayer cannot be ignored.
These models assign a surface $\Sigma$ to each of the two monolayers
of a bilayer, and a director field $\mathbf{n}:\Sigma \to \mathbb{R}^3$
for lipid orientation. Under very general continuum mechanical assumptions,

(Surface-director models).

The holy grail of biologically inspired fluid-interface models is
to have a formulation sufficient detailed and robust to make predictions
about biological processes.  Here, the Helfrich and Canham-Helfrich models
used throughout the field unfortunately fall short. For one,
realistic membranes are composed of a mixture of lipid species proteins.
Secondly, the membranes of living cells are constantly changing topology,
whether through fusion and fission with other membranes, or through
insertion or penetration by proteins and peptides.  Topological
changes are clumsy to incorporate into 
sharp interface formulations of the Helfrich and Canham-Helfrich energies
since computations rely on a parametrized surface of fixed topology.
Topological changes are permitted in phase field and level set formulations
of \eqref{eq:Canham-Helfrich}, but these methods smear out the 
details of lipid orientations, for example, that are important to the
process.  Recent work by [] and [] have incorporated lipid mixtures in \eqref{eq:Canham-Helfrich}
(although not \eqref{eq:Helfrich}), but these models have formidable complexity
due to the additional constraints and transport equations involved.

Nematic liquid crystals on surfaces is a related are of study.  These models,
which are closely related to \eqref{eq:Canham-Helfrich}, consider the energy
of a nematic liquid crystal (or Q-tensor ?) on a surface to study how the
surface geometry effects the director field and vice versa. Researchers
have formulated finite element methods for [][] and studied the existence
of minimizers []-[].  Liquid crystals are an extremely well-studied
are in mathematics [], where there are still a lot of open problems.

\begin{figure}
True bilayer; vesicle zero thickness; phase field; surface director
\end{figure}

We introduces \eqref{}-\eqref{} to overcome the challenges faced by
traditional Helfrich-type models of elastic membranes.

\noindent\textbf{Modeling vesicles} To make granular surfaces, we
form a collection of amphiphilic granules.  The granules have a
director $\mathbf{d} \in \mathbb{R}^n$ and the boundary condition
$g$ takes the form
\begin{equation}
g(\mathbf{x}) = \frac{1}{2}(\mathbf{d}\cdot(\mathbf{x}-\mathbf{a})/c + 1)
\end{equation}
where $c$ is granule radius.
This $g$ is smooth (real analytic even)
and takes values close to $1$ on the pole in the direction $\mathbf{d}$,
representing a hydrophobic face, and values close to $0$ on the pole
in the direction $-\mathbf{d}$ representing the hydrophilic face.
Then we define the well potential $f(\eta) = \tfrac{1}{2}\eta^2.$
This way, the hydrophilic side has the same phase as the bulk
($\eta = 0$, the minimum of $f$) and the hydrophobic side has
a higher potential energy phase.

Using \eqref{}-\eqref{}, suspensions of these granules self-organize
so that (i) apposing hydrophobic sides abut and (ii) granules align
side by side into sheets.  Vesicles are a stable configuration
of this process and the self-organization automatically expels or
absorbs the number of granules needed to minimize frustration between
the two leaflets.  But even for random initial configurations,
the granules form a collection of small vesicles and bilayer segments.
These disordered collections typically represent a local energy minimum 
because under a slight perturbation, the small vesicles and segments
further combine to form larger, stable vesicles.  


\subsubsection{Problem 1}

\subsubsection{Problem 2}

\subsubsection{Problem 3}


%\setcounter{page}{1}
%\addcontentsline{toc}{section}{Bibliography}
%\bibliographystyle{abbrvnat}
%%
%\bibliography{refs}
%
%
%
%\end{document}
