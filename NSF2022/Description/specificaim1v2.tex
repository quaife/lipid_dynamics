\section{Proposed Research}
\label{sec:proposed-work}

\subsection{Specific Aim 1: Membrane vesicles as self-organized granules}
\label{sec:specificaim1}
We introduced \eqref{eq:RBT}--\eqref{eqn:stressbalance} to complement
existing methods for studing vesicles and bilayers. 
The simulations results in \S \ref{sec:vesicles_as_granules}
show that two-dimensional, granule-based vesicles replicate the 
equilibria and hydrodynamics for continuous curves.
Having shown that the model is effective in terms of simulation,
our next goal is to study the model using rigorous analysis
and consider three-dimensionsional effects. 

The first step is to obtain some solid well-posedness results
for \eqref{eq:RBT}--\eqref{eqn:stressbalance}.
This system is an autonomous ordinary differential equation for
rigid body transformations $F_i$.  The right-hand side of this
equation is defined by a system of PDE.  In the single-well case,
\eqref{eqn:phase} is a linear equation
and for fixed $t$ there is a unique solution
($\phi, \uu$) for the interactions under appropriate
assumptions for the boundaries and boundary data
\cite{manasthesis,rac-gre2016,LAX}.
This leads to 
\begin{quotation}
  \textbf{Problem 1.} 
  Consider a collection the closed, disjoint granules
  $U_i$ with smooth boundary, with no-slip, rigid body
  motion boundary conditions, and a linear background flow.
  Assume a single-well potential and a smooth phase field
  boundary condition.  Study the well-posedness in
  time of the coupled system \eqref{eq:RBT}--\eqref{eqn:stressbalance}.
\end{quotation}
Problem 1 involves mathematically interesting techniques.
First, we must establish that the system has a local-in-time solution.
This involves domain perturbations and estimates for the
solutions of second order elliptic equations
\cite{Savar2002DomainPA, DANERS20081, Lamboley2015EstimatesOF},
which also find extensive application in extremal domain problems
\cite{Schiffer1954VariationOD, Henrot2006ExtremumPF,
  bogosel:hal-03607776,Bogosel2022OnTP}, for example.
For global-in-time solutions,
we can ensure positivity in time for the distance between granules
using the fluid mechanical principle of
lubrication forces \cite{cawthorn_balmforth_2010, leal_2007}.
Finally, we can describe the granule-flow
behavior in the limit zero screening length limit.

In the double-well potential case, \eqref{eqn:phase}
is a nonlinear Allen-Cahn
equation.
This situation models granules immersed in a two-phase fluid
(e.g. granules in an emulsion) 
where there is an interesting interplay
(1) the long-range alignment of granules with (2) 
the diffusive interface energy of the free boundary
between fluid phases.
On unbounded domains like $\Omega(t)$,
Allen-Cahn equations can have multiple of solutions
\cite{Alama1997StationaryLS,Alikakos2008OnAE,Bronsard1993OnTB,
  Byeon2014SolutionsOH, Byeon2013OnAP, Alessio2005ENTIRESI,
  Trumper2007ExistenceOA,Benci2019MultipleSF}
and so we must reformulate
\eqref{eqn:phase}.
A natural modification changes
\eqref{eqn:phase} to an initial value problem with
transport by the background fluid;
\begin{equation}
  \label{eqn:phase_transport}
  \phi_t + \uu \cdot \nabla
  = -\lambda \frac{\delta E}{\delta \phi}
  =  \lambda(\epsilon^2 \Delta \phi - W'(\phi)),
  \quad \phi(\xx,0) = \phi_0(\xx),
\end{equation}
for initial data $\phi_0$.
The modified systems enjoys energy law similar to \eqref{eq:energy_law} 
with an addtional dissipation term involving the square 
the Euler-Lagrange derivative $\frac{\delta E}{\delta \phi}$ and dampening factor $\lambda > 0$.
There are several works dealing with the
geometric characterization of the geometric evolution
of interfaces under the Allen-Cahn and Cahn-Hilliard equations
\cite{Christlieb2019CompetitionAC, Gavish2011CurvatureDF, Dai2019WeakSF,
  Promislow2017ExistenceBA, Dai2015CompetitiveGE, Promislow2012CriticalPO,
  Dai2022GeometricEO, Dai2020MinimizersFT,Dai2013GeometricEO,Promislow2022UndulatedBI} and
well-posedness when coupled to flow in static domains
\cite{Jiang2017TwophaseIF, Liu2012StrongSF, Giorgini2019WellPosednessOA,
Wu2022WellposednessOA, Gal2010AsymptoticBO,
Giorgini2020DiffuseIM,Giorgini2019UniquenessAR}.
The proposed research extends these results to
problems with moving domains.

A closely related and strongly developing direction in
the area of nematics liquid crystals concerns the interaction of
Landau de Gennes functionals with colloids, and studies
the qualitative structure of defects around single or multiple
coloidal particles
\cite{doi:10.1098/rsta.2020.0432, Alama2015MinimizersOT, Alama2021SaturnRD, PhysRevE.96.042702}.

The second problem considers the approximation
of equilibrium elastic energy of curves and surfaces,
and seeks to make rigorous the simulation results of
\S \ref{sec:vesicles_as_granules}.
\begin{quotation}
\textbf{Problem 2.} Place a collection of
two-dimensional amphiphilic granules along a curve $\mathcal{C}$.
Study the sharp interface limit of the energy $E$ 
as the number of granules goes to infinity and their size to zero.
\end{quotation}
Problems like these have been addressed for bulk nematic liquid
crystal potentials in the area of colloidal homogenization
\cite{Canevari2019DesignOE,doi:10.1137/18M1163919,doi:10.1137/18M1163919,
  BERLYAND200597,doi:10.1137/130910348},
and is closely related to 
dimension reduction for curved nematics confined to thin films
\cite{Golovaty2017DimensionRF, Golovaty2015DimensionRF,
doi:10.1142/S0218202516500470, FoFrLe07}.
Given the relatively complicated geometry consising
of a plane with a number of disjoint disks removed, 
we start by assuming that the directors are normal to the curve
and that the granules are all disks with radius proportional to $\epsilon$. 
The bending $k_B$ and stretching $k_A$ coefficients
depend on the granual size relative to their spacing.
We can then generalize the problem to account for aspect ratio of
elliptical granules and nonzero tilt. 

Finally, to consider three-dimensionsional effects, we
will generalize Problem 2 to \emph{numerically}
study the energies of amphiphilic granules confined to surfaces.
Recent work on nematics on a surface has studied
the interplay between 
surface geometry and a director field
\cite{Nestler2020PropertiesOS, Nitschke2018NematicLC,
  Nestler2018OrientationalOO, Nitschke2019HydrodynamicII,
  Nitschke2020LiquidCO}.
Researchers have formulated finite element methods for
\cite{Bartels2012FiniteEM, Nochetto2015NumericsFL,
  Nestler2019AFE}
  and studied minimizers
\cite{Segatti2014EquilibriumCO,Segatti2014AnalysisOA}.
On the other hand, in biophysics,
\cite{Hamm2000ElasticEO,Terzi2019CurvatureTiltTO,Terzi2019ACQ,Terzi2017NovelTC}
and most recently \cite{Pinigin2020NewCT}
derived an elastic energy for lipid bilayers,
the generalized Helfrich energy.
This energy involves the lipid director $\dd$,
tilt vector $\mathbf{T}$, and area per lipid $\alpha$,
and coupling terms and contains
the Helfrich elastic energy \eqref{eq:Canham-Helfrich}
as a special case.
\begin{quotation}
\textbf{Problem 3.}
Simulate surfaces made up of three-dimensional, amphilic granules.
Do you recover the generalized Helfrich elastic energy
or related Landau deGennes energies for large $N_b$ simulations?
\end{quotation}

The completion of Problems 1, 2, and especially 3 
requires additional algorithmic implementation including a
fast summation method such as the fast multipole method.
The implementation details, especially how we deal with large
granule numbers and three-dimensional systems, are
described in \S \ref{sec:specificaim2} Specific Aim 2.

PI Ryham has made fundamental contributions to interface modeling in
membrane biology.  His first works include
phase field formulations for the elastic bending energy \cite{0951-7715-18-3-016,Du05} and topological indicators \cite{DuEuler} of vesicle membranes
and fluid-vesicle interactions \cite{QiangDu09}.
In his much noticed Biophysical Journal
and Nature papers~\cite{RyKlYaCo16,Chetal16}, Ryham and collaborators 
calculated, for the first time, a least energy path for membrane fusion.  

%\begin{equation}
%\label{eq:Helfrich}
%  \begin{aligned}
% &\int_{\Sigma}
%  %\label{eq:Pinigan}
%\tfrac{1}{2}k_{b}[(\nabla_{\Sigma} \cdot \mathbf{d} + k_{0})^{2} -  k^{2}_{0}]  
%+ \tfrac{1}{2}k_{\theta}\mathbf{T}^{2} + \tfrac{1}{2}k_a[(\alpha - \alpha_0)^2 - \alpha_0^2] \\
%&+ k_{c}\textbf{T} \cdot \nabla_{\Sigma} \nabla_{\Sigma} \cdot \mathbf{d}  + \tfrac{1}{2}k_{g}(\nabla_{\Sigma} \nabla_{\Sigma} \cdot \mathbf{d})^{2}
% + A\alpha \nabla_{\Sigma} \cdot \mathbf{d}
%+ B \mathbf{T} \cdot \nabla_{\Sigma} \alpha \\
%&- \tfrac{1}{2}k_c |\nabla_{\Sigma} \alpha|^2 + C \nabla_{\Sigma} \alpha \cdot \nabla_{\Sigma} \nabla_{\Sigma} \cdot \mathbf{d}\,dS
%\end{aligned}
%\end{equation}
%in the lipid director $\dd$,
%tilt vector $\mathbf{T}$
%the difference between the lipid director $\mathbf{d}$ and the surface normal
%$\mathbf{N}$; area per lipid $\alpha$,
%and where $\nabla$ and $\nabla \cdot$ are the surface gradient and surface divergence, respectively. 
%The two models \eqref{eq:Canham-Helfrich} and \eqref{eq:Helfrich} agree
%when $\mathbf{T} = 0$ and $\alpha = \alpha_0$ everywhere.
%The remaining term $k_B$, $k_{\theta}$, $k_a$, $k_c$, $k_g$, $A$, $B$, $C$ are
%appropriate elastic moduli.




