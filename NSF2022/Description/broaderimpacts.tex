\section{Broader Impacts}
\label{sec:BroaderImpacts}
This project aims to advance the mathematical modeling of collective
dynamics of amphiphilic granules. The framework uses a new PDE-based
formulation that accounts for important and complex systems in soft
matter. These complex systems include optimal shape design in
metamaterials and fusion and fission of amphiphilic bilayer membranes.
The simulations will be performed with computational tools designed to
solve the governing PDEs both efficiently and with high-order accuracy.
The models describing granular systems could be transformative in
biomedicine and material science. The research draws from expertise in
scientific computing, physics of fluids, and mathematics. The
mathematical component incorporates variational techniques and offers
insight into fundamentals of self-organization and collective dynamics.
The project brings socially consequential research into the classroom
and offers undergraduates the opportunity to train alongside faculty and
graduate students. With its unique combination of mathematical modeling,
analysis, and scientific computing, the project highlights the potential
advancement of STEM from Applied Mathematics.

\subsection{Educational Impacts}
\label{subsec:Educational_plans}
To foster training in mathematical sciences, the proposal supports two
summer undergraduate researchers (URs) per year at Fordham University for
eight weeks. The URs will collaborate directly with the PIs,
receive tutorials (e.g., numerical methods), and be guided on
mathematical writing and presentation. Fordham Mathematics recently
completed a collaboration studio purpose-built for UR support. Work
products include a summary manuscript, a presentation in the Fordham
Undergraduate Research Symposium and departmental seminars, and
participation in a national conference. We have outlined three projects
for the present proposal:
\begin{enumerate}[noitemsep,topsep=0pt]
  \item Translate project code from MATLAB into a compiled language;
  \item Use machine learning and visualizations to post-process data of
    granule dynamics;
  \item Mathematical modeling of collective behavior in social sciences.
\end{enumerate}
PI Ryham has worked side-by-side with dozens of undergraduates and
high-schoolers in supported summer research, Girls in Science Technology
and Mathematics, and the Clare Boothe Luce Scholarship program and the
activity benefited their careers in machine learning, water resources
engineering, threat detection, and the NYC Department of Education.
Presently PI Ryham is mentoring a Clare Boothe Luce Scholar on polling
accessibility in New York City. The PI is finishing works on energy
densities for lipid bilayers with a recently graduated UR, now working
at Raytheon Technologies. His publications include works with seven
undergraduate coauthors~\cite{Figueroa2012CuttingCI, RYHAM20112929,
RyWaCo13, RyKlYaCo16}.

To increase the participation of women, persons with disabilities and
underrepresented minorities in STEM, the PIs will specifically target
students whose socio-economic background prevents them from
participating in out-of-state research experiences. The inclusion
objectives will be aided by the fact that Fordham's class of 2026 weighs
in with 63\% women and more than 46\% domestic students of color. PI
Ryham will solicit applicants from Fordham University's Collegiate
Science and Technology Entry Program. Students from the Bronx High
School of Science (PI Ryham) and Newark Science Park High School (PI
Young) will also be encouraged to join the research team. 

PI Young has been actively involved with promotion of underrepresented
students at New Jersey Institute of Technology. A former PhD student,
Herve Nganguia, is African American and now an Assistant Professor at
Towson University. PI Quaife is currently advising three female PhD
students, and one is African American and the recipient of the McKnight
Doctoral Fellowship which addresses the underrepresentation of African
American and Hispanic faculty at colleges and universities in Florida.
PI Quaife will also continue to work with undergraduates and high school
students through the Undergraduate Research Opportunity Program and
Young Scholar's Program at Florida State University.

Finally, we will create several modules to include in our courses. These
modules will introduce topics from numerical linear algebra and
optimization and highlight concrete applications of mathematics based on
the ideas of the proposal.

\section{Intellectual Merit}
\label{sec:IntellectualMerit}
This collaborative project focuses on modeling, simulation, and analysis
of self-assembly and collective dynamics of a suspension of granules
with non-uniform hydrophobicity on their boundaries. The main ingredient
is a nonlocal interaction through the solution of moving domain elliptic
PDEs that encompasses long-range (non-additive) amphiphilic and
short-range steric interactions. The PIs have validated this
coarse-grained model against well-studied vesicle hydrodynamics. The
technical research tasks include quantifying collective properties of
ensembles of granules with tunable hydrophobicity, analyzing the
mathematical model in distinguished limits to make connections with
continuum models, designing efficient and high-order numerical
algorithms for large-scale two- and three-dimensional simulations with
confinement, and developing a kinetic theory to quantify both the polar
and nematic characteristics in the collective dynamics.

