\section{Broader Impacts}
\label{sec:BroaderImpacts}
This project aims to advance the mathematical modeling of collective
dynamics of amphiphilic granules. The framework uses a new PDE-based
formulation that accounts for important and complex systems in soft
matter. These complex systems include optimal shape design in
metamaterials and fusion and fission of amphiphilic bilayer membranes.
The simulations will be performed with computational tools designed to
solve the governing PDEs both efficiently and with high-order accuracy.
The models describing granular systems could be transformative in
biomedicine and material science. The research draws from expertise in
scientific computing, physics of fluids, and mathematics. The
mathematical component incorporates leading variational techniques and
offers insight into fundamentals of self-organization and collective
dynamics. The project brings socially consequential research into the
classroom and offers undergraduates the opportunity to train alongside
faculty and graduate students. With its unique combination of
mathematical modeling, analysis, and scientific computing, the project
highlights the potential advancement of STEM from applied mathematics.


\subsection{Educational Impacts}
\label{subsec:Educational_plans}
To foster training in mathematical sciences, the proposal supports two
undergraduate researchers (URs) per year at Fordham University. PI Ryham
has worked side-by-side and published with undergraduate
coauthors~\cite{RYHAM20112929, RyWaCo13, RyKlYaCo16}. He is presently
mentoring a Clare Boothe Luce Scholar on mathematical analysis of trends
in election redistricting. With another Fordham supported UR, he is
finishing a paper on a closed-form energy density for translationally
invariant bilayers. Based on these experiences, we have outlined three
projects for the present proposal.
\todo[inline]{Rolf: Can you update this?}
\begin{enumerate}[noitemsep,topsep=0pt]
\item Translate project code from MATLAB into a compiled language 

\item Use machine learning to post-process data from the granules self-assembly

\item Mathematically investigate self-assembly in the zero-screen-length limit
\end{enumerate}
The UR support is for eight summer weeks. In addition to collaborating
with PI Ryham, they will have tutorials and receive training in
mathematical writing and presentation. The URs will have desks and
meeting room in a recently completed collaboration space built for the
Mathematics Department. As a condition of support, they will be required
to write a summary report and give a presentation in the university's
undergraduate research symposium. PI Ryham will encourage the URs to
participate in a national conference.

To meaningfully engage the community, the PIs will prioritize students
coming from underrepresented groups and specifically target students
whose socio-economic background prevents them from participating in
out-of-state research experiences. PI Ryham will solicit applicants from
Fordham University's Collegiate Science and Technology Entry Program.
Students from the Bronx High School of Science (PI Ryham) and Newark
Science Park High School (PI Young) will also be encouraged to join the
research team. 

PI Young has been actively involved with promotion of
underrepresented students at NJIT. One PhD student, Herve Nganguia, is
now an Assistant Professor at Towson University. PI Young has taught a
broad spectrum of courses in fluid mechanics and applied math modeling.
PI Quaife will work with undergraduates and high school
students through the Undergraduate Research Opportunity Program and
Young Scholar's Program, as he has done in the past.

%To maximize vertical integration, the PIs and personnel will travel to
%New York for one or two weeks during the summers. 
Finally, we will
create several modules to include in our courses. These modules will
introduce topics from numerical linear algebra and optimization and
highlight concrete applications of mathematics based on the ideas of the
proposal.

\section{Intellectual Merit}
\label{sec:IntellectualMerit}
This collaborative project focuses on modeling, simulation, and analysis
of self-assembly and collective dynamics of a suspension of granules with 
non-uniform hydrophobicity on their boundaries. The main
ingredient is a nonlocal interaction through the solution of moving
domain elliptic PDEs that encompasses long-range (non-additive) amphiphilic and
short-range steric interactions. The PIs have validated this
coarse-grained model against well-studied vesicle hydrodynamics. The
technical research tasks include quantifying collective properties of
ensembles of granules with tunable hydrophobicity, analyzing the mathematical model in 
distinguished limits to make connection with continuum models,
designing efficient and
high-order numerical algorithms for large-scale two- and
three-dimensional simulations with confinement, and developing a kinetic
theory to quantify the both the polar and nematic characteristics in the collective dynamics.
%
%The purpose of this research is to reach interesting physical phenomena
%with less computational cost than molecular dynamics, and account for
%more general features that continuum theory misses. The main ingredient
%is defining a nonlocal interaction through the solution of an elliptic
%boundary value problem that has the phenomenological characteristics of
%long-range hydrophobic attraction. This minimal model, while intuitive,
%is quite a general description of amphiphiles in solvent and gives rise
%to rich phenomena from Janus particle aggregates to correctly predicting
%elastic properties of bilayer. The technical research tasks include
%quantifying collective properties of amphiphilic ensembles, improving on
%mathematical models, efficient, high-order numerical algorithms for
%large-scale simulations, and incorporating external fields through
%electric charge.

