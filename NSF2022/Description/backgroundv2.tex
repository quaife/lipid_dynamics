\noindent
{\bf Collaborative research: Self-organization modeled by moving domain
elliptic PDEs} \\
{\em Rolf Ryham (PI, Fordham University),
Bryan Quaife (lead PI, Florida State University), and
Yuan-Nan Young (PI, New Jersey Institute of Technology)}

\section{Background}
\label{sec:background}

Mathematical models of self-organization encompasses a broad range of
interesting and challenging phenomena. For example, understanding the
organization of lipids and proteins in vesicles is crucial to design
effective drug and vaccine carriers. The COVID-19 pandemic has led to an
unprecedented and concerted worldwide effort in biophysical modeling of
the mechanisms of RNA condensation by structural proteins, protein
oligomerization, and cellular membrane–protein interactions that
ultimately control virus structure, in an effort to curtail pathogens
like the Severe Acute Respiratory Syndrome Coronavirus 2. Using powerful
capillary and van der Waals forces that exist at the nanoscale,
engineers have developed superior techniques to print high-resolution,
nanoscale features on devices like sensors, lasers, and LEDs. Finally,
researchers have introduced data-driven methods to study complex
patterns that emerge in particle and agent-based systems like swarms and
flocking.

A number of mathematical techniques have been developed to handle
self-organizing systems. Each technique has its advantages and
disadvantages. Prominent in chemistry and biology, molecular dynamics
(MD) simulations offers unparalleled resolution of protein-lipid
interactions, say, but because of the large number of non-linear
interactions involved, is limited to short time and small spatial
scales. In terms of inverse problems, agent based systems also involve
relatively large numbers of particles and researchers extensively
consider the problem of using trajectory data to learn interaction
kernels of the dynamics when constrained to evolve on Riemannian
manifolds. Once the interaction kernels are known, one can in principle
derive partial differential equations (PDE) for the macroscopic
behavior. 

The phase field method and self-consistent field theory are popular
approaches to self-organization in Applied Mathematics. Motivated by
Ginzburg-Landau theory, these methods introduce a scalar-valued phase
field function for the volume fraction of lipids and water in modeling
vesicles membranes, for example. Sometimes called diffuse interface
methods, one devises energy functionals involving the gradient and
well-potentials of the phase field, and possibly long range interaction
kernels, that yield approximate mathematical surfaces of almost
arbitrary topology. Because the fluid interfaces are given by the
level-set of a smooth function, the evolution equations can be studied
by well-established tools like variational methods and regularity
theory, and are amenable to numerical approximation. On the downside,
granular details like the shape changes of a protein, for example, are
clumsy to incorporate in the model. Moreover, numerical simulations of
phase field models typically discretize the entire fluid domain under
consideration, and apply cutoff boundary conditions when periodic
domains are not under consideration. 

\begin{figure}
  \begin{center}
%    \includegraphics[width=0.45\textwidth]{figures/SpecificAim1/Domain.jpg}
%    \includegraphics[width=0.45\textwidth]{figures/SpecificAim1/3Particles.jpg}
    \includegraphics[keepaspectratio,height=2.7cm]{figures/SpecificAim1/Domain.jpg}
    \includegraphics[keepaspectratio,height=2.7cm]{figures/SpecificAim1/3Particles.jpg}
  \end{center}
  \caption{\label{fig:flow_map} \footnotesize {\em Left:} In the HAP
  formulation~\cite{Fu2018_SIAM}, a system of exterior problems defines
  the hydrodynamic and hydrophobic interactions. {\em Right:} Particles
  translate and rotate to lower the free energy $E[u]$ subject
  to~\eqref{eqn:phase}, $W(\phi) = \phi^2$, and reach a minimal
  configuration in panel C. The colormap is blue for $\eta = 0$ and red
  for $eta = 1$. The arrows are the velocity field $\uu$.}
\end{figure}

\subsection{Problem formulation}
Recently, we have developed a new mathematical model of
self-organization~\cite{FuQuRyYo22,fu-ryh-qua-you2022,Fu2018_SIAM}. The
model takes ideas from the phase field method and leverages boundary
integral equations to resolve complex domains and speed up computations.
To formulate the problem, consider a collection of $N_b$-many rigid
bodies $U_i(t) \subset \RR^n$ with boundary $\Gamma_i(t)$. We call these
bodies ``granules''. The granules are immersed in an incompressible,
zero-Reynolds number fluid $\Omega(t) = \RR^n \setminus \cup_{i=1}^{N_b}
U_i(t)$ (Figure~\ref{fig:flow_map}).

We look for $N_b$-many rigid body transformations
\begin{align}
\label{eq:RBT}
  F_i(\XX,t) = R_i(t)(\XX - \aa_i(0)) + \aa_i(t),\quad \XX \in U_i(0),
\end{align}
where $i = 1,\ldots,N_b,$ $R_i(t)$ is a rotation (proper orthogonal) $n
\times n$ matrix, $\aa_i(t)$ are the granule centers, and $\XX$ is a
reference coordinate. The transformations~\eqref{eq:RBT} determine how
the granules move through space. Throughout, $\nnu$ refers to unit
normal pointing into $\Omega(t)$ (Figure~\ref{fig:flow_map}).

The key idea is to define the dynamics using the following boundary
value problem system:
\begin{alignat}{4}
  \label{eqn:stokes} 
  -\mu\Delta \uu + \nabla p &= \mathbf{0}, 
  \quad \nabla \cdot \uu = 0, &&\xx \in \Omega(t), \\
  \label{eqn:phase}
  -\epsilon^2 \Delta \phi + W'(\phi) &= 0, &&\xx \in \Omega(t),\\
  \label{eqn:noslip}        
  \frac{dF_i}{dt}(\XX,t) & = \uu(\xx) = 
    \vv_i + \omega_i \times (\xx - \aa_i), 
  \quad &&\xx \in \Gamma_i(t),\\
  \label{eqn:material}
  \phi(\xx) &= h_i(\XX),  &&\xx \in \Gamma_i(t),
\end{alignat}
for $i=1,\ldots,N_b$ and where $\xx = F_i(\XX,t)$.  Finally,
\begin{align}
\label{eqn:stressbalance}
\int_{\Gamma_i} \left(\sigma  + T_i\right)\nnu \,\dif s = \mathbf{0},\quad
\int_{\Gamma_i} (\xx - \aa_i)\times \left(\sigma + T_i\right) \nnu \,\dif s = \mathbf{0}
\end{align}
where
\begin{align}
\label{eqn:hydro_stress}
\sigma = \mu(\nabla \uu + \nabla \uu^T) - pI,\quad 
T_i = \gamma\left[\epsilon^{-1} W'(\phi)I
  + \epsilon\left(\tfrac{1}{2}|\nabla \phi|^2I - \nabla \phi \nabla
  \phi^T\right)\right]
\end{align}
are the hydrodynamic and phase stresses, respectively; $I$ is the
$n\times n$ identity matrix. For vectors $\aa, \bb \in \mathbb{R}^2$,
we follow the usual convention of defining the ``cross product'' as $\aa
\times \bb = (\aa^{\perp} \cdot \bb) \kk$ where $\kk = (0,0,1) \in
\RR^3$ and $(a_1,a_2)^{\perp} = (-a_2,a_1)$.

The above form a semi-linear elliptic system for evolution of the
granules. Equation~\eqref{eqn:stokes} is the Stokes equations that
describes the velocity $\uu$ and pressure $p$ of the aqueous region with
viscosity $\mu$. Equation~\eqref{eqn:phase} describes the phase
transitions of the scalar order parameter $\phi$ with a decay length
$\epsilon$. The scalar function $W(\phi)$ is a nonnegative well
potential where the local minima represent metastable states $\phi_0,
\phi_1,\ldots$ of the water phase. To have finite bulk energy, the
minimum potential phase $\phi_0$ has $W(\phi_0) = 0$. Mathematically,
this order parameter is responsible for the attraction and adhesion
between granules that want to align boundaries of like phase.

The next three equations are the moving domain boundary conditions.
Equations~\eqref{eqn:noslip} specify a no-slip boundary condition for a
rigid body with translation velocity $\aa'_i(t) = \vv_i$ and angular
velocity $\omega_i$ about the point $\aa_i(t) \in U_i(t)$. ($\omega_i$
is the axial vector of the skew symmetric matrix $R'_i(t)$.) In
equation~\eqref{eqn:material}, the function $h_i$ is a material label
specifying the phases at water-granule interface. The boundary condition
for $\phi$ is transported by the flow and satisfies $\phi_t + \uu \cdot
\nabla \phi = 0$ on $\Gamma_i$. Finally,
equation~\eqref{eqn:stressbalance} closes the system by requiring that
the force and torque generated by hydrodynamic stress balances that
generated by phase stress $T_i$ in~\eqref{eqn:hydro_stress}. The
parameter $\gamma > 0$ is surface tension. 

We call the rigid bodies $U_i(t)$ granules since they are small, compact
parcels of a suspended media, e.g.~lipids, Janus spheres. We use the
shape and boundary conditions for $U_i$ to capture the local properties
of the suspended media. We have departed from the terminology
``particles'' to emphasize the role played by their finite size. In our
simulations (see \S\ref{sec:specific_aim1}), we include a finite length
repulsion to avoid particle collisions.

\begin{wrapfigure}[]{l}{2.1in}
  \centering
  \includegraphics[width=1in]{figures/Background/Peanut/PeanutFEM.pdf}
  \includegraphics[width=1in]{figures/Background/Peanut/PeanutIE.pdf}
  \caption{\label{fig:fem_vs_bie} \footnotesize A finite element method
  ({\em left}) requires meshing the entire domain (typically into
  triangles), while a BIE ({\em right}) requires meshing only the
  boundary of the domain.}
\end{wrapfigure}

Computational methods to solve the PDEs~\eqref{eqn:stokes}
and~\eqref{eqn:phase} are challenging to develop because of the complex
and unbounded domain. This challenge is compounded when dense
suspensions are considered. Integral equations (IEs) offer an
alternative formulation of the PDEs that only require meshing the
boundary of the granules rather than all of $\Omega(t)$
(Figure~\ref{fig:fem_vs_bie}), and the automatically satisfy far field
conditions. Additional advantages of an discretizations of IE
formulations include high-order, or even spectral, accuracy and
well-conditioned linear system that can be solved iteratively with a
mesh-independent number of Krylov iterations such as the generalized
minimal residual method (GMRES). However, IEs have challenges: nearly
touching granules requires a quadrature rule for nearly-singular
integrals; their discretization results in dense matrices; and they
cannot immediately be applied to non-linear PDEs. The IE community,
including PI Quaife, has developed a variety of tools to address these
numerical challenges, and we discuss our planned approach in
Section~\ref{sec:specificaim2}.

In the absence of external flow, the
system~\eqref{eq:RBT}--\eqref{eqn:stressbalance} has the energy law
\begin{align}
\label{eq:energy_law}
  \frac{d}{dt}E
  = - \int_{\Omega(t)}\tfrac{1}{2}\mu|\nabla \uu + \nabla
  \uu^T|^2 \,d\xx,\quad
    E = \gamma \int_{\Omega(t)}
  \frac{\epsilon}{2} |\nabla \phi|^2 + \frac{1}{\epsilon} W(\phi) \,d\xx.
\end{align}
Here $E$ is the energy stored in the phase transitions. Mathematically,
the energy law states that changes in granule configuration
instantaneously change the phase energy $E$. The dynamics are such that
the change energy is lost due to hydrodynamic dissipation. Further
details and generalizations are provided in \S\ref{sec:specific_aim1}
and \S\ref{subsec:specific_aim_3}. Because $E$ depends both on $\phi$
and $\Omega(t)$, the derivation of~\eqref{eq:energy_law} is somewhat
subtle and has been worked out in detail in~\cite{Fu2018_SIAM}. The
calculation uses the Rayleigh Transport Theorem to write the time
derivative of $E$ as an energy density flux involving the phase stress
$T_i$ on the boundary, and a bulk term involving the Euler-Lagrange
derivative of $E$. The Euler-Lagrange derivative vanishes due
to~\eqref{eqn:phase} and the boundary flux transforms into viscous
dissipation due to~\eqref{eqn:hydro_stress} and the Stokes equations. 

%It goes as follows. Using the Rayleigh Transport Theorem, the time
%derivative out $E$ yields an energy density flux involving the phase
%stress $T_i$ at the boundary, plus a bulk term involving the
%Euler-Lagrange derivative of $E$.

In the far field, $\lim_{\xx \to \infty} \phi(\xx) = \phi_0$ and for the
background flow, we require that $\lim_{\xx \to \infty} \uu(\xx) -
\uu_{\infty}(\xx) = \mathbf{0}$. In the case of shear background flow,
for example,
\begin{align}
\label{eqn:shear_BG_flow}
\uu_{\infty}(\xx) = \dot{\gamma} \xx \cdot \mathbf{e}_y \mathbf{e}_x,
\end{align}
where $\dot \gamma$ is the shear rate and $\mathbf{e}_x$, $\mathbf{e}_y$
are horizontal and vertical basis vectors, respectively. Other
background flows include extensional, parabolic Poiseuille, and
Taylor-Green flow. For numerical stability and physical realism, we
introduce pairwise repulsive forces and torques
in~\eqref{eqn:stressbalance} to prevent the collision between
neighboring granules. We include the repulsive potential in the energy
law~\eqref{eq:energy_law}. A background flow adds elastic energy to the
granule configurations and is quantified by simulation. We refer to the
PIs work~\cite{FuQuRyYo22, fu-ryh-qua-you2022, Fu2018_SIAM} for details. 

In terms of physical justification, the derivation of the
one-dimensional version of our model was carried out using a Landau
expansion of the free energy density for a single-well
potential~\cite{MaRa76, ErLjCl89}. The double-well potential case was
considered by Gompper~\cite{GoHaKo94} to describe metastable
coordination of water. In the field of surface
chemistry~\cite{Israelachvili1954}, there is a large empirical
literature on properties of long-range hydrophobic
interaction~\cite{LeRaPa77, KoNa15, Nagle17, Lum1999, Lin2005,
Meyer2006, Ducker2016, Jackson2016, Gletal88, Aketal17, Ch05}. The
stress~\eqref{eqn:stressbalance} was first identified by PIs Ryham and
Young in~\cite{Fu2018_SIAM} and generalizes one-dimensional hydrophobic
interaction to the $n$-dimensional setting. 

Finally, in terms of mathematical justifications, we mention that in the
single-well case $W(\phi) = \tfrac{1}{2}\phi^2$, equation
\eqref{eqn:phase} takes the form of a screened Laplace equation
$-\epsilon^2 \Delta \phi + \phi =0$. For a fixed domain $\Omega$ with
class $\mathbf{C}^2$ boundary~\cite{}, if the material label $h_i = 1$
everywhere, then $\phi$ has a boundary layer going from $1$ at the
boundary to $0$ in the bulk and one can prove that 
\begin{align}
\label{eqn:math_motivation}
\lim_{\epsilon \to 0} E = \frac{\gamma}{2} \int_{\partial \Omega} 1 \,dS.
\end{align}
In other words, the phase energy can be viewed as an approximation of
surface energy that ``extends'' into the bulk. In the general case,
$h_i^2$ replaces $1$ in~\eqref{eqn:math_motivation}. We are confident
that~\eqref{eq:RBT}--\eqref{eqn:stressbalance} is a correct description
of long-range interactions between granule surfaces.

\begin{wrapfigure}[23]{r}{3.2in}
  \vspace{-15pt}
  \begin{center}
    \begin{tabular}{m{0.9in}m{0.9in}m{0.9in}}                 
    \includegraphics[width=0.85in]{figures/SpecificAim1/N100B1.jpg}
    &\includegraphics[width=0.85in]{figures/SpecificAim1/N100B2.jpg}
     &\includegraphics[width=0.85in]{figures/SpecificAim1/N100B3.jpg}    \\
    \includegraphics[width=0.85in]{figures/SpecificAim1/N100C1.jpg}
    &\includegraphics[width=0.85in]{figures/SpecificAim1/N100C2.jpg}
      &\includegraphics[width=0.85in]{figures/SpecificAim1/N100C3.jpg}    \\
      \includegraphics[width=0.85in]{figures/SpecificAim1/N100A1.jpg}
      &\includegraphics[width=0.85in]{figures/SpecificAim1/N100A2.jpg}
      &\includegraphics[width=0.85in]{figures/SpecificAim1/N100A3.jpg} 
  \end{tabular}
  \end{center}
  \vspace{-5pt}
  \caption{\footnotesize \label{fig:self-assembly2} Rows (a)--(c) have
  the same initial configuration but different boundary conditions $g$.
  {Top Row:} Granules with an amphiphilic boundary condition
  results in a bilayer pattern shielding the hydrophobic core (yellow,
  $\eta > 0$) from the aqueous phase (blue, $\eta = 0$). {\em Middle
  Row:} Biased granules with a hydrophobic intensity that is greater on
  one side of the particle (magenta) than the other (yellow). {\em
  Bottom Row:} Bipolar granules where green is for $\eta < 0$, yellow is
  for $\eta > 0$, and blue is for $\eta = 0$. Particles orient like
  sides to form a checkerboard pattern.}
\end{wrapfigure}

\subsection{Applications}
Since the formulation is scale independent, it can be applied broad
range of nonspecific self-organization phenomena in surface chemistry,
biology and engineering. For one, the
formulation~\eqref{eq:RBT}--\eqref{eqn:stressbalance} captures exactly
the right behavior expected from amphiphilic self-assembly. For example,
three bodies with amphiphilic label (\S\ref{sec:specific_aim1})
spontaneously self-organize to sequester the hydrophobic fluid phase
(the red phase in Figure~\ref{fig:flow_map}).  Moreover, a simple change
of boundary conditions from amphiphilic, to biased hydrophobic, to polar
results in distinct equilibrium morphologies
(Figure~\ref{fig:self-assembly2}). (i) Chemistry: Janus spheres Since
the formulation is scale independent, it can represent a suspension of
amphiphiles, where the granules represent the mean properties of a small
collection of lipids, at the nanoscopic scale. For larger scales, the
representation is literal and the granules Janus spheres. (ii) Pickering
emulsion (iii) \todo[inline]{Bertozzi's problem} (iv) Tension driven
microprinting (v) Vesicle hydrodynamics.


\subsection{Vesicles as self-organized granules}
\label{sec:vesicles_as_granules}
Interface problems coming from biology are a challenging
area of applied mathematics.  Broadly speaking, these problems
involve a fluid structure interaction where the modeler must
solve for the velocity in the aqueous phase, track the interface,
and couple the velocity boundary condition to the interface through
stress balance boundary conditions.  The aqueous phase can
be a low Reynolds-number fluid, a porous medium, or viscoelastic fluid,
and the interfacial forces can come from surface tension, elasticity,
or electrostatics.
Incorporating all these details can lead to models with substantial
complexity and there is a need for robust, general models that
cut across these issues. 

This section shows the principles of PDE-driven
self-organization apply to the concrete
problem of \emph{vesicle hydrodynamics}.
A vesicle is a small, fluid filled membrane sack.
Microscopically, the membrane of a vesicle is comprised
elongated amphiphilic lipids that form inner and outer monolayers.  
The goal in vesicle hydrodynamics is to study the flow patterns e.g.,
tank-treading, tumbling, that arrise due to the interaction
between membrane elasticity and surrounding fluid.
Traditional models represent a vesicle as a
continuous surface (or curve in $\mathbb{R}^2$) $\Sigma$.
The elastic energy takes the form
\begin{equation}
\label{eq:Canham-Helfrich}
  \int_{\Sigma} k_B(H - k_0)^2\, dS 
\end{equation}
where $H$ is the mean curvature, $k_0$ is a spontaneous curvature,
and $k_B$ is a bending modulus. 
The variation of \eqref{eq:Canham-Helfrich} with respect to position
gives rise to a surface force
that pointwise balances the jump in hydrodynamic stress across the interace.
While energy density does depend on Gaussian curvature,
it does not appear in \eqref{eq:Canham-Helfrich}
since Gauss-Bonnet Theorem guarantees that
the integral of Gaussian curvature over a closed surface is independent of the shape of $\Sigma$.

Depending on the situation,
one can further enforce constant volume constraints or permeability relationships.
If the interface is area incompressible, then the surface divergence
of velocity is zero.  On the other hand, if the surface is area compressible,
then \eqref{eq:Canham-Helfrich} contains an additional stretching term of the
form $k_A(\alpha - \alpha_0)^2/\alpha_0^2$ where $\alpha$ and $\alpha_0$ are the
deformed and equilibrium area densitities respectively. 

There are downsides to traditional models in terms of generality and physical realism.
(i) Motivated by drug delivery applications, experimenters use electric
fields, for example, to controllably open and close pores in membranes
\cite{}.
These kinds of topological changes are clumsy to incorporate in
sharp interface formulations that assume a continuous surface \cite{}.
(ii) Sharp-interface models assume a membrane of neglible thickness. 
However, in biological processes
like fusion, fission, or insertion and penetration by proteins and peptides, 
the length scales of the
deformation (tens of nanometers) are comparable to membrane thickness (about 5 nm).
Moreover, these processes crucially rely on variable lipid
orientation relative to the surface normal (the so-called \emph{tilt}
deformation),
whereas \eqref{eq:Canham-Helfrich} implicitly assumes that the lipids
are everywhere normal to the surface.  
(iii) Finally, realistic membranes are comprised of a mixtures of lipids and proteins.
Recent work \cite{} has incorporated lipid mixtures in \eqref{eq:Canham-Helfrich},
but these models are formidably complicated 
due to the additional constraints and transport equations involved.

\begin{wrapfigure}[12]{r}{0.5\textwidth}
  \vspace{-5pt}
\centerline{\includegraphics[width=0.5\textwidth]{figures/SA1Figures/AmphiphilicAssembly.pdf}}
  \vspace{-5pt}
\caption{\label{fig:amphiphilic_assembly}
Self-organization under amphiphilic boundary condition.
(a) Granules first match hydrophobic tails (red),
(b) then align length-wise.
The principle of self-organization holds in three dimensions 
as simulated by Kohl, Corona, and Veerapaneni
\cite{koh-cor-che-vee2021}.}
\end{wrapfigure}
Diffusive interface models and surface-director models partially address 
these shortcomings. 
Topological changes are permitted in phase field and level set formulations,
but these methods smear out the details of lipid orientations.
The detailed energies of lipid orientation that include tilt are
closely related to the celebraated and well-studied
field of nematic liquid crystals.  Here, there have been recent development
in simulation and theory of nematic liquid crystals on surfaces.
These models consider the interplay between defects and surface geometry \cite{},
finite element formulations \cite{}, and existence of minimizers \cite{}.

\subsubsection{Main modeling innovations}
We developed the model \eqref{eq:RBT}-\eqref{eqn:hydro_stress} 
as a robust and flexible approach
that cuts across many of the aforementioned challenges.
(i) The granules are not fixed to a stencil, 
affording flexibility in terms of vesicle morphology and topology--sufficiently
large forces will even pull the granules apart as physically required.
(ii) The model formulation is built around important details like lipid orientations,
and exhibits inter-monolayer slip that must be added into diffusive interface models
post hoc. 
(iii) It is straightforward to consider mixtures of granules with different
types of shapes and boundary conditions of the granules.
With the help of boundary integral equations
and fast summation methods, the dynamics can be solved in 
near-linear complexity in $N_b$, which is equal to or better than 
the computational complexity of related sharp interface methods. 

Inspired
by the amphiphilic structure of lipids,
we formulated a so-called ``amphiphilic'' boundary condition
that leads a collection of granules to form bilayers.
Referring to Figure \ref{fig:amphiphilic_assembly}, 
\begin{equation}
\label{eq:amphiphilic_BC}
\phi(\mathbf{x}) = h_i(\mathbf{x}) = \tfrac{1}{2}\left(\dd_i \cdot (\xx - \aa_i)/c_i + 1\right), \quad
\mathbf{x} \in \Gamma_i,
\end{equation}
where $\dd_i$ is the director, $\aa_i$ is the center, and $c_i$ the
``radius'' of the $i$th granule.  
The phase $\phi$ takes values close to $1$ on in the direction $\dd_i$,
representing a hydrophobic tail, and values close to $0$ 
in the direction $\dd_i$ representing the hydrophilic head.

\begin{figure}[t!]
\begin{center}
\includegraphics[width=0.9\textwidth]{figures/Background/coarsening.pdf}
\end{center}
\vspace{-20pt}
\caption{\label{fig:coarsening}
Modeling procedure.
We initilize a suspension of granules.
Under the moving domain problem
\eqref{eq:RBT}-\eqref{eqn:hydro_stress}
with amphiphilic boundary condition, the granules
self-organize into vesicle bilayers.
We then introduce external hydrodynamic effects to study elastic properties.
}
\vspace{5pt}
\end{figure}
\subsubsection{How do granule-based vesicles have an elastic energy?}
A natural question to ask is how 
elastic energy enters a formulation
coming from surface interactions like \eqref{eqn:math_motivation}.
Conceptually, random suspensions of amphiphilic
granules undergo a rapid coarsening to form bilayers.
For tens of granule, the bilayers form a single stable, ring-shaped vesicle
with energy $E_{\text{stable}}$ (Figure \ref{fig:coarsening}, (c)).
This vesicle configuration, however, is only a local minimizer,
whereas the planar configuration (Figure \ref{fig:coarsening}, (e))
is the global minimizer.  The difference between
$E(t)$ and $E_{\min}$ is the vesicle elastic energy.
Indeed, if a background flow overcomes the energy barrier to force granules apart,
then the bilayer unfolds from its circular shape into a flat shape.  

When hundreds granules are involved,
the two-dimensional granules reach an intermediate
state consisting of several bilayer segments (see Figure \ref{fig:self-assembly2}, top right).
Placing this metastable state into a mild background flow
causes the segments to join up into isolated vesicles \cite{fu-ryh-qua-you2022}.

Having a non-specific model that replicates detailed
physics and biology is the crown jewel of applied mathematics
and we believe that ours is such a model. 

\begin{wrapfigure}[11]{r}{0.5\textwidth}
\includegraphics[width=0.5\textwidth]{figures/PreliminaryWork/TankTreading.jpg}
\caption{\label{fig:JPv_linearshear}Granular vesicles undergo
tank-treading in background shear flow.}
\end{wrapfigure}
\textit{Summary of results from the paper \cite{Fu2018_SIAM}.}
A striking aspect of the
model is its fidelity to physics.
The model contains a few
parameters; granule radius $c$,
screening length $\epsilon$,
surface tension $\gamma$.
When set to realistic values,
\cite{Fu2018_SIAM, ErLjCl89, Lin2005, Parsegian, Israelachvili80, GarciaSaez, KUZMIN2005, Petelska2012,Jackson2016},
we recover many of the well-known
elastic moduli for single-component lipid bilayers
\cite{Nagle17, Nagle17-2, LeVeWa14,NAGLE2000159}.
\emph{Bending:}
The bending deformation enters through the splay of the
granule directors.  In Figure \ref{fig:JPv_linearshear},
for example, the directors have negative, respectively positive,
splay $\nabla_{\Sigma} \cdot \dd$
in the outer, respectively inner monolayers.
Here $\nabla_{\Sigma}\cdot{} = \nabla \cdot {} - \mathbf{N}\mathbf{N}^T \nabla$
is the surface divergence for the unit normal $\NN$
and the splay energy density
for the granule configuration is Hookean with bending modulus $k_B$.
It is a fact from differential geometry that
when $\dd$ is everywhere parallel to $\NN,$
then $\nabla\cdot \dd = \pm 2H$ and this is how the
curvature term arises for \eqref{eq:Canham-Helfrich}
($2H = \kappa$ in the two-dimensional setting).
In the case of mixtures of granules of different sizes,
spontaneous curvature $k_0$ arises
from asymmetry in the monolayer leaflets.
\emph{Stretching.}
When stretched, the distance between
neighboring granules is increased from rest
and the energy for stretching behaves as 
$\tfrac{1}{2}k_A(L - L_0)^2/L_0$
for a stretching modulus $k_A$,
length $L$, and
length at rest $L_0$ of the two-dimensional vesicle. 
\emph{Tilt.}
Finally, the granules possess a 
tilt decay length scale $\lambda = \sqrt{k_B/k_{\theta}}$
well-known from the biophysical literature
for the case when the directors $\mathbf{d}$ are no longer normal
to the surface
\cite{KUZMIN2005}.

\begin{wrapfigure}[14]{l}{0.475\textwidth}
%\vspace{-10pt}
\includegraphics[width=0.475\textwidth]{figures/PreliminaryWork/Rupture.jpg}
\caption{\label{fig:JPv_rupture}Rupture of a granular vesicle under large shear rates.}
\end{wrapfigure}
\textit{Summary of results from the paper \cite{FuQuRyYo22}.}
In terms of hydrodynamics, the granule vesicles in linear
background flows \eqref{eqn:shear_BG_flow} also replicates
the behaviour of an continous, inextensible,
elastic vesicle with permeability.
The simulations showed, for the first time,
a granule-vesicle suspension behaving
as a tank-treading vesicle \cite{Finken2008, Shaqfeh11}
(see Figure \ref{fig:JPv_linearshear}).
Movies of Figure \ref{fig:JPv_linearshear} reveal
intermonolayer slip and derived values for intermonolayer
friction were in good quantitative agreement 
those derived by atomistic and Martini force field
simulation studies \cite{WuoEd06, denOtter2007, SHKULIPA2005823, Zgorski2019}.
In Figure \ref{fig:JPv_rupture},
membrane rupture occurs at large shear rates $\dot \gamma$
and we derived a dimensional scaling to 
predict critical shear rate for rupture
\cite{VLAHOVSKA2009775,keller_skalak_1982}.
Finally, a side-by-side comparison
of granule-based vesicles and the continuous model using
\eqref{eq:Canham-Helfrich} yielded shapes and trajectories that
basically overlapped.


\subsection{Outline of the proposal}
The outline of the proposal goes as follows.
Section \ref{sec:resume} describes the
PIs's expertise and prior works
in (1) membrane-fluid mechanics (2) boundary integral equations
and (3) complex fluid dynamics
that are critical to the success of this proposal. 
Section \ref{sec:proposed-work}
goes over the three specific aims of the proposal.

