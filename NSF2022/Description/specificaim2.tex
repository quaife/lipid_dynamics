\subsection{Specific Aim 2: Efficient, high-order methods for
large-scale simulations}
% ----------------------------------------------------------------------
Over the past decade, there has been an explosion of interest in
small-scale processes that utilize capillary forces and van der Waals
interactions to coordinate movement and bind microscopic components in
solvent~\cite{Pandey2011, Zhang2017, Siontorou2017}. Additionally,
hydrodynamic interactions are central to fabricate complex,
three-dimensional microstructures~\cite{Dasgupta2017, Leong2007,
Reynolds2019, Cho2010}. Hydrodynamic effects cannot be ignored since
they, like for the rates of biological functions like pore
dynamics~\cite{RYHAM20112929}, set the rate of dissipation. In Specific
Aim 2, we propose novel and efficient high-order numerical methods to
simulate dynamic assembly of particles under the HAP in a viscous
solvent. As described in~\eqref{eq:stokes} and illustrated in
Figure~\ref{fig:flow_map}, the solvent phase is modeled by the Stokes
equations for an incompressible, viscous fluid, and these equations are
coupled to the screened-Laplace equation~\eqref{eq:SL} through viscous
and hydrophobic stress balance. 
 
The HAP model requires solving exterior problems for the order
parameter~\eqref{eq:SL} and the Stokes equations~\eqref{eq:stokes}. We
will develop methods based on combinations of volume and layer
potentials, and inexact Newton's method. Numerical methods to solve
these integral equations (IEs) have several advantages over their
PDE-based counterpart: by construction, they satisfy far-field and mass
conservation conditions; only $\partial\Omega$ needs to be discretized
for layer potentials; carefully chosen IEs result in a well-conditioned
linear system that is solved with a mesh-independent number of GMRES
iterations; and high-order or spectral accuracy is attainable with
appropriate quadrature methods. The PIs have demonstrated that IEs are a
powerful tool to simulate two-dimensional suspensions of amphiphilic
particles~\cite{Fu2018_SIAM, FuQuRyYo20}. Moreover, elements of these
formulations have been described in three dimensions~\cite{ying_2006,
manasthesis, rac-gre2016}. This proposal will extend the two-dimensional
methods to more general amphiphilic particle models that
satisfy~\eqref{eq:SL}, and all the methods can naturally be extended to
three dimensions. These methods will be motivated by models and analysis
developed in Specific Aims 1 and 3. \S\ref{subsec:AC} describes how the
non-linear Allen-Cahn equation~\eqref{eq:SL} will be iteratively solved
using an IE formulation. Then, \S\ref{subsec:NumericalIssues} describes
how we will combine preconditioning strategies with fast summation
methods to accelerate the time to solution, and
\S\ref{subsec:timeStepping} describes adaptive time stepping methods.

\subsubsection{Iterative methods for the Allen-Cahn HAP model}
\label{subsec:AC}

The PIs have developed numerical methods to solve~\eqref{eq:SL} when
$f(u) = u^2$---this choice results in the linear screened Laplace
equation. For a more general $f$, such as a double well potential, we
will interpret the solution of~\eqref{eq:SL} as the root finding problem
for
\begin{align}
  \label{eq:F}
  J[u] = -\rho^2 \Delta u + \tfrac{1}{2}f'(u).
\end{align}
This non-linear PDE will be solved with Newton's method
\begin{align*}
  u^{N+1} = u^{N} - \alpha (dJ^N)^{-1} J(u^N),
\end{align*}
where $dJ^N = -\rho^2 \Delta + \tfrac{1}{2}f''(u^N)$ is the Fr\'{e}chet derivative
of $F$, and $\alpha$ is the step size which will be determined using
backtracking. As expected, the biggest computational challenge is
solving $- \rho^2 \Delta v + \tfrac{1}{2}f''(u^{N}) v = J[u^N]$. Solving this
variable coefficient PDE with an integral equation is challenging since
a fundamental solution is not readily available. We propose to instead
use an inexact Newton's method where $dJ^{-1}$ is defined in terms of
the continuous, piecewise quadratic approximation
\begin{align*}
  f(u) \approx \tilde{f}(u) = bu + \begin{cases}
    a(u - u_0)^2, &\mbox{if } u \leq (u_0 + u_1)/2, \\
    (u - u_1)^2 + \tfrac{1}{4}(a-1)(u_0-u_1)^2 , &\mbox{if } u > (u_0 + u_1)/2,
  \end{cases}
\end{align*}
where $a>0$, and the resulting inexact Newton's method requires solving
the piecewise constant PDE
\begin{align}
  \label{eq:screenedPoisson}
  -\rho^2 \Delta v + \tfrac{1}{2}\tilde{f}''(u^{N})v = J[u^N],
\end{align}
which is a screened Poisson equation with piecewise a constant screening
coefficient. We test the efficacy of exact and inexact Newton's method
applied to~\eqref{eq:F} in $(0,1)$ by discretizing $dJ$ with a
second-order finite difference method. The choice of parameters $a$,
$b$, $u_0$, and $u_1$ result in the double well potential illustrated in
Figure~\ref{fig:CA}(a), and both the exact and inexact Newton's methods
converge to the solution in Figure~\ref{fig:CA}(b).
Figure~\ref{fig:CA}(c) shows the max norm of the residual at each Newton
iterate when using the exact second derivative (red curve) and the
piecewise constant approximate (blue curve). While the inexact Newton's
method requires roughly twice as many iterations to reach of a tolerance
of $10^{-4}$, each iteration is significantly cheaper in higher
dimensions.

Having demonstrated that the inexact Newton method converges in one
dimension, it will be coupled with IE formulations to 
solve~\eqref{eq:SL} in two and three dimensions. We first partition
$\Omega$ into $\Omega_0 = \{\xx \in \Omega \:|\: u(\xx) < (u_0 +
u_1)/2\}$ and $\Omega_1 = \{\xx \in \Omega \:|\: u(\xx) > (u_0 +
u_1)/2\}$, and we let $\Lambda = \{\xx \in \Omega \:|\: u(\xx) = (u_0 +
u_1)/2 \}$ be their common boundary, and we let $G_0$ be the fundamental
solution of $(-\rho^2 \Delta + a)$ and $G_1$ be the fundamental
solution of $(-\rho^2 \Delta + 1)$. Then, the solution
of~\eqref{eq:screenedPoisson} can be written as $u = u^P + u^H$ where
$u^P$ is a volume potential and $u^H$ is a layer potential. The volume
potential is
\begin{align*}
  u^{P}(\xx) = \int_{\Omega_i} G_i(\xx,\yy) J[u^N](\yy)\, d\yy, 
    \quad i = 1,2,
\end{align*}
but it is discontinuous across $\Lambda$ and does not satisfy the
prescribed hydrophobic boundary conditions on the particles $\Gamma_i$.
Assuming the particles are entirely contained in the hydrophobic region,
the boundary conditions will be enforced with the layer potentials
\begin{align*}
  u^{H}(\xx) = \begin{cases}
    \displaystyle\int_{\Lambda} G_0(\xx,\yy) \eta_0(\yy) \, ds_\yy + 
    \displaystyle\int_{\Gamma} \pderiv{}{\nn_\yy} G_0(\xx,\yy)
      \sigma(\yy) \,ds_\yy, \quad & \xx \in \Omega_0, \\
    \displaystyle\int_{\Lambda} G_1(\xx,\yy) \eta_1(\yy) \, ds_\yy,
     & \xx \in \Omega_1.
  \end{cases}
\end{align*}
The density functions $\sigma$, $\eta_0$, and $\eta_1$ are determined so
that $u = u^P + u^H$ satisfies the hydrophobic boundary condition on
$\Gamma = \cup_i \Gamma_i$, and that it is continuous and differentiable
across $\Lambda$. This approach of using IEs to solve elliptic PDEs with
piecewise constant coefficients has successfully been used in other
application such as scattering~\cite{hyu-bar2014, che-cho-cai2018}.

% ----------------------------------------------------------------------
\subsubsection{Accelerating the iterative solvers}
\label{subsec:NumericalIssues}

When solving for the density functions $\eta_0$, $\eta_1$, and $\sigma$,
a carefully chosen IE formulation only requires a mesh-independent
number of GMRES iterations~\cite{cam-ips-kel-mey-xue1996}. Therefore,
the required CPU time is proportional to the cost of a matrix-vector
multiplication that can be performed in optimal or near-optimal time
with the fast multipole method (FMM)~\cite{fmm5} and its
extensions~\cite{fmm1, fmm2, fmm3, fmm4, fmm6, fmm7, fmm8}. PI BQ is
experienced with applying FMMs for the Stokes
equation~\cite{qua-bir2014, bys-sha-qua2020} and the screened Laplace
equation~\cite{kro-qua2011, qua2011}. Furthermore, to reduce the number
of GMRES iterations, the efficacy of preconditioning strategies such as
the inverse Fast Multipole Method (IFMM)~\cite{cou-pou-dar2017} and
domain decomposition~\cite{che-bir2021} will be investigated. PI BQ used
the IFMM to precondition Stokes equations~\cite{qua-cou-dar2018}. A
suite of other preconditioners including sparse approximate
inverses~\cite{che2000} and multigrid~\cite{hem-sch1981, sch1982} will
be investigated.


% ----------------------------------------------------------------------
\subsubsection{Eliminating contact with adaptive time stepping}
\label{subsec:timeStepping}

A numerical issue when simulating the self-assembly of amphiphilic
particles is avoiding particle collision. The hydrophobic attraction
potential drives the amphiphilic particles towards one another and this
leads to physical contact in finite time. Such particle collisions in a
dense, rigid-body suspension are a great challenge and can be a
bottleneck in large-scale simulations. We propose to remedy this
computational challenge by developing novel high-order adaptive time
stepping methods. PI BQ developed a high-order adaptive time
stepping method for hydrodynamic suspensions \cite{qua-bir2016}, and it
has served as a robust method to simulate processes including mixing and
adhesion in suspensions~\cite{qua-vee-you2019, kab-qua-bir2017}.
Following this previous work, the proposed work will use a spectral
deferred correction method~\cite{dut-gre-rok2000} since it iteratively
applies a low-order, single-step method to achieve high-order accuracy.
At each time step, we will compute the total force and torque of the
system which is computationally cheap and physically zero, and therefore
provides an estimate of the local truncation error. PI BQ's experience
is that error estimates based on physical constraints, such as force-
and torque-free conditions, appropriately determine adaptive time step
sizes so that the dynamics are resolved without the computational
expense of embedded Runge-Kutta methods, step-doubling, Richardson
extrapolation.


