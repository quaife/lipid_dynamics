\noindent
{\bf Collaborative research: Self-organization modeled by moving domain
elliptic PDEs} \\
{Bryan Quaife (lead PI, Florida State University), \em Rolf Ryham (PI,
Fordham University), and Yuan-Nan Young (PI, New Jersey Institute of
Technology)}

\section{Background}
\label{sec:background}

Self-organization encompasses a broad range of
interesting and challenging phenomena in engineering and science.
For example, understanding the
organization of lipids and proteins in vesicles is crucial to design
effective drug carriers
and functional mimics of cell membranes
\cite{Marui2022IncreasedEO,https://doi.org/10.1002/adma.202206288}.
The COVID-19 pandemic has led to a
concerted, worldwide efforts to understand
the mechanisms of RNA condensation by structural proteins protein
oligomerization that control virus structure
\cite{Kim2021SelfassembledMV}.
Engineers have developed superior techniques
that rely on nanoscale  
capillary and van der Waals forces
to print high-resolution
features on devices \cite{Zeng20223DprintedMT}.
Finally, researchers design efficient algorithms for
constructing for the interaction kernels to study complex
patterns in agent-based systems like swarms and flocking
\cite{Lu2019NonparametricIO,Tadmor2021OnTM}.

%A number of mathematical techniques have been developed to handle
%self-organizing systems. Each technique has its advantages and
%disadvantages. Prominent in chemistry and biology, molecular dynamics
%(MD) simulations offers unparalleled resolution of protein-lipid
%interactions, say, but because of the large number of non-linear
%interactions involved, is limited to short time and small spatial
%scales. In terms of inverse problems, agent based systems also involve
%relatively large numbers of particles and researchers extensively
%consider the problem of using trajectory data to learn interaction
%kernels of the dynamics when constrained to evolve on Riemannian
%manifolds. Once the interaction kernels are known, one can in principle
%derive partial differential equations (PDE) for the macroscopic
%behavior. 

When fluids are involved,
tracking self-organization
becomes difficult due to the deformability of
suspended media in the presence of non-local hydrodynamic interactions.  
A dominant approach in this area,
the phase field method
introduce a scalar-valued phase
field function for the volume fraction of
amphiphilic lipids or polymers in solvent, for example.
Sometimes called diffuse interface
methods, one devises energy functionals involving the gradient and
well-potentials of the phase field, and possibly long range interaction
kernels, 
\cite{Promislow2022UndulatedBI,C9SM01983A,doi:10.1063/5.0009734,
LiAn-Chang16,Choksi2003OnTD}.
Because the fluid interfaces are given by the
level-set of a smooth function,
they can have almost arbitrary topology
e.g., free ends, junctions
\cite{Promislow2017ExistenceBA,Promislow2022UndulatedBI},
the evolution equations can be studied
by well-established tools variational and analytical means,
\cite{Gavish2011CurvatureDF,Dai2019WeakSF,Dai2015CompetitiveGE,
Dai2022GeometricEO,Dai2020MinimizersFT,Dai2013GeometricEO},
and the governing equations are amenable to numerical approximation
\cite{Christlieb2020BenchmarkCO,Christlieb2019CompetitionAC}.
One of the downsides, though, 
is that granular details like molecular orientations and
shape changes induced by proteins that may be crucial to the physics, are
clumsy to incorporate in the model. Moreover, numerical simulations of
phase field models typically discretize the entire fluid domain under
consideration, and apply cutoff boundary conditions when periodic
domains are not under consideration. 

The PIs recently developed a new mathematical model of
self-organization in fluid systems~\cite{FuQuRyYo22,fu-ryh-qua-you2022,Fu2018_SIAM}.
The framework uses small, rigid bodies called ``granules''
to represent parcels of suspended media.
The granule shapes and boundary conditions
capture the local properties of the suspension,
and instead of volume fraction, a phase field function represents
the properties of the solvent.
Our approach falls into the category of work that considers
particles as a discretization of a continuous problem;
in this case one is dealing with a particle
method, as an alternative to other discretization methods
such as finite-difference, finite-element, and spectral methods
\cite{Wilson2021ComparisonOT}.
We have departed from the terminology
``particles'', though, to emphasize the role played the granules' finite size.


The framework is robust, flexible, and
cuts across many of the aforementioned challenges.
(i) The granules are not fixed to a stencil, 
affording flexibility in terms of morphology and topology---sufficiently
large forces will even pull granules apart as physically required.
(ii) The model formulation is built around important details like molecular orientations,
and exhibits inter-molecular slip that must be added into diffusive interface models
post hoc. 
(iii) It is straightforward to consider mixtures of granules with different
types of shapes and boundary conditions.
Leveraging boundary integral and fast summation methods, the dynamics can be solved in 
near-linear complexity in the number of granules,
which is equal to or better than 
the computational complexity of related sharp interface methods. 

\begin{figure}
  \begin{center}
%    \includegraphics[width=0.45\textwidth]{figures/SpecificAim1/Domain.jpg}
%    \includegraphics[width=0.45\textwidth]{figures/SpecificAim1/3Particles.jpg}
    \includegraphics[keepaspectratio,height=2.7cm]{figures/SpecificAim1/Domain.jpg}
    \includegraphics[keepaspectratio,height=2.7cm]{figures/SpecificAim1/3Particles.jpg}
  \end{center}
  \caption{\label{fig:flow_map} \footnotesize {\em Left:} In the HAP
  formulation~\cite{Fu2018_SIAM}, a system of exterior problems defines
  the hydrodynamic and hydrophobic interactions. {\em Right:} Particles
  translate and rotate to lower the free energy $E[u]$ subject
  to~\eqref{eqn:phase}, $W(\phi) = \phi^2$, and reach a minimal
  configuration in panel C. The colormap is blue for $\eta = 0$ and red
  for $eta = 1$. The arrows are the velocity field $\uu$.}
\end{figure}

\subsection{Problem formulation}
To formulate the problem, consider a collection of $N_b$-many rigid
bodies $U_i(t) \subset \RR^n$ with boundary $\Gamma_i(t)$. We call these
bodies ``granules''. The granules are immersed in an incompressible,
zero-Reynolds number fluid $\Omega(t) = \RR^n \setminus \cup_{i=1}^{N_b}
U_i(t)$ (Figure~\ref{fig:flow_map}). Throughout, $\nnu$ refers to unit
normal pointing into $\Omega(t)$.

To determine how the granules move through space,
we look for rigid body transformations
\begin{align}
\label{eq:RBT}
  F_i(\XX,t) = R_i(t)(\XX - \aa_i(0)) + \aa_i(t),\quad \XX \in U_i(0),
\end{align}
where $i = 1,\ldots,N_b,$ $R_i(t)$ is an $n
\times n$ rotation matrix, $\aa_i(t)$ are the granule centers, and $\XX$ is a
reference coordinate.
The dynamics come from the following system of semi-linear, elliptic boundary value problems:
\begin{alignat}{4}
  \label{eqn:stokes} 
  -\mu\Delta \uu + \nabla p &= \mathbf{0}, 
  \quad \nabla \cdot \uu = 0, &&\xx \in \Omega(t), \\
  \label{eqn:phase}
  -\epsilon^2 \Delta \phi + W'(\phi) &= 0, &&\xx \in \Omega(t),\\
  \label{eqn:noslip}        
  \frac{dF_i}{dt}(\XX,t) & = \uu(\xx) = 
    \vv_i + \omega_i \times (\xx - \aa_i), 
  \quad &&\xx \in \Gamma_i(t),\\
  \label{eqn:material}
  \phi(\xx) &= h_i(\XX),  &&\xx \in \Gamma_i(t),
\end{alignat}
for $i=1,\ldots,N_b$ and where $\xx = F_i(\XX,t)$.  Finally,
\begin{align}
\label{eqn:stressbalance}
\int_{\Gamma_i} \left(\sigma  + T_i\right)\nnu \,\dif s = \mathbf{0},\quad
\int_{\Gamma_i} (\xx - \aa_i)\times \left(\sigma + T_i\right) \nnu \,\dif s = \mathbf{0}
\end{align}
where
\begin{align}
\label{eqn:hydro_stress}
\sigma = \mu(\nabla \uu + \nabla \uu^T) - pI,\quad 
T_i = \gamma\left[\epsilon^{-1} W'(\phi)I
  + \epsilon\left(\tfrac{1}{2}|\nabla \phi|^2I - \nabla \phi \nabla
  \phi^T\right)\right]
\end{align}
are the hydrodynamic and phase field stresses, respectively.
Equations~\eqref{eqn:stokes} are the Stokes equations 
describing the velocity $\uu$ and pressure $p$ of
the incompressible fluid region with viscosity $\mu$.
Equation~\eqref{eqn:phase} describes the
transitions of the scalar order parameter $\phi$ with a decay length
$\epsilon$.
The scalar function $W(\phi) : \mathbb{R} \to \mathbb{R}$ is a nonnegative
well-potential with local minima $\phi_0, \phi_1$  etc.
representing metastable states of the solvent.
For the single well case, $W(\phi) = \frac{1}{2}\phi^2$
has the minimum $\phi_0 = 0$ and the phase field equation
\eqref{eqn:phase} becomes the \emph{linear} screened Laplace equation.
For the double well case e.g, $W(\phi) = \frac{1}{4}(\phi^2-1)^2+cx$, $c > 0$,
the phase field equation \eqref{eqn:phase} takes the form of an
Allen-Cahn equation with local minima $\phi_0 < 0 < \phi_1$.
Section \ref{sec:specificaim1} discusses certain
modifications needed for \eqref{eqn:phase} in the nonlinear case.

\begin{wrapfigure}[21]{r}{3.2in}
  \vspace{-10pt}
  \begin{center}
    \begin{tabular}{m{0.9in}m{0.9in}m{0.9in}}                 
    \includegraphics[width=0.85in]{figures/SpecificAim1/N100B1.jpg}
    &\includegraphics[width=0.85in]{figures/SpecificAim1/N100B2.jpg}
     &\includegraphics[width=0.85in]{figures/SpecificAim1/N100B3.jpg}    \\
    \includegraphics[width=0.85in]{figures/SpecificAim1/N100C1.jpg}
    &\includegraphics[width=0.85in]{figures/SpecificAim1/N100C2.jpg}
      &\includegraphics[width=0.85in]{figures/SpecificAim1/N100C3.jpg}    \\
      \includegraphics[width=0.85in]{figures/SpecificAim1/N100A1.jpg}
      &\includegraphics[width=0.85in]{figures/SpecificAim1/N100A2.jpg}
      &\includegraphics[width=0.85in]{figures/SpecificAim1/N100A3.jpg} 
  \end{tabular}
  \end{center}
  \vspace{-20pt}
  \caption{\footnotesize \label{fig:self-assembly2} Changes in the
  boundary condition \eqref{eqn:material} yield distinct morphologies.
  Each row has the same initial configuration, $W(\phi) =
  \tfrac{1}{2}\phi^2$. (\emph{Top}) Granules with an amphiphilic
  boundary condition results in a bilayer pattern (yellow, $\phi = 1$,
  blue, $\phi = 0$). (\emph{Middle}) Granules with $\phi = 2$ (magenta)
  on one side and $\phi = 1$ on the other form multilamellar bilayers.
  (\emph{Bottom}) Bipolar granules with $\phi = 1$ (yellow) on one side
  and $\phi=-1$ (green) on the other side form a checkerboard pattern.}
\end{wrapfigure}

For the moving domain boundary conditions, equations~\eqref{eqn:noslip}
specify a no-slip boundary condition for a rigid body with translation
velocity $\aa'_i(t) = \vv_i$ and angular velocity $\omega_i$ (the axial
vector for $R'_i(t)R_i(t)^T$) about the point $\aa_i(t) \in U_i(t)$. In
equation~\eqref{eqn:material}, the function $h_i$ is a material label
specifying the phase at water-granule interface. Finally,
equation~\eqref{eqn:stressbalance} closes the system by requiring that
the force and torque generated by hydrodynamic stress balances that
generated by phase stress $T_i$ in~\eqref{eqn:hydro_stress}. The
parameter $\gamma > 0$ is surface tension, $I$ is the identity matrix,
and we define the ``cross product'' $\aa \times \bb = (\aa^{\perp} \cdot
\bb) \kk$ where $\kk = (0,0,1) \in \RR^3$ and $(a_1,a_2)^{\perp} =
(-a_2,a_1)$ whenever $\aa, \bb \in \mathbb{R}^2$.

\subsection{Applications}
Since the framework is scale independent, it can be applied broad range
of nonspecific self-organization phenomena in surface chemistry, biology
and engineering. For one, the
formulation~\eqref{eq:RBT}--\eqref{eqn:stressbalance} captures exactly
the right behavior expected from amphiphilic self-assembly. For example,
three bodies with amphiphilic label (\S\ref{sec:specificaim1})
spontaneously self-organize to sequester the hydrophobic fluid phase
(Figure~\ref{fig:flow_map}). Changes in the boundary conditions result
in distint morphologies (Figure~\ref{fig:self-assembly2}) whose rheology
can be further investigate (\S\ref{sec:specificaim3}).
%
A concrete application the PIs have considered in our
SIAM Multiscale Modeling and Simulation and Journal of Fluid Mechanics works
\cite{Fu2018_SIAM, FuQuRyYo22}
is the problem of \emph{vesicle hydrodynamics}.
A vesicle is a small, fluid filled membrane sack.
The goal in vesicle hydrodynamics is to study the flow patterns 
that arrise due to the interaction
between membrane elasticity and the fluid.

A striking aspect of our hydrodynamic
simulation is its fidelity to physics.
The model contains a few
parameters; granule radius $c$,
screening length $\epsilon$,
surface tension $\gamma$.
When set to realistic values in the literature
\cite{Fu2018_SIAM, ErLjCl89, Lin2005, Parsegian, Israelachvili80, GarciaSaez, KUZMIN2005, Petelska2012,Jackson2016},
we quantitatively recover many of the well-known
elastic moduli for single-component lipid bilayers
\cite{Nagle17, Nagle17-2, LeVeWa14,NAGLE2000159}. 
Such agreement is a validation that equations~\eqref{eq:RBT}--\eqref{eqn:stressbalance} 
correctly capture the long-range interactions between amphiphilic granules (such as lipid macromolecules) surfaces.

\begin{wrapfigure}[9]{l}{0.46\textwidth}
\vspace{-10pt}
\includegraphics[width=0.46\textwidth]{figures/PreliminaryWork/TankTreading.jpg}
\vspace{-20pt}
\caption{\label{fig:JPv_linearshear} \footnotesize Granular vesicles
  undergo tank-treading in background shear flow.}
\end{wrapfigure}
%
In terms of hydrodynamics, the granule vesicles in background flows
replicate the behaviour of an continous, inextensible, elastic vesicle
with permeability. The simulations showed, for the first time, a
granule-vesicle suspension behaving as a tank-treading vesicle
\cite{Finken2008, Shaqfeh11} (see Figure \ref{fig:JPv_linearshear}).
Movies of Figure \ref{fig:JPv_linearshear} reveal intermonolayer slip
and derived values for intermonolayer friction were in good quantitative
agreement those derived by atomistic and Martini force field simulation
studies \cite{WuoEd06, denOtter2007, SHKULIPA2005823, Zgorski2019}. In
Figure \ref{fig:JPv_rupture}, membrane rupture occurs at large shear
rates and yields a dimensional scaling for critical rupture shear rate
\cite{VLAHOVSKA2009775,keller_skalak_1982}. A side-by-side comparison of
granule-based vesicles and the sharp interface representation yielded
shapes and trajectories that overlapped.

\begin{wrapfigure}[14]{r}{0.475\textwidth}
%\vspace{-10pt}
\includegraphics[width=0.475\textwidth]{figures/PreliminaryWork/Rupture.jpg}
\caption{\label{fig:JPv_rupture} \footnotesize Rupture of a granular
  vesicle under large shear rates.}
\end{wrapfigure}

Having a non-specific model that replicates physical details
is the crown jewel of applied mathematics.
\emph{Further applications} include
(i) amphiphilic Janus spheres that form diverse suprastructures
\cite{HaBr20,McBr21,Bradley2017},
(ii) pickering emulsions involving
two-phase fluids and dense suspensions \cite{Bradley2016},
(iii) creation and manipulation of small droplets
using amphiphilic microparticles \cite{Ha2022SurfaceEM,Ha2020MinimalSC},
and (iv) tension drivent fabrication of microstructures
~\cite{Dasgupta2017, Leong2007, Reynolds2019, Cho2010,Zeng20223DprintedMT,Russell2016EnergyLF}.
\begin{wrapfigure}[9]{l}{2.1in}
  \vspace{-5pt}
  \centering
  \includegraphics[width=1in]{figures/Background/Peanut/PeanutFEM.pdf}
  \includegraphics[width=1in]{figures/Background/Peanut/PeanutIE.pdf}
  \caption{\label{fig:fem_vs_bie} \footnotesize A finite element method
  ({\em left}) requires meshing the entire domain (typically into
  triangles), while a BIE ({\em right}) requires meshing only the
  boundary of the domain.}
\end{wrapfigure}

\subsection{Computational method and mathematical properties}
Computational methods to solve the PDEs~\eqref{eqn:stokes}
and~\eqref{eqn:phase} are challenging to develop because of
the complicated and unbounded domain.
This challenge is compounded when dense
suspensions are considered. Integral equations (IEs) offer an
alternative formulation of the PDEs that only require meshing the
boundary of the granules rather than all of $\Omega(t)$
(Figure~\ref{fig:fem_vs_bie}), and the automatically satisfy far field
conditions. Additional advantages of an discretizations of IE
formulations include high-order, or even spectral, accuracy and
well-conditioned linear system that can be solved iteratively with a
mesh-independent number of Krylov iterations such as the generalized
minimal residual method (GMRES). However, IEs have challenges: nearly
touching granules requires a quadrature rule for nearly-singular
integrals; their discretization results in dense matrices; and they
cannot immediately be applied to non-linear PDEs. The IE community,
including PI Quaife, has developed a variety of tools to address these
numerical challenges, and we discuss in \S~\ref{sec:specificaim2}.

In the absence of external flow, the
system~\eqref{eq:RBT}--\eqref{eqn:stressbalance} has the energy law
\begin{align}
\label{eq:energy_law}
  \begin{split}
  \frac{d}{dt}E
    &= - \int_{\Omega(t)}\tfrac{1}{2}\mu|\nabla \uu + \nabla
  \uu^T|^2 \,d\xx, \\
    E &= \gamma \int_{\Omega(t)}
  \frac{\epsilon}{2} |\nabla \phi|^2 + \frac{1}{\epsilon} W(\phi) \,d\xx.
  \end{split}
\end{align}
%\begin{align}
%\label{eq:energy_law}
%  \frac{d}{dt}E
%  = - \int_{\Omega(t)}\tfrac{1}{2}\mu|\nabla \uu + \nabla
%  \uu^T|^2 \,d\xx,\quad
%    E = \gamma \int_{\Omega(t)}
%  \frac{\epsilon}{2} |\nabla \phi|^2 + \frac{1}{\epsilon} W(\phi) \,d\xx.
%\end{align}
Here $E$ is the energy stored in the phase transitions
and the energy law states that
the granules move to decrease energy,
which is lost due to viscous dissipation.
The derivation of \eqref{eq:energy_law} is interesting
because $E$ depends both on $\phi$
and $\Omega(t)$. It uses the Rayleigh Transport Theorem to write the time
derivative of $E$ as an energy density flux through the boundary
involving the phase stress $T_i$, and a bulk term involving the Euler-Lagrange
derivative of $E$~\cite{Fu2018_SIAM}. The Euler-Lagrange derivative vanishes due
to~\eqref{eqn:phase} and the boundary flux transforms into viscous
dissipation due to~\eqref{eqn:hydro_stress} and the Stokes equations. Figure~\ref{fig:coarsening} summarizes the general evolution of $E$ from the self-assembly process to
the collective dynamics of the assembly under external forcing.
\begin{wrapfigure}[16]{r}{4in} 
  \includegraphics[width=4in]{figures/Background/coarsening.pdf}
  \vspace{-20pt}
  \caption{\label{fig:coarsening} \footnotesize Energy $E$ in
  equation~(\ref{eq:energy_law}) of a suspension of amphiphilic granules
  versus time $t$. Under the moving domain problem
  \eqref{eq:RBT}--\eqref{eqn:hydro_stress} with amphiphilic boundary
  condition, the granules self-organize into vesicle bilayers. External
  hydrodynamic effects are introduced to study elastic properties of
  the granules assembly~\cite{Fu2018_SIAM,FuQuRyYo22}.}
  \vspace{5pt}
\end{wrapfigure}

In the far field, $\lim_{\xx \to \infty} \phi(\xx) = \phi_0$ and for the
background flow, we require that $\lim_{\xx \to \infty} \uu(\xx) -
\uu_{\infty}(\xx) = \mathbf{0}$, where
the background flow takes the form
%In the case of shear background flow,
%for example,
%\begin{align}
%\label{eqn:shear_BG_flow}
%\uu_{\infty}(\xx) = \dot{\gamma} \xx \cdot \mathbf{e}_y \mathbf{e}_x,
%\end{align}
%where $\dot \gamma$ is the shear rate and $\mathbf{e}_x$, $\mathbf{e}_y$
%are horizontal and vertical basis vectors, respectively.
%Other
%background flows include
shear, extensional, parabolic Poiseuille, and
Taylor-Green flow, for example.
For numerical stability and physical realism, we
include pairwise repulsive forces and torques
in~\eqref{eqn:stressbalance} to prevent the collision between
neighboring granules which introduces a repulsive potential in the energy
law~\eqref{eq:energy_law}. Background flow adds elastic energy to the
granule configurations.
%and is quantified by simulation. We refer to the
%PIs work~\cite{FuQuRyYo22, fu-ryh-qua-you2022, Fu2018_SIAM} for details. 

In terms of physical justification, the derivation of the
one-dimensional version of our model was carried out using a Landau
expansion of the free energy density for a single-well
potential~\cite{MaRa76, ErLjCl89}. The double-well potential case was
considered by Gompper~\cite{GoHaKo94} to describe metastable
coordination of water. In the field of surface
chemistry~\cite{Israelachvili1954}, there is a large empirical
literature on properties of long-range hydrophobic
interaction~\cite{LeRaPa77, KoNa15, Nagle17, Lum1999, Lin2005,
Meyer2006, Ducker2016, Jackson2016, Gletal88, Aketal17, Ch05}. The
stress~\eqref{eqn:stressbalance} was first identified by PIs Ryham and
Young in~\cite{Fu2018_SIAM} and generalizes one-dimensional hydrophobic
interaction to the $n$-dimensional setting. 

Regarding mathematical motivation for~\eqref{eq:RBT}--\eqref{eqn:stressbalance},
when equation \eqref{eqn:phase} takes the form of a screened Laplace equation
$-\epsilon^2 \Delta \phi + \phi =0$, for a fixed domain $\Omega$ with
smooth boundary, if the material label $h_i = 1$
everywhere, then $\phi$ has a boundary layer going from $1$ at the
boundary to $0$ in the bulk and one can prove that 
$\lim_{\epsilon \to 0} E = \frac{1}{2}\gamma |\partial \Omega|$.
In other words, 
interfacial energy enters the problem even in the linear case,
\emph{because of the introduction of granule boundaries.}
We are confident
that~\eqref{eq:RBT}--\eqref{eqn:stressbalance} is a correct description
of long-range interactions between granule surfaces.


\subsection{Outline of the proposal}
The aim of this proposal is to advance the simulation
and theory for soft-matter systems
and foster training in mathematical sciences.
Section \ref{sec:BroaderImpacts} summarizes the broader
impacts and section \ref{sec:IntellectualMerit} summarizes the intellectual merit of our research activity.
Section \ref{sec:proposed-work} contains our
three specific aims.  They are
mathematical analysis of the system
(\S \ref{sec:specificaim1}: PI Ryham),
numerical implementation
(\S \ref{sec:specificaim2}: PI Quaife),
and  kinetic theory
(\S \ref{sec:specificaim3}: PI Young).



