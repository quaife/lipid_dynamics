\documentclass{article}
\usepackage[utf8]{inputenc}
\usepackage{fullpage}
\pagestyle{empty}

\begin{document}
\begin{center}
{\bf \Large Summary}\\ Collaborative Research
\begin{tabular}{ccc}\\ 
Fordham University & Florida State University& New Jersey Institute of Technology\\
Rolf Ryham &            Bryan Quaife &              Yuan-Nan Young
\end{tabular}

\end{center}

\section*{Overview}
Numerical simulation has become an essential tool in nearly all areas of science and engineering, especially biophysics. This proposal specifically aims to advance mathematical modeling and analysis in membrane biophysics through the study of
collective dynamics of amphiphilic self-assembly. Self-assembly is a ubiquitous process in biology and is a major source of 
nonspecific interactions in soft matter. In the case of membranes, lipids self-assemble into bilayer and micelle morphologies. Over the past decades, macroscopic continuum mechanics has done a good job of modeling bilayer membranes at large length scales but can miss details at small length scales brought about by inclusions or topological changes that are crucial to the dynamics. Molecular dynamics models account for these granular details but are computationally costly when accurately accounting for nonlocal interactions. To design better membrane models, it is imperative to quantify the collective physical properties and processes of amphiphiles, and the proposed mathematics establishes a platform for efficiently simulating the collective dynamics at large scales in a manner that allows for direct comparison with preexisting theory and experiment.

\section*{Intellectual Merit}
The purpose of this research is to reach interesting physical phenomena with 
less computational cost than molecular dynamics, and account for more general
features that continuum theory misses. The main ingredient is defining a 
nonlocal interaction through the solution of an elliptic boundary value problem
that has the phenomenological characteristics of long-range hydrophobic
attraction. It turns out that this minimal model gives rise to rich phenomena
for Janus particle aggregates and correctly predicts elastic properties of bilayer. 
The technical research tasks include quantifying collective properties of 
amphiphilic ensembles, mathematical analysis of continuum elastic energies, 
efficient, high-order numerical algorithms for large-scale simulations, and 
incorporating external fields through electric charge. Lastly, the proposal 
extends the results using three-dimensional boundary integral formulations.

\section*{Broader Impacts}
This project aims to advance the mathematical modeling of collective 
dynamics of amphiphilic particles. The simulations use a new, yet intuitive,
approach that can account for important and complex systems that are out of 
reach in computational material science. These complex systems include 
fusion and fission of amphiphilic bilayer membranes and optimal shape design
in metamaterials. The development of three-dimensional 
models describing colloidal systems could be transformative in biomedicine
and material science. The research draws from expertise in scientific 
computing, physics of fluids, and mathematics. The mathematical component 
incorporates leading techniques from geometric analysis and gives deep insight 
into fundamental material science. The project offers undergraduates 
in a socially impactful manner the opportunity to do research and train 
with graduate and postdoctoral personnel. It incorporates research in the 
classroom, and with its combination of mathematical modeling, analysis, 
and scientific computing, the project highlights the importance of 
mathematics and computation to all areas of science and engineering.




\end{document}
