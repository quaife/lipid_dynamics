\section{Relevant Results from Prior NSF Support}
\noindent {\bf Rolf Ryham}: no prior NSF support.

\noindent
{\bf Yuan-Nan Young}: {\em NSF-DMS-1222550, Mathematical and
experimental study of lipid bilayer shape and dynamics mediated by
surfactants and proteins}, \$212,603, 9/15/2012 - 08/31/2016 (with
no-cost extension), PI. {\em Intellectual merit:} The focus of this
grant is modeling the interaction between a pure lipid bilayer membrane
with surfactant, cholesterol and protein.
%Hence, there is no overlap between these grants and the current proposal.

%YNY and collaborators have studied (1) asymmetry of lipid bilayer due to its interaction
%with proteins and surfactants, (2) the electrohydrodynamics of a LBM under an
%electric field, and (3) the coupling between LBM dynamics and a transmembrane protein,
%such as a mechanosensitive channel. 

\noindent
{\it Broader impacts:} 
One PhD student (Szu-Pei Fu) was funded to work with YNY, and work has
resulted in seven papers
\cite{Nganguia2013_PoF,Nganguia2013_PRE,Young2014_JFM,Young2015_PoF,Nganguia2015_CiCP,Pak2015_PNAS,fu2015pre}.
YNY has been actively involved with promotion of underrepresented
students at NJIT. The other PhD student (Herve Nganguia) is African.
YNY has taught a broad spectrum of courses in fluid mechanics and
applied math modeling.

\noindent
{\bf Bryan Quaife}: {\em NSF DMS-2012560, Erosion, Transport, and
Dispersion in Granular and Porous Media}, \$249,636,
08/01/2020--07/31/2023, PI. {\em Intellectual Merit:} The goal of this
research is to develop high-order numerical methods to
simulate hydrological processes including erosion.

\noindent
{\it Broader impacts:} 
A second-year PhD student (Jake Cherry) has been assigned to this
project. Since the funding just began, results have not yet
been published or presented. 

\section{Project Management, Collaboration Plan, and Schedules of
Research Tasks}
\setlength{\parindent}{0pt}
% N.B. wrapfigure is not compatible with \noindent.

\begin{wrapfigure}[8]{r}{0.56\textwidth}
\vspace{-19pt}
\definecolor{barblue}{RGB}{51,102,254}
%\renewcommand\sfdefault{phv}
%\renewcommand\mddefault{mc}
%\renewcommand\bfdefault{bc}
%\sffamily
\begin{ganttchart}[
    canvas/.append style={fill=none, draw=black!5, line width=.75pt},
        x unit =4.5mm,
        y unit chart =\baselineskip,
    hgrid style/.style={draw=black!5, line width=.75pt},
    vgrid={*1{draw=black!5, line width=.75pt}},
    title/.style={draw=none, fill=none},
    title label font=\bfseries\footnotesize, % numbers across the top
    title label node/.append style={below=3pt},
    include title in canvas=false,
    bar label font=\mdseries\footnotesize\color{black!70}, 
    % titles across the left
    %bar label node/.append style={left=2cm},
    bar/.append style={draw=none, fill=barblue},
    bar incomplete/.append style={fill=barblue},
    bar progress label font=\mdseries\footnotesize\color{black!70},
    milestone label font=\mdseries\small\color{black!70},
        milestone left shift =0.9,
        milestone right shift =0.1,
        % Don't draw group bars
        group height =0,
        group peaks height =0,
        group label font =\bfseries\small,
]{1}{12}
\gantttitle[
    title label node/.append style={below left=3pt and -6pt}
]{QUARTERS:\quad1}{1}
\gantttitlelist{2,...,12}{1} \\
%\ganttgroup{Tasks\hfill}{1}{12}\\
  \ganttbar{Specific Aim 1 (2D)}{1}{4} \\
  \ganttbar{Specific Aim 1 (3D)}{8}{12}\\
\ganttbar{Specific Aim 2}{1}{8}\\
\ganttbar{Specific Aim 3}{1}{12}\\
\ganttbar{Program management}{1}{12}
\end{ganttchart}
  \vspace{-10pt}
\caption{\footnotesize Schedule for the proposed work, measured in
  quarters from the beginning of the project.}
\label{fig:schedule}
\end{wrapfigure}
\textbf{Project management}: 
%
The success of the proposed research requires complementary expertise
and collaborative efforts in physics, applied mathematics, algorithms,
and computing. Ryham has been working on mathematical modeling with a
strong analytical background. Young has been working on many areas of
computational fluid dynamics and applications to math biology for many
years. Quaife has been working on integral equation methods, fast
algorithms, and their applications to fluid dynamics for many years.
Their recent collaborative work on the HAP model in two dimensions has
provided a solid foundation for the proposed research.

\medskip

\textbf{Collaboration plan}: 
%
The management responsibility of this collaborative research will reside
with the lead PI (Ryham) for this endeavor. The research work is
structured to meet the tasks discussed in \S\ref{sec:proposed-work}.
%
The PIs, postdocs, and students will meet frequently on Zoom and
in person when possible. During the pandemic, the three PIs will meet
weekly on Zoom for a research meeting. Once the pandemic is under control, PI RR and PI YNY will conduct biweekly meetings in person, with PI BQ Zoom in from Florida.
The PIs will share software packages, paper
sources, and references on a common \textsf{Git} repository. The
resulting software packages will be posted on the \textsf{Github}
software repository.

\medskip

\textbf{Research Schedule:} The detailed schedule for the proposed work
is shown in Figure~\ref{fig:schedule}. Specific Aim 1 is split over two
periods to allow time for 3D code to be developed.


