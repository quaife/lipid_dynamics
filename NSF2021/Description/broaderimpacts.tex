\section{Broader Impacts}
\label{sec:BroaderImpacts}
This project aims to advance the mathematical modeling of collective
dynamics of amphiphilic particles. The simulations use a new, yet
intuitive, approach that accounts for important and complex systems
that are presently out of reach in computational material science. These complex
systems include optimal shape design in metamaterials and
fusion and fission of amphiphilic bilayer membranes. The further
development to three-dimensional models describing colloidal systems could be
transformative in biomedicine and material science. The research draws
from expertise in scientific computing, physics of fluids, and
mathematics. The mathematical component incorporates leading
variational techniques and offers insight into fundamentals of material
science. The project brings socially consequential research into the
classroom, and offers undergraduates the opportunity to train alongside
faculty and postdoctoral fellows. With its combination of
mathematical modeling, analysis, and scientific computing, the project
highlights the importance of mathematics and computation to all areas of
science and engineering.
%The proposed mathematical analysis, modeling, and numerical algorithms
%will transform our understanding of the collective dynamics of
%amphiphilic particles such as (1) their self-assembly into micelles and
%bilayers, (2) the material properties of such self-assembly, and (3) the
%interaction between these building blocks. In addition to biophysical
%applications, amphiphilic Janus particles have recent popularity in the
%fabrication of smart materials. The proposed work will have a
%transformative impact on precision design for specific mechanical
%properties of materials made of amphiphilic nanoparticles. An important
%component of this proposal is the interdisciplinary education and
%training of both undergraduate students, graduate students, and
%postdoctoral fellows. The combination of mathematical modeling,
%analysis, and scientific computing in this project provides a compelling
%example of the importance of mathematics and computation in biophysics
%and engineering applications. The concepts and methods described here go
%beyond the context of amphiphilic particles and lipid molecules. They
%extend to other problems featuring microscopic phase separation that
%leads to formation of mesoscopic domains. This situation arises, for
%example, in biological development in systems biology. The methods
%developed here have the potential to impact those and other related
%areas in biomedicine and biotechnology.

\subsection{Educational Impacts}
\label{subsec:Educational_plans}
%The topics of this proposal have application in robotics, machine
%learning, and engineering.
To foster training in mathematical sciences,
the proposal supports one undergraduate researcher (UR) per year
from the lead institution.
PI RR has worked side-by-side and
published with undergraduate coauthors in
\cite{RYHAM20112929, RyWaCo13, RyKlYaCo16}.
He is presently mentoring a Clare Boothe Luce Scholar 
on mathematical analysis of trends in election redistricting.
With another Fordham supported UR,
he is finishing a paper on a closed-form 
energy density for translationally invariant bilayers. 
Based on these experiences, we have outlined three projects for the present proposal.
\begin{enumerate}[noitemsep]
%\noitemsep
\item Translate project code from MATLAB into a compiled language   

\item Use machine learning to post-process data from the Janus particle self-assembly
%  This project will train UR in the MATLAB and Python programming language,
%  how to effectively deal with large data sets, and how to represent data meaningfully
%  using plots and graphics. 

\item Mathematically investigate self-assembly in the zero-screen-length limit
%  The UR will address the following conjecture about self-assembly:
%  in the $\rho = 0$ limit, two convex polygons interacting through the hydrophobic attraction potential
%  converge to a limit with maximally overlapping boundaries.
%  This is a variational problem with ample numerical evidence. The student
%  will learn elements of boundary layer analysis. 

%\item Implementing a fast-multipole method
  %  The UR will adapt code written by undergraduates in \cite{RyKlYaCo16}.
%  The will analyze the energies of bilayer configurations under the new, closed form
%  energy and compare them to energies calculated in molecular dynamics studies.
  
\end{enumerate}
The UR support is for eight summer weeks. 
In addition to collaborating with the PI,
they will have tutorials and receive training in
mathematical writing and presentation.
The UR will have a desk and meeting room
in a recently completed collaboration space
built for the Mathematics Department. 
As a condition of support, 
they will be required to write a summary report and
give a presentation in the university's undergraduate
research symposium.  The PI 
will encourage the
UR to participate in a national conference.

To meaningfully engage the community,
the PIs will prioritize students coming
from underrepresented groups and
specifically target students whose socio-economic
background prevents them from participating in out-of-state research
experiences.  PI RR will solicit applicants from Fordham University's
Collegiate Science and Technology Entry Program.
Students from the Bronx High School of Science (PI
RR) and Newark Science Park High School (PI YNY) will also be encouraged to
join the research team. PI BQ will work with undergraduates and high
school students through the Undergraduate Research Opportunity Program
and Young Scholar's Program, as he has done in the past.

To maximize vertical integration, the PIs and personnel
will travel to New York for one or two weeks during the summers.
Finally, we will create several modules to include in our courses. These
modules will introduce topics from numerical linear algebra and
optimization, and highlight concrete applications of mathematics
based on the ideas of the proposal.
%The proposed research will have an immediate impact on undergraduate and
%graduate education. At the undergraduate level, the project will include
%and support undergraduate researchers from Fordham University. The
%supported summer researchers will gain valuable first-hand
%interdisciplinary experience in mathematical modeling and computation.
%We will incorporate research into teaching, especially in differential
%equation and programming courses. PI YNY will work on simple modeling of
%self-assembly problems with both undergraduate and high school students
%from Newark Science Park High School. PI YNY has a track record of
%working with local high school students for their research projects
%before they apply for colleges. PI BQ will work with undergraduate and
%high school students through the Undergraduate Research Opportunity
%Program and Young Scholar's Program as he has done in the past.

%Over the past eight years, PI RR has included many undergraduates in the
%execution of research initiatives and coauthoring publications. He has a
%track record for supporting underrepresented groups, being called on by
%the Collegiate Science and Technology Entry Program (CSTEP) to make
%opportunities for promising students, for example. He has mentored two
%Clare Boothe Luce Scholars, one a U.S.~Marine Corp veteran, and a high
%school student in the NYU GSTEM program. 

%The PIs will capitalize on the synergy between both FSU's doctoral
%program in Scientific Computing, NJIT's doctoral program in Mathematical
%Sciences, and Fordham's undergraduate focus in the Department of
%Mathematics. Promising, young undergraduate scientists from the Bronx
%community will find a natural pipeline into graduate studies by working
%directly or indirectly on this project. At the graduate-level, PI YNY
%expects to train and support a PhD student for two more years. PI BQ
%will advise a postdoctoral fellow for two years. The PIs will foster
%vertical integration between the senior personnel, their collaborators,
%postdocs, doctoral, and Bachelor students, enhancing the learning
%environment.

%The PIs expect
%to involve undergraduate students in this project through
%their continuing and active involvement in undergraduate advising and research mentoring:
%YNY has mentored undergraduate students as a co-Investigator in CSUMS: Research and Education
%in Computational Mathematics for undergraduates in the Mathematical 
%Sciences at NJIT (funded by NSF) and the lecturer for Capstone Applied
%Mathematics Lab at NJIT (also funded by NSF).
%YNY has track records in involving undergraduate students in research
%that emphasizes both numerical computations and desktop experiments.
%
%such as MATH 340: Advanced Numerical Methods.
%Simple MATLAB codes have been used as teaching tools to show students how to use MATLAB
%to simulate the slender-body equations for an elastic filament in Stokes flow.
%
%In the past year YNY has been mentoring Ufuomaefe Ogbe, a high school student
%from Newark. Ufuomaefe is an African American who 
%worked with PI on a simplified model for vesicles in flow. Based
%on what he learned about modeling and programming, 
%he has applied to the Applied Mathematics/Computer Science programs at both NJIT
%and Rutgers/Newark.

\section{Intellectual Merit}
The purpose of this research is to reach interesting physical phenomena
with less computational cost than molecular dynamics, and account for
more general features that continuum theory misses. The main ingredient
is defining a nonlocal interaction through the solution of an elliptic
boundary value problem that has the phenomenological characteristics of
long-range hydrophobic attraction. It turns out that this minimal model
gives rise to rich phenomena for Janus particle aggregates and correctly
predicts elastic properties of bilayer. The technical research tasks
include quantifying collective properties of amphiphilic ensembles,
improving on mathematical models, 
efficient, high-order numerical algorithms for large-scale simulations,
and incorporating external fields through electric charge.
%Lastly, the
%proposal extends the results using three-dimensional boundary integral
%formulations.

%A central theme of the project is the mathematical development and
%analysis of the physical model for interaction between many amphiphilic
%particles. The model formulates the interaction potential through a
%screened Laplace equation boundary value problem possessing the physical
%properties like non-additivity and decay of realistic hydrophobic
%attraction. Colloidal systems collectively self-assemble into bilayer
%morphologies, and we analyze the elastic properties of these amphiphilic
%particle ensembles. This allows us to interpret the Helfrich free energy
%in terms of hydrophobic interactions and specific molecular
%characteristics. We extend the capabilities of the boundary integral
%equation and time stepping methods to stably perform large particle
%number simulations in three dimensions. Results from these computations
%will be used to compare collective amphiphilic against experiments and
%to study the optimal design of three-dimensional functional materials. 

