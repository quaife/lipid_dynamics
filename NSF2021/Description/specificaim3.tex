\subsection{Specific Aim 3: Self-assembly of Janus particles in ionic solutions}
\label{subsec:specific_aim_3}

\subsubsection{Dynamics of a pair of Janus particles in ionic
solutions\label{subsubsec:JP_electrolyte}}
\begin{wrapfigure}[7]{r}{0.45\textwidth}
  \vspace{-5pt}
\centerline{\includegraphics[width=0.44\textwidth]{Figures/fig2A_Chen2011_Science}}
  \vspace{-5pt}
\caption{\label{fig:helices_of_JPs}Formation of helices
  of Janus particles as salt concentration increases from 3.8 mM NaCl
  (left) to 5 mM NaCl (right)~\cite{Chen2011_Science}.}
\end{wrapfigure}
In experiments, Janus particles are often immersed in a solution of
ions~\cite{Chen2011_Science} or surfactants~\cite{Goodwin2009}. In
electrolytic solutions, the interaction between salt and Janus particle
gives rise to a slip velocity (on the particle surface) that depends on
the ion distribution and the zeta potential~\cite{BayatiNajafi2016_JCP}
(the potential difference between the Janus particle surface and the
surrounding conducting fluid). The short-range interactions between
Janus particles depend on the salt concentration: At very low salt
concentrations, Janus particles repel one another electrostatically,
whereas at high salt concentration, van der Waals forces cause Janus
particles to aggregate irreversibly~\cite{Goodwin2009}. At intermediate
concentrations of monovalent salt (NaCl), Janus particles are found to
assemble in the form of a small number of elemental units of building
blocks~\cite{Chen2011_Science}. Figure~\ref{fig:helices_of_JPs} shows
that Janus particles can assemble into a helix at high concentration of
monovalent salt when the volume fraction of Janus particles is low to
moderate. At high volume fraction, helices of Janus particles form
wormlike structures whose lifetime allow fusion of
helices~\cite{Chen2011_Science}. 

We propose to investigate the effects of ion transport and distribution
on the self-assembly of Janus particles within the HAP framework as
follows. First the Stokes equations are modified to incorporate the
transport of ions as
\begin{equation}
\label{eq:EKstokes}
\begin{aligned}
  &-\Delta \mathbf{u} + \nabla p = \Delta\psi\cdot\nabla\psi, \qquad
  \nabla \cdot \mathbf{u} = 0,  \quad \mathbf{x} \in \Omega,\\
  &\delta^2\Delta\psi = -\frac{1}{2}\left(n_+-n_-\right),\qquad
  \frac{\partial n_{\pm}}{\partial t} + \nabla\cdot{\bf j}_{\pm} = 0,\qquad {\bf j}_{\pm} = -\nabla n_{\pm} \mp n_{\pm} \nabla\psi + \mbox{Pe} n_{\pm}\mathbf{u},
\end{aligned}
\end{equation}
where $\psi$ is the electric potential, $\delta$ is the ratio of Debye
screening length (inversely proportional to the square root of ion
concentration) to the particle size, $\mbox{Pe}$ is the Peclet number, a
ratio between the convection and thermal diffusion time scales, and
$n_{+}$ ($n_{-}$) is the concentration of cation (anion). In the limit
of small $\delta$, PI Young and collaborator Y.~Mori (U Penn) used
asymptotic analyses to show that the jump in electric potential across
the interface (zeta potential on the Janus particle surface) may give
rise to different physical consequences such as
electrophoresis~\cite{Mori2018_JFM}.
%
An important question to address is: {\it How would such electrophoresis
be affected by the HAP between two Janus particles?} As the first part
of Specific Aim 3, we will address this important question. Results from
this effort will help elucidate the effects of salt concentration on the
assembly of Janus particles (\S\ref{subsubsec:em_effects}).

We will couple the electrokinetics in~\eqref{eq:EKstokes} with the
mobility problem of Janus particles in a viscous solvent through the
velocity field, ion distribution, and the hydrophobic potential. First
the screen length $\rho$ in the HAP model is now a function of ion
concentrations: $\rho = \rho(n_+,n_-)$. 
%
Second the value of the HAP may depend on the distribution of ions and
electric potential right next to the particle
surface~\cite{Mori2018_JFM}. Within this framework, we will first
investigate the interaction between two Janus particles under HAP in an
electrolytic solution described by the above equations. We will conduct
analytical calculations to investigate the effects of ion transport on
the hydrophobic attraction between two Janus particles using the
reciprocal theorem as in~\cite{BayatiNajafi2016_JCP} for two Janus
particles with a small surface activity.


\subsubsection{Electromechanical effects on the dynamic assembly of Janus particles \label{subsubsec:em_effects}}
Results from the proposed work in \S\ref{subsubsec:JP_electrolyte} will
shed light on how the electrokinetics described in~\eqref{eq:EKstokes}
may be simplified to a dielectric-like model where the bulk charges are
constant, and the fluid flow and particle mobility are determined by the
surface charge and the zeta potential as shown in~\cite{Mori2018_JFM}.
Under such simplification, equation~\eqref{eq:EKstokes} may be
re-formulated into integral forms. PIs Ryham and Young will conduct the
asymptotic analyses to derive this form, and work with PI Quaife to
design numerical algorithms to simulate the modified system. We will
first investigate the effects of salt concentration on the assembly of
Janus particles to form helices, as shown in
Figure~\ref{fig:helices_of_JPs}.

With these results, the PIs will be ready to make quantitative
connections between the a Janus particle membrane and a lipid bilayer
membrane in an electrolytic solution. In recent
experiments~\cite{FaizEtAl2019_SoftMatt}, membrane bending rigidity was
determined as a function of lipid composition from 0 to 100 mol $\%$ of
charged lipids using flicker spectroscopy of the shape fluctuations of a
giant unilamellar vesicle (GUV). Membrane bending rigidity increases
with increasing lipid surface charge, but decreases with increasing salt
concentration in the bulk solution due to the screening of the lipid
surface charge. This result agrees with several theoretical
models~\cite{Kralchevsky1996_JCIS, May1996_JChemPhys,
LoubetEtAl2013_PRE} that also assume the quadratic form of the elastic
energy density in the presence of surface charge and bulk
charge~\cite{DuplantierGoldstein1990_PRL, Winterhalter1992_JPC}. As the
electrostatic interaction is non-local in nature, we expect that the
controversy of the HK elastic energy form (see
\S\ref{subsec:specific_aim_1}) would worsen in the presence of
electrostatic interactions and electrokinetics. The PIs will extend the
approaches in \S\ref{subsec:specific_aim_1} to charged lipids to
calculate the bending moduli and compare against the experimental
results in~\cite{FaizEtAl2019_SoftMatt}. We propose to incorporate an
explicit surface charge on each particle boundary $P_i$ and compute the
electrostatic potential as a functional of particle configuration.  By
adding the electrostatic force to the hydrophobic attraction force
between particles, we will assess the dependence of elastic moduli on
electric charges using the methods described in
\S\ref{subsec:specific_aim_1}. Results from these proposed calculations
will provide further comparisons and validations of the HAP model
against the continuum mechanics.

Once the effects of lipid charges on the moduli are verified, the PIs
propose to examine how to use an external electric field to control the
amphiphilic self-assembly in solvent. PI Young has a track record of
working on electrohydrodynamics of an elastic, inextensible membrane
using both asymptotic analysis~\cite{Nganguia2013_PRE, Young2014_JFM,
Young2015_PoF} and numerical simulations~\cite{Nganguia2015_CiCP} and
will work with both PIs Ryham and Quaife to extend the HAP model to
study the electromechanical effects on the assembly of amphiphile.
Results from this research will yield a quantitative understanding of
how to utilize an electric field to achieve optimal control of assembly
of amphiphiles in solvent.




