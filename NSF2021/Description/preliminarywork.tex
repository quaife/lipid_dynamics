%\begin{wraptable}[6]{r}{0.56\textwidth}
%\vspace{-0.8cm}
%\centering{
%  \begin{tabular}{|l|r|r|p{0.1\textwidth}|}
%    \hline
%          & HAP                     & experiment          & ref.\\
%    \hline 
%    $\KB$ & $11 \pm 2.5$ \kBT       & 10 \kBT               & \cite{Naetal15,VeBrPa15,NAGLE2000159,PhysRevLett.113.248102}\\
%    \hline 
%    $\KA$ & $34$ \kBT \; nm$^{-2}$  & 30--40 \kBT nm$^{-2}$ & \cite{Nagle17, Nagle17-2}\\
%    \hline 
%    $\KTH$ & $12$ \kBT \; nm$^{-2}$ & 10 \kBT nm$^{-2}$     & \cite{KUZMIN2005, KoNa15} \\ \hline
%  \end{tabular}
%}
%  \vspace{-8pt}
%\caption{\label{tab:moduli} \footnotesize Comparison of values of elastic
%  moduli from the experimental literature and derived by HAP
%  simulation.} 
%\end{wraptable}


\section{Preliminary work}
\label{sec:preliminary_work}


%While the results show great promise in the field of collective body
%hydrodynamics, several outstanding issues need to be addressed. These
%include a thorough analysis of elastic properties of our coarse-grained
%bilayers and efficiently simulating three-dimensional collective
%hydrodynamics of amphiphilic particles.
%We must also incorporate electric charge and account
%for how external fields control particle self-assembly.


%
%The HAP project was initiated in 2017 by PI RR and YNY. The lead author
%on our first paper~\cite{Fu2018_SIAM} was Szu-Pei Fu, who was PI YNY's
%PhD student and is currently a postdoc with PI RR. The collaboration
%involved theoretical work and simulation. The theoretical work included
%model development, proving the first variation formula~\eqref{stress}
%and force-free law, and an energy-based uniqueness theorem for the
%exterior domain problem~\eqref{SL}. 
%Coauthors Andreas Kl\"ockner
%(Computer Science, UIUC, NSF CAREER \#1654756) and his PhD student Matt
%Wala provided their Quadrature by Expansion, or QBX algorithm and Fu ran
%simulations on NJIT clusters.
%
%\begin{wraptable}[11]{l}{0.43\textwidth}
%\centering{
%  \begin{tabular}{|l|l|l|l|}
%    $n$   & (rel.~err.)$_F$
%    &  (rel.~err.)$_G$ \\
%    \hline
%% 32   &   4.07466     &   1.56437\\
%% 64   &   0.24664     &   0.03476\\
%% 128  &   0.00063     &   0.00006\\
%  32  & $4.07 \times 10^{+0}$ & $1.56 \times 10^{+0}$ \\
%  64  & $2.47 \times 10^{-1}$ & $3.48 \times 10^{-2}$ \\
%  128 & $6.30 \times 10^{-4}$ & $6.00 \times 10^{-5}$ 
%  \end{tabular}}
%\vspace{-5pt}
%\caption{\label{tab:spectral_force} 
%\footnotesize  Relative numerical errors (rel.~err.)$_F = \max_{i}
%  \|\mathbf{F}_i-\mathbf{F}_i^{\text{exact}}\|/\|\mathbf{F}_i^{\text{exact}}\|$
%  and (rel.~err.)$_G = \max_{i}
%  \|\mathbf{G}_i-\mathbf{G}_i^{\text{exact}}\|/\|\mathbf{G}_i^{\text{exact}}\|$
%  for force and torque respectively as a function of number of grid
%  points $n$ per particle.  The data is for $N = 5$ particles; two of
%  the particles are nearly touching at a distance 1\% of particle
%  diameter.} 
%\end{wraptable}
PIs RR and YNY started work on HAP modeling with Szu-Pei Fu (then PI
YNY's student and now PI RR's postdoc) and collaborators in 2017. The
interactions between Janus particles (JPs) are formulated as a
second-kind integral equation, which is coupled to the Stokes equations
for the surrounding incompressible fluid at the zero-Reynolds-number
limit. Numerical simulations of a JP suspension revealed self-assembly
of JPs into micelles, bilayers, and vesicles (self-enclosing bilayers),
providing an alternative means for computing mechanical moduli, which
often requires the knowledge of an equation of state from experiments on
a colloidal membrane~\cite{Balchunas2019_SM}.

\begin{wrapfigure}[11]{r}{0.5\textwidth}
\includegraphics[width=0.5\textwidth]{figures/PreliminaryWork/TankTreading.jpg}
\caption{\label{fig:JPv_linearshear}The JP vesicles undergo tank-treading in background shear flow.}
\end{wrapfigure}
The results are summarized in~\cite{Fu2018_SIAM}, where small
particle-number simulations successfully demonstrated gradient-driven
self-assembly of amphiphilic particles into two-dimensional micelles and
bilayer membranes with realistic values of the model parameters: decay
length $\rho=2.5$~nm (based on measurements of hydrophobic attraction
between surfactant-coated surfaces~\cite{Eriksson1989, Lin2005,
Parsegian, Israelachvili80}), particle diameter $\delta = 2.0$~nm (the
physical monolayer thickness~\cite{Boal}), and interfacial tension
$\gamma = 4$~pN nm$^{-1}$ (based on measurements for single-component
bilayer lipid membranes~\cite{GarciaSaez, KUZMIN2005, Petelska2012,
Jackson2016}).
%
%\begin{wraptable}[10]{l}{0.43\textwidth}
%  \vspace{-6pt}
%\centering{
%  \begin{tabular}{|l|l|l|l|}
%    $n$   & (rel.~err.)$_F$
%    &  (rel.~err.)$_G$ \\
%    \hline
% 32   &   4.07466     &   1.56437\\
% 64   &   0.24664     &   0.03476\\
% 128  &   0.00063     &   0.00006\\
%  32  & $4.07 \times 10^{+0}$ & $1.56 \times 10^{+0}$ \\
%  64  & $2.47 \times 10^{-1}$ & $3.48 \times 10^{-2}$ \\
%  128 & $6.30 \times 10^{-4}$ & $6.00 \times 10^{-5}$ 
%  \end{tabular}}
%\vspace{-8pt}
%\caption{\label{tab:spectral_force} 
%\footnotesize  Relative numerical errors (rel.~err.)$_F = \max_{i}
%  \|\mathbf{F}_i-\mathbf{F}_i^{\text{exact}}\|/\|\mathbf{F}_i^{\text{exact}}\|$
%  and (rel.~err.)$_G = \max_{i}
%  \|\mathbf{G}_i-\mathbf{G}_i^{\text{exact}}\|/\|\mathbf{G}_i^{\text{exact}}\|$
%  for force and torque, respectively, as a function of number of grid
%  points $n$ per particle. The data is for $N = 5$ particles; two of the
%  particles are nearly touching at a distance of 1\% of the particle
%  diameter.} 
%\end{wraptable}
%
Results in this work show great potential to study Janus
colloids~\cite{Bradley2017,Mallory2017} and the morphology of colloid
surfactants~\cite{Bradley2016}.  For example, with the flexibility of
the model, we can specify the boundary condition on JP surfaces based on
the chemicals used in experiments.

%The paper \cite{Fu2018_SIAM} considered three types of simulations.
%The first measured bending by
%loading a partially clamped planar bilayer. The second used a harmonic
%bond to dilate a circular bilayer. The third measured tilt using a decay
%equation from~\cite{KUZMIN2005}. These simulations isolated three of the
%five deformations of~\eqref{ansatz3}, enabling us to read off elastic
%moduli from simulation data. The results agreed remarkably well with the
%values reported in the experimental literature.% (Table~\ref{tab:moduli}).
%The agreement is underscored by the fact that the two main HAP
%parameters, $\gamma$ and $\rho$, were assigned physical values from the
%outset rather than being tuned to fit data. 



%
%This study presents a coarse-grained model for JP vesicles that focuses
%on the hydrophobic interactions between coarse-grained lipids and
%hydrodynamic interactions. In contrast to molecular dynamic simulations,
%the hydrophobic interactions exhibit the self-assembly property of the
%lipid bilayer membrane. The proposed energy potential combined with the
%mobility problem for rigid body motions captures membrane mechanics
%including deformations, stretching modulus, and permeability of a JP
%membrane. 
%

%
%This work is highly applicable with fields in material science
%and condensed matter physics such as Janus
%colloids~(\cite{Bradley2017,Mallory2017}) and the setup can be tuned to
%match experimental results such as the morphology of colloid
%surfactants~(\cite{Bradley2016}). 
%%

%This can be an assist to study of interparticle interactions and chemical compositions in industry.

%Integral equation methods are used to solve the mobility problem and the
%previously introduced HAP model. We describe a method to calculate the
%hydrophobic forces and torques on each individual rigid body that avoids
%singular integrals. In addition to the HAP forces and torques, we use a
%short-range repulsion to avoid unphysical contact. 


\begin{wrapfigure}[14]{l}{0.475\textwidth}
%\vspace{-10pt}
\includegraphics[width=0.475\textwidth]{figures/PreliminaryWork/Rupture.jpg}
\caption{\label{fig:JPv_rupture}The JP vesicles rupture under large shear rates.}
\end{wrapfigure}
%
PIs RR, BQ, and YNY initiated a collaboration in summer 2020 to study
the hydrodynamics of JP suspensions. They used the integral formulation
in~\cite{Fu2018_SIAM} and an improved numerical algorithm to simulate
the hydrodynamics of JP vesicles in background flows. Under a linear
shear flow (Figure~\ref{fig:JPv_linearshear}), JP vesicles exhibited
elongation and tank-treading dynamics observed for a lipid bilayer GUV.
%%
%%With the use of suitable parameters, the
%%numerical results yield qualitative phenomena such as a tank-treading JP
%%vesicle. 
%%
%%
%The results showed that the reduced area decreases with shear rate
%\begin{wrapfigure}[15]{l}{0.5\textwidth}
%\includegraphics[width=0.5\textwidth]{figures/PreliminaryWork/Rupture.jpg}
%\caption{The JP vesicles rupture under large shear rates.}
%\end{wrapfigure}
The total length of a JP vesicle is conserved and the decay rate of the
reduced area was independent of the shear rate.  Moreover, the proposed
model describes membrane rupture in high shear rates
(Figure~\ref{fig:JPv_rupture}). Therefore, the method can be applied to
vesicles undergoing topological changes which is difficult to simulate
when using a continuum model that represents vesicles as closed and
continuous curves.
%
%Our coarse-grained model is also able to calculate inter-monolayer
%friction. 
%

%
%Our method requires
%calculations of the hydrodynamic stress tensor, and these are used to
%
The PIs used the simulation data to estimate the inter-monolayer
friction $b$, membrane permeability constant $P$, and the membrane
stretching modulus $K_A$~\cite{chabanon2017, sch_vla_mik2010}. The range
of friction coefficients agree with values reported
by~\cite{denOtter2007} in their molecular dynamics study. The
coarse-graining level of the JP vesicle has a larger length scale than
molecular dynamics simulations, and we are now conducting convergence
studies to investigate how physical properties of the assembly like the
friction coefficient and membrane permeability depend on the JP particle
shape and size.

\begin{wrapfigure}[11]{l}{0.6\textwidth}
\includegraphics[width=0.6\textwidth]{figures/PreliminaryWork/Slipper.jpg}
\caption{\label{fig:JP_poiseuille}The JP vesicle takes on the asymmetric
  slipper shape in parabolic flow.}
\end{wrapfigure}
The study also simulated the spatial migration of a JP vesicle in a
parabolic shear flow (Figure~\ref{fig:JP_poiseuille}). Replicating the
hydrodynamics of a GUV in a Poiseuille flow~\cite{Kaoui09,
dan_vla_mis2009, cou_kao_pod_mis2008}, the JP vesicle moves toward the
center of the shear flow, and has a slight decrease in reduced area.
For the parameters we used in the simulation, the JP vesicle takes on an
asymmetric, ``slipper" shape as it settles above the centre of flow and
exhibits tank-treading motion. A qualitative agreement is made by
comparing the JP vesicle shape with the two-dimensional inextensible
membrane in continuum modeling simulations~\cite{Kaoui09,
dan_vla_mis2009, cou_kao_pod_mis2008}. An interesting result contrast
with continuum results is that the JP vesicle oscillates at a height
slightly above the centre of the Poiseuille flow. 

%The single JP vesicle simulations end with obtaining JP bilayer stretching 
%modulus $K_A$ and hydraulic permeability constant $P$. We extract these 
%physical quantities from the Laplace pressure relationship that $R\Delta P$ is 
%proportional to JP bilayer stretch. We describe the aqueous flux in membrane 
%using a differential equation~\ref{eq:perm} and solve for the theoretical 
%permeability constant $P$. In spite of producing these values with some discrepancy
%using the default parameter set, we provide a new framework in measuring coarse-grained 
%membrane permeability.


\begin{wrapfigure}[15]{l}{0.5\textwidth}
\includegraphics[width=0.5\textwidth]{figures/PreliminaryWork/ShearDoublet.jpg}
\caption{\label{fig:JPv_interactions} Vesicle-vesicle interaction for JP
  vesicles in shear flow.}
\end{wrapfigure}
The PIs further simulated the hydrodynamics interactions between two JP
vesicles (Figure~\ref{fig:JPv_interactions}), and drew comparisons with
simulation results of two vesicles described by the Helfrich continuum
model. A comparison of the vesicle shapes and the streamlines
demonstrate remarkable similarities. The two overlapping trajectories of
the JP vesicles' centroids in a shear flow evolve as expected when the
two centroids are initialized on the same horizontal level. The PI
observed a rotating behaviour that is observed for models involving
vesicle adhesion~\cite{qua-vee-you2019, abb-far-ezz-ben-mis2021}. The
hydrophobic attraction led to this adhesive effect when two JP bilayers
are sufficiently close. We also performed several simulations of a pair
of JP vesicles suspended in an extensional flow. By varying the initial
vertical displacement of the vesicles' centroids, we can control for
divergent trajectories and obtain similar results to the continuum
model.

One of the main mathematical innovations in the PIs collaboration is a
novel alternative integral form for calculating the force and torque
that avoids singular integral evaluation. These alternative integrals
allow the accurate resolution of trajectories over long times without
having to rely on computationally expensive quadratures. Future goals
include extending the current framework to a three-dimensional JP
vesicle system. This will require additional algorithmic implementation
including a fast summation method such as the fast multipole method.

%
%\begin{wrapfigure}[22]{l}{0.5\textwidth}
%\vspace{-5pt}
%  \centering{\includegraphics[width=0.49\textwidth]{Figures/TankTreading.pdf}}
%  \vspace{-10pt}
%  \caption{\label{fig:tank_treading} \footnotesize
%    Panel A: A two-dimensional, particle-based vesicle in a shear flow
%    (arrows); $\mu = 1$ mPa s and $\dot{\gamma} = 0.5$ $\mu$s$^{-1}$
%    (compare with~\cite{Brandner2019}).  Panel B: Angular velocities for
%    the outer and the inner leaflets.  Panel C: The trajectory of the
%    particle centers forms two ellipses with well-defined inclination
%    angle $\Theta$. Panel D: The evolution of the inclination angle.
%    Panels B and D share the same time axis.}
%\end{wrapfigure}
%We have simulated two-dimensional vesicles in shear flow
%$\mathbf{u}_{\infty} = \dot{\gamma} x_2 \mathbf{i}_1$ with shear rate
%$\dot{\gamma}$,
%%(The symbol shear rate is unrelated to interfacial tension.)
%and the particle-based system mimics the behavior observed in continuous
%vesicles~\cite{torres-sanchez_millan_arroyo_2019,
%mahapatra_saintillan_rangamani_2020, Steigmann99, C6SM02452A}. The basic
%pattern is that of a vesicle bilayer approaching a steady, tank-treading
%ellipse (Figure~\ref{fig:tank_treading}C). We analyzed the simulation
%data in the context of theoretical work on
%tank-treading~\cite{Finken2008, PhysRevLett.106.158103}.
%Jeffrey's equation predicts a tank-treading frequency $-\dot{\gamma}/2$ (dashed line in
%Figure~\ref{fig:tank_treading}B). In the particle simulation, we measured
%a tank-treading frequency as the mean of the particle angular
%velocities. The measured steady-state frequencies were within 7\% of the predicted 
%frequency (dotted and solid curves in Figure~\ref{fig:tank_treading}B).
%The inclincation angle of the particle-based simulation also
%follows the inclination angle $\Theta(t)$ predicted by Jeffrey's
%equation (Figure~\ref{fig:tank_treading}D), albeit with higher frequency oscillation
%coming from particle reorganization.
%
%Analyzing for any underlying constitutive laws, the particles reorganize
%so that there is an initial increase and then plateau in vesicle
%circumference. As is the case with real lipid bilayer membranes, the
%particle collection permits a small amount of stretching. The
%theoretical calculations give an increase of length by 3.6\%, in
%agreement with our measured length increase. The simulations also
%suggest that the particle system behaves as a semipermeable
%vesicle~\cite{323e9a2f0c58487ea82518d7a1f96485, YAO2017728}.
%
%
%\begin{wrapfigure}[10]{r}{0.22\textwidth}
% \vspace{-3pt}
%  \centering{\includegraphics[width=0.21\textwidth]{Figures/PW_fig5.pdf}}
% \vspace{-7pt}
%  \caption{\label{fig:flow_rupture} \footnotesize Rupture of a
%  two-dimensional vesicle at large shear rates.}
%\end{wrapfigure}
%The above considerations further justify using the particle-based model
%to study material properties as outlined in
%\S\ref{subsec:specific_aim_1}. Furthermore, the particle system allows
%for intermonolayer slip~\cite{SHKULIPA2005823, ShkulipaThesis}. In
%Figure \ref{fig:tank_treading}B, the inner angular velocity is slightly
%less than the outer angular velocity. Intermonolayer slip enters
%zero-thickness membrane models as a velocity jump boundary condition
%\cite{schwalbe_vlahovska_miksis_2010}, for example, but in the present
%model, it is a consequence of monolayer independence. For large shear
%rates, viscous forces exceed the hydrophobic attraction leading to
% vesicle rupture~\cite{C8SM01501E}, at which point the flow carries away
% segmented membrane patches (Figure \ref{fig:flow_rupture}). 
%
%In conclusion, the HAP model accurately mimics the behavior of continuous
%membranes while at the same time capturing the
%reconnection during the topological change of 
%lipid molecules on the scales of membrane thickness,
%overcoming one of the long-standing challenges of continuum modeling.
%



%%%%%%%%%%%%%%%%%%%%%%%%%%%%%For Specific Aim 3
%The background flow enters by replacing the third equation in \eqref{eq:stokes} 
%with the condition $(\mathbf{u} -\mathbf{u}_{\infty})(\mathbf{x}) 
%\to \mathbf{0}$ as $|\mathbf{x}| \to \infty$. To incorporate the far-field flow, 
%and using the representation 
%\begin{align}
%\label{PowerMiranda}
%  {\bf u} = {\bf u}_{\infty} + K\boldsymbol{\eta} + 
%    \sum_{i=1}^N S(\cdot, {\bf a}_i) {\bf F}_i + 
%                 R(\cdot, {\bf a}_i) {\bf G}_i.
%\end{align}
%The symbols $S$ and $R$ are Stokeslets and rotlets supported at the respective
%particle centers \cite{leal_2007} and $K\boldsymbol{eta}$ is a layer potential for the
%unknown density function $\boldsymbol{\eta}.$ The
%representation~\eqref{PowerMiranda} automatically satisfies all
%equations, with the exception of the rigid motion conditions. The rigid
%motion conditions follow by requiring the viscous stress vanishes across
%the particle boundaries.

%\begin{wrapfigure}[13]{l}{0.30\textwidth}
%\centerline{\includegraphics[width=0.30\textwidth]{figures/BIError.pdf}}
%  \vspace{-8pt}
%\caption{
%\label{fig:bierror}  
%\footnotesize The false color map shows how numerical quadrature of
%  layer potentials loses accuracy near the particle boundaries.  The
%  color bar is for $\log_{10}.$}
%\end{wrapfigure}
%\textbf{Novel reciprocal identities.} 
%The force and torque formulas~\eqref{forceandtorque} require the
%hydrophobic stress along the particle boundaries. Unfortunately,
%standard quadrature formulas to estimate $\nabla u$ on the particle
%boundaries introduces large amounts of quadrature error
%(Figure~\ref{fig:bierror}). Therefore, it is useful to devise reciprocal
%identities for the force and torque on body $i$ that does not
%require integration along body $i$. One identity, that we have proved,
%is
%\begin{align}
%    \label{eq:reciprocal}
%{\bf F}_{\text{hydro},i} = \sum_{j \neq i} \int_{\partial P_i}
%[\boldsymbol{\sigma}_{ij} + \boldsymbol{\sigma}_{ji}]\boldsymbol{\nu}\,\dif S,\quad
%{\bf G}_{\text{hydro},i} = \sum_{j \neq i} \int_{\partial P_i} ({\bf
%  x}-\mathbf{a}_i) \times [\boldsymbol{\sigma}_{ij} +
%  \boldsymbol{\sigma}_{ji}]\boldsymbol{\nu}\,\dif S, 
%\end{align}
%for $i=1,\ldots,N$. Here, $u_i$ is the solution of the screened Laplace
%equation when only the contribution from particle $i$ is considered and
%$\boldsymbol{\sigma}_{ij} = \rho^{-1} u_iu_j {\bf I} + \rho(\nabla u_i
%\cdot \nabla u_j {\bf I} - 2 \nabla u_i \otimes \nabla u_j)$.
%Because of the double layer potential jump relations, \eqref{eq:reciprocal} 
%does not require knowledge of $\nabla u$ on the particle boundary, and this
%is enormously beneficial for calculating force and torque.



%\subsection{Two-dimensional vesicle hydrodynamics in shear flow} 
%The motion of vesicles in shear flow is an important problem in the
%applied mathematics because simulations can reveal mechanical
%properties of membranes and lead to an enhanced understanding of
%deformable particle laden flows \cite{Sinha15}. 
%
%
%\begin{wrapfigure}[10]{r}{0.2\textwidth}
%\centerline{\includegraphics[width=0.2\textwidth]{figures/PW_fig5.pdf}}
%\vspace{-8pt}
%\caption{\label{fig:rupture} \footnotesize Rupture of tank-treading vesicle under strong shear flow.}
%\end{wrapfigure}
%
%To implement a vesicle in shear flow in the context of hydrophobic
%potentials and mobility problem, we consider a shear flow in the
%far-field $\mathbf{u}_{\infty} = Uy\mathbf{i}_x$ in the direction of the
%$x$-axis (Figure \ref{fig:tanktreading}A). As illustrated in Figure
%\ref{fig:tanktreading}, the particle based approach supports
%inter-leaflet slip, and this can be used to determine inter-leaflet and
%in-plane shear viscosities. 
%
%This field satisfies the linear Stokes system but does not give rise to a rigid motion at the particle interfaces. 
%To have a rigid motion, we change variables $\mathbf{u} = \tilde{\mathbf{u}}+ \mathbf{u}_{\infty}$ and 
%for the new field $\tilde{\mathbf{u}}$ vanishing at infinity we let 
%$\tilde{\mathbf{u}}|_{\partial P_i} = \mathbf{v}_i + \boldsymbol{\omega}_i \times (\mathbf{x} - \mathbf{a}_i)$ 
%where $(\mathbf{v}_i,\boldsymbol{\omega}_i)$ are the unknown translation and angular velocities of the 
%$i$th particle $P_i.$  
%
%\begin{wrapfigure}[17]{l}{0.4\textwidth}
%\centerline{\includegraphics[width=0.4\textwidth]{figures/PW_fig2.pdf}}
%\caption{\label{fig:demixing} An initial assembly of small and 
%large particles spontaneous segregates into two smaller bodies. }
%\end{wrapfigure}
%The HAP simulations show vesicle tank-treading. Under the external shear flow, the initially circular 
%vesicle rotates in the clockwise direction. As the rate of rotation increases, the vesicle approaches
%a steadily tank-treading ellipse. In Figure \ref{fig:tanktreading}B-D, the solid curves are ellipses fit to the particle centers
%and midplane respectively. In the non-dimensionalized system, the particles have diameter 2, on the order of $\rho,$ 
%and the vesicle diameter is about 14. 
%\todo[inline]{missing units. nm?}
%Figure~\ref{fig:tanktreading}E shows the aspect ratio of the major to minor axes reaching an equilibrium value in the 
%red and blue curves, yet oscillating in the high-shear rate (yellow) curve.
%The tank-treading vesicle elongates and becomes more horizontal 
%with an increase in flow rate or 
%with a decrease in stiffness (effected by decreasing $\rho = 4$ to $\rho = 2$). 


%For large shear flow rates, there is an increase in arc length. Here arc
%length refers to the the mid-plane circumference. Thus, some of the
%external force is going into stretching the vesicle---the other part is
%going into bending and viscous dissipation. From our experiments, we
%find that the vesicle ruptures once stretching exceeds about 5 \% (see
%Figure \ref{fig:rupture}). Finally, movies of the tank-treading motion
%show a slip velocity between the outer and inner leaflets Figure
%\ref{fig:tanktreading}G. We have illustrated this by tracking the
%distance between two reference particles in the inner and outer leaflet
%(Figure \ref{fig:tanktreading}B \& D, green and blue particles). With
%moderate shear rates or greater adhesion, the particle pair moves in
%tandem (in Figure \ref{fig:tanktreading}H, blue and red curves, their
%distance is more or less constant). For a large shear rate, the
%particle separates as the two leaflets slide against one another. 

%\begin{figure}
%\begin{center}
%\includegraphics[width=1\textwidth]{figures/PW_fig1A-D.pdf}
%\includegraphics[width=1\textwidth]{figures/PW_fig1E-H.pdf}
%\end{center}
%\vspace{-0.3in}
%\caption{\label{fig:tanktreading}\footnotesize (A) A vesicle formed by
%  amphiphilic particles in shear flow, and the tank-treading motion
%  (B)--(D). The separation of particle pairs in (B) and (C) illustrate
%  inter-leaflet slip.  (E)--(G) Tank-treading reaches a steady state in
%  elliptical aspect ratio, major-axis angle, and circumference.}
%\end{figure}


%\begin{wrapfigure}[12]{r}{0.2\textwidth}
%\centerline{\includegraphics[width=0.2\textwidth]{figures/PW_fig5.pdf}}
%\caption{\label{fig:rupture} Rupture of tank-treading vesicle under strong shear flow.}
%\end{wrapfigure}
%




