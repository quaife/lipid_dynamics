\subsection{Specific Aim 3: Dynamics of self-assembly by HAP under an external flow/field}
\label{subsec:specific_aim_3}

\subsubsection{Vesicles in background flows}
Numerically solving the Stokes and screened Laplace system~\eqref{SL}
and~\eqref{eq:stokes} is nontrivial, and we will develop efficient,
robust, high-order in time and space algorithms suited to the problem.
The challenges are: {\bf (1)} The equations express a two-way coupling
since the flow changes the position of the suspended particles,
and in return the geometry-dependent hydrophobic stresses impart a force
on the flow. {\bf (2)} The inputs to the Stokes equations are particle
configurations, forces, and torques. The outputs are the rigid body
translation and angular velocities used to update the particle positions
(see Figure~\ref{fig:flow_map}). Although the underlying equations are
linear, the overall problem is highly nonlinear because the domain is
constantly changing. {\bf (3)} Self-assembly causes the particles to
come into close contact. As a result, an exceptional spatial accuracy is
required to resolve the fields between adjacent particles. Finally, the
physically relevant elastic properties of bilayer become apparent at
large length scales and for large particle-numbers, which increases the
computational complexity of our simulations. 

We include a background flow by replacing the third equation in
\eqref{eq:stokes} with the condition $\mathbf{u}(\mathbf{x})
\to \mathbf{u}_{\infty}(\mathbf{x})$ as $|\mathbf{x}| \to
\infty$. To incorporate the far-field flow, we use the representation 
\begin{align}
\label{PowerMiranda}
  {\bf u} = {\bf u}_{\infty} + K\boldsymbol{\eta} + 
    \sum_{i=1}^N \left(S(\cdot, {\bf a}_i) {\bf F}_i + 
                 R(\cdot, {\bf a}_i) {\bf G}_i\right),
\end{align}
where $S$ and $R$ are stokeslets and rotlets supported at the respective
particle centers~\cite{leal_2007} and $K\boldsymbol{\eta}$ is a layer
potential for the unknown density function $\boldsymbol{\eta}$. With the
exception of the rigid body conditions, the
representation~\eqref{PowerMiranda} automatically satisfies all the
hydrodynamic equations including the far-field condition. The rigid
motion conditions follow by requiring the viscous stress vanishes across
the particle boundaries.
%
We will compare the behavior of our particle-based vesicles in Stokes
flow to well-established models that enforce area incompressibility and
volume conservation in vesicle membrane
dynamics~\cite{torres-sanchez_millan_arroyo_2019,mahapatra_saintillan_rangamani_2020, Steigmann99, C6SM02452A}.
Some preliminary results are in \S \ref{sec:preliminary_work}. 
%
%Our
%preliminary results show that the particle-based bilayers have the same
%large area modulus as real lipid bilayer membranes, so we expect
%realistic results from our flow simulations. Under moderate shear rates,
%the vesicle elongates and takes on the tank-treading elliptical shape.
%We can directly check for area compressibility and volume conservation
%properties from particle simulations by tracking the change in area
%density of the midplane as a function of shear flow rate, along with
%evaluating the tangential divergence of the velocity.
%
%Bilayer membranes have a small permeability to
%water~\cite{323e9a2f0c58487ea82518d7a1f96485}, and modelers often assume
%a vesicle membrane that conserves volume. In our particle-based
%approach, water motion across the bilayer is limited by the size of the
%gaps between the particles, which is an artifact of coarse-graining.
%Making these gaps smaller comes at the expense of numerical accuracy,
%and we will assess if there is a reasonable trade-off between
%approximate volume conservation and simulation complexity. 
%
%Finally, the particle-based vesicle bilayers have two distinct leaflets
%consisting of the particles in contact with the vesicle interior, and
%those in contact with the surrounding fluid. Since there is a nonzero
%separation between these layers, an important effect we observe is that
%the leaflets slide against one another under shear flow. This implies
%that part of the viscous dissipation in the aqueous phase is enhanced by
%intermonolayer friction~\cite{SHKULIPA2005823, ShkulipaThesis}.
%Intermonolayer slip enters zero-thickness membrane models as a velocity
%jump boundary condition and leads to multiple solution
%states~\cite{schwalbe_vlahovska_miksis_2010}. In the present proposal,
%intermonolayer slip is an artifact of monolayer independence and we can
%control for slip as a function of shear rate, vesicle diameter, and
%particle geometry.

\subsubsection{Electromechanical effects on the dynamic assembly of amphiphiles \label{subsubsec:em_effects}}
Using flicker spectroscopy of the shape fluctuations of a giant
unilamellar vesicle (GUV), the bending rigidity is determined as a
function of lipid composition from 0 to 100 mol $\%$ of charged lipids
in recent experiments~\cite{FaizEtAl2019_SoftMatt}---membrane bending
rigidity increases with lipid surface charge, while decreases with salt
in the bulk due to the screening of the lipid surface charge. This
agrees with several theoretical models~\cite{Kralchevsky1996_JCIS,
May1996_JChemPhys, LoubetEtAl2013_PRE} that also assume the quadratic
form of the elastic energy density in the presence of surface charge and
bulk charge~\cite{DuplantierGoldstein1990_PRL, Winterhalter1992_JPC}. As
the electrostatic interaction is non-local in nature, we expect that the
controversy of the HK elastic energy form (see
\S\ref{subsec:specific_aim_1}) would be worse in the presence of
electrostatic interactions and electrokinetics. The PIs will extend
the approaches in \S\ref{subsec:specific_aim_1} to charged lipids to
calculate the bending moduli and compare against the experimental
results in~\cite{FaizEtAl2019_SoftMatt}. We propose to incorporate an
explicit surface charge on each particle boundary $P_i$ and compute the
electrostatic potential as a functional of particle configuration.
Adding the electrostatic forces to the hydrophobic attraction forces
between particles to compute the particle assembly dynamics, we can then
calculate elastic moduli as outlined in \S\ref{subsec:specific_aim_1}.
Results from these proposed calculations provide further comparisons and
validations of the HAP model against the continuum mechanics.

Once the effects of lipid charges on the moduli are verified, the PIs
propose to examine how to use an external electric field to control the
amphiphilic self-assembly in solvent. PI YNY has a track record of
working on electrohydrodynamics of an elastic, inextensible membrane
using both asymptotic analysis~\cite{Nganguia2013_PRE, Young2014_JFM,
Young2015_PoF} and numerical simulations~\cite{Nganguia2015_CiCP}, and
will work with both PI RR and BQ to extend the HAP model to study the
electromechanical effects on the assembly of amphiphile. Results from
this research will yield a quantitative understanding of how to utilize
an electric field to achieve optimal control of assembly of amphiphiles
in solvent.




