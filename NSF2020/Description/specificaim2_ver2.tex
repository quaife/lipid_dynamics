%\subsection{Specific Aim 2: Functional Materials}
\subsection{Specific Aim 2: Hydrodynamics}
\label{subsec:specific_aim_2}
Specific Aim 1 dealt with the collective material property of amphiphiles which does not involve the movement  of the solvent.
In Specific Aim 2, we include hydrodynamics effects. Hydrodynamic effects are relevant
because the rates of biological functions like fusion, fission, and pore dynamics rely on viscous dissipation \cite{RYHAM20112929}. 
Additionally, coarse-grained and molecular dynamics theories include water either explicitly or implicitly in their models. 

A further area of relevance is in the fabrication of complex microscopic three-dimensional structures \cite{Cho2010}.
Over the past decade, there has been an explosion of interest in small-scale processes that utilize capillary forces dominate,  van der Waals interactions and thermal noise  \cite{Zhang2017},
to coordinated movement and bind  material subunits. Two prominent fabrication techniques are capillary origami 
\cite{Pandey2011,Leong2007,Reynolds2019} and colloidal self-assembly \cite{Dasgupta2017,Siontorou2017}. Capillary origami uses principles of elastocapilirity wherein elastic solids deform under surface tension \cite{Bico2018,VanHonschoten2010}. In self-assembly of colloids, 
particles exhibit similar thermodynamic behavior of molecular systems, except that the behavior occurs over observable, long time scales \cite{Zhang2017}. 

The hydrodynamic interactions enter through the transmission of viscous stresses from one particle to the another.
This involves solving the system 

\setcounter{midequation}{\theequation}
\addtocounter{midequation}{2}

\begin{minipage}[t]{0.44\textwidth}
\begin{subequations}
\begin{alignat}{2}
\label{St1} -& \mu \Delta {\bf u} + \nabla p = {\bf 0}, \\
\label{St2}  & \nabla \cdot {\bf u} = 0,                &&   \text{in } \Omega\\
\label{St3}  & {\bf u }({\bf x}) = {\bf v}_i + {\bf w_i}\times {\bf x}, \;  && \text{for } {\bf x} \in \partial P_i\\
\label{St4}  & ({\bf u } - {\bf u}_{\infty})({\bf x}) \to {\bf 0} && \text{as } {\bf x} \to \infty\\
\notag \\
\tag{\themidequation}
\label{HSB1}  & \int_{\partial P_i} \boldsymbol{\sigma} \boldsymbol{\nu} \,\dif S + {\bf F}_{i} = {\bf 0}\span \span
\end{alignat}
\end{subequations}
\end{minipage}
\addtocounter{equation}{1}
\setcounter{midequation}{\theequation}
\addtocounter{midequation}{2}
\begin{subequations}
\begin{minipage}[t]{0.5\textwidth}
\begin{alignat}{2}
\label{SL1}  - & \rho^2 \Delta u +u = 0, && \text{in } \Omega\\
\label{SL2}   & u({\bf x}) = f_i({\bf x}),\quad  && \text{for } {\bf x} \in \partial P_i\\
\label{SL3} &  u({\bf x}) \to 0 && \text{as } {\bf x} \to \infty \\
\notag \\
\notag \\
\tag{\themidequation}
\label{HSB2}   & \int_{\partial P_i} {\bf x} \times \boldsymbol{\sigma} \boldsymbol{\nu} \,\dif S + {\bf G}_{i} = {\bf 0} \span \span\\
\notag
\end{alignat}
\end{minipage} 
\end{subequations}


\noindent for $i = 1,\dots, N$.
The inputs to \eqref{St1}--\eqref{HSB2} are the collection of particle configurations.
Within this system, the activity $u$ solves the screened Laplace equation \eqref{SL1} with the material label boundary condition \eqref{SL2}
and far-field condition \eqref{SL3}. The forces ${\bf F}_i$ and torques ${\bf G}_i$ come from \eqref{totalFG}.
In turn, the fluid velocity ${\bf u}$ solves the Stokes equations (\ref{St1}, \ref{St2}) with background flow boundary condition (\ref{St3}).
The force and torque equations \eqref{HSB1}, \eqref{HSB2} couple the (\ref{St1}--\ref{St4}) to (\ref{SL1}--\ref{SL3}).
The outputs are the collection of velocity and rotation pairs $({\bf v}_i, {\bf w}_i)$ that uniquely satisfy the rigid body boundary condition \eqref{St3}. 

The particle velocity and rotation $({\bf v}_i, {\bf w}_i)$ allow us to update the particle position:
\begin{alignat}{3}
\label{Hupdate1}   \frac{\dif {\bf a}_i}{\dif t} = {\bf v}_i,      \quad  \frac{\dif {\bf R}_i}{\dif t}  = A({\bf w}_i).
\end{alignat}
Recall that $A({\bf w}_i)$ is the skew-symmetric matrix with axial vector ${\bf w}_i$.
Also, ${\bf a}_i$ is the particle center the rotation matrix ${\bf R}_i$ defines the orientation.
That way, the particle configurations come from the flow maps $F_i({\bf x},t) = {\bf a}_i(t) + {\bf R_i}(t){\bf x}$ with
$P_i(t) = F_i(P_{0i},t)$ and $f_i(F_i({\bf x},t)) = f_{i0}({\bf x})$ where $P_{i0}$ and $f_{i0}$ are the $i$th reference particle
and material label respectively.






