\subsection{Specific Aim 2: Efficient high-order numerical algorithm for large-scale simulation of the HAP model}\label{subsec:specific_aim_2}
% -----------------------------------------------------------------------------
% {{{
\subsubsection{Overview of the potential theory and boundary integral equation methods}
The HAP model requires solving the exterior Dirichlet problem of the screened Laplace
equation and the mobility problem of the Stokes flow at each time step.
Standard numerical methods such as finite difference and finite element
methods have to truncate the infinite domain to a finite
computational domain by imposing certain artificial boundary conditions
on the truncated domain and need the discretization of the whole volume
of the truncated domain. The artificial boundary conditions are very often
inexact, introducing another layer of approximation that
lowers the accuracy of the solution. The discretization of the whole volume is
very expensive
for three dimensional problems, leading to excessively large number of unknowns in
the linear system. Furthermore, it is rather difficult to obtain high-order
discretization when the boundary surfaces are irregular.

Potential theory and boundary integral equation (BIE) methods remove the aforementioned
obstacles in a very elegant way. The starting point is that for constant-coefficient
PDEs the fundamental solutions, i.e., the Green's functions, are readily available.
This allows us to represent the solution via the so-called layer potentials
that are the convolution integrals of a kernel and an unknown density
only on the boundary. The kernel is a linear combination of the Green's function
and its derivatives, which ensures that the representation satifies the underlying
PDE and the condition at infinity automatically. And the boundary condition
at the material interfaces together with the jump relation of the layer potential
leads to a boundary integral equation for the unknown density.

The BIE method removes the need of imposing artificial boundary conditions
for exterior problems
and reduces the dimension of the problem by one, leading to optimal number of unknowns
in the solve phase. Very often, one is able to construct the so-called
{\it second kind integral equation} (SKIE) formulation for these problems,
where the operator is a sum of the identity operator and a compact operator.
It is well-known that the condition number of the
resulting linear system for SKIEs is {\it independent} of the mesh size $h$ or equivalently,
the number of discretization points. This leads to
a constant number of iterations when iterative solvers are applied to solve
the resulting linear system. It further
improves the accuracy of the solution as the relative error of solution
will never increase as the number of total discretization points increases.
On the other hand, the condition
number of the linear system from finite difference and finite element methods generally
increases as a power function of the total number of dicretization points, with
the exact power depending on the underlying spatial dimension and the highest
order of the differential operator in the PDE. This leads to an increase in the number
of iterations (and hence the total computational cost) and an increase in the relative
error of the solution as the number of dicretization points increases.

The discretization of the integral operator often leads to a dense matrix,
while finite difference/finite element methods result in sparse matrices,
making matrix-vector product much cheaper to compute. However,since the invention
of the original fast multipole method (FMM)\cite{fmm5}, there
have been many fast algorithms~\cite{fmm1,fmm2,fmm3,fmm4,fmm6,fmm7,fmm8}
that reduce the cost of matrix-vector product
to linear ($O(N)$) or quasilinear complexity ($O(N \log^kN)$) for dense matrices resulting
from the BIE discretization.
Fast direct solvers have also been developed
recently~\cite{fds1,fds2,fds3,fds4,fds5,fds6,fds7,fds8,ho2016cpam2,ho2016cpam1,minden2016,minden2017siammms}.
The development of these fast algorithms have removed a huge
hurdle for the use of boundary integral equations, making the BIE method the method
of choice for constant-coefficient PDEs with linear boundary conditions. For the exterior
Dirichlet problem of the screened Laplace equation, the SKIE formulation in our previous
work~\cite{Fu2018_SIAM} can be readily extened to the 3D case. For the mobility problem
of the Stokes flow in three dimensions that
takes the hydrodynamic effect into account, an SKIE formulation
can be found in \cite{manasthesis}.

\subsubsection{High-order discretization of surface integrals in three dimensions}
The practical application of integral equation methods requires the
accurate evaluation of boundary integrals with singular, weakly
singular or nearly singular kernels. There are many numerical quadrature schemes
for dealing with line integrals in two dimensions~\cite{alpert,kapur,sidi,duffy,bruno1,bruno2,davis_1984,graglia_2008,hackbusch_sauter_1994, jarvenpaa_2003,khayat_2005,kress_boundary_1991,schwab_1992, ying_2006,beale1,beale2,goodman_1990, haroldson_1998, lowengrub_1993,schwab_1992,ggq1,ggq2,ggq3,helsing_2008a,helsing_integral_2009,helsing_tutorial_2012}
and some of them are extremely efficient and can achieve arbitrary high order.

The high-order quadrature rules for the evaluation of surface integrals
in three dimensions, however, are much less developed than the line integrals in two
dimensions. For example, there are no Gaussian quadratures for integraing
polynomials on a flat triangle, even though efficient quadratures
\cite{xiao2010cma,vioreanu2014} have been
developed recently for such purpose. For weakly singular or singular integrals,
\cite{bremer2012jcp,bremer2013jcp} constructed high-order quadratures
for surface integrals on a general triangle, while \cite{gimbutas2013sisc}
presented a fast algorithm for integrating $1/r$-type singular integrals
for surfaces that are homeomorphic to a sphere. We plan
to study the so-called quadrature by expansion (QBX) scheme~\cite{klockner2013jcp,qbx2}
for the evaluation
of both singular and near-singular surface integrals encountered in the
discretization of BIEs in three dimensions. Conceptually, the idea of the QBX
to evaluate singular, hypersingular and near singular integrals
on smooth surfaces is straightforward. That is, the surface is discretized
into smooth triangles and smooth high-order quadratures are applied to evaluate
the expansion coefficients
on all source triangles with the QBX expansion center placed at a point off the surface.
One may then form a suitable expansion (for example, a Taylor expansion) around that
center and evaluate this expansion back at the target point on the surface (or close
to the surface in the near singular case). Compared to the competing aforementioned
quadrature schemes, the QBX quadrature is attractive
because it offers a clear path for being extended to: \textbf{(1)}
handle any singularity, including
hypersingular operators, \textbf{(2)} be usable with any high-order surface
discretization, \textbf{(3)} generate well-conditioned discrete
operators to which iterative methods such as GMRES~\cite{gmres} can be
applied in a black-box fashion, \textbf{(4)} integrate well with fast algorithms
such as the FMM. 

In practice, there are still many issues that need to be resolved.
For example, there are now many variants of QBX including global and local
QBX~\cite{klockner2013jcp,rachh2017jcp}, the target-specific QBX~\cite{siegel2018jcp},
kernel-independent QBX~\cite{abtin2018bit}, and
quadrature by two expansions~\cite{ding2019arxiv}. The coupling of the QBX
and the FMM may also lead to certain instability issues which may require
some changes in the fast multipole method~\cite{wala2018jcp}. Similar
to other quadrature methods, there have been extensive study on the QBX
methods in two dimensions, while its three dimension
version~\cite{wala2019jcp,af2018sisc,siegel2018jcp,wala2019arxiv} has not been
fully studied and the implementation is even more scarce. We plan to investigate
the accuracy and the convergence order of the various QBX schemes mentioned above,
its coupling with the FMM, parallel implementation issues for large-scale
problems, and the application to our target problems.
\subsubsection{A hybrid method for close-to-touching Janus particles}
The HAP model for membranes will inevitably lead to the case where some Janus particles
will be very close to each other. Even though QBX can handle such case
accurately, it may lead to excessively large number of unknowns in order to resolve
the unknown densities at those close-to-touching points. In \cite{gan2016sisc},
a hybrid scheme is developed to treat the close-to-touching spheres for the interface
problem of the Laplace equation. The key idea is to combine the analytic image method
and spectrally accurate method of moments, where the analytic image method will capture
the most singular part of the unknown density at the close-to-touching points.
With the aid of the FMM, the scheme achieves linear complexity with much less number
of unknowns on the material interfaces.
We plan to study the hybrid scheme
to treat the close-to-touching case for both the screened Laplace equation and the
mobility problem to further improve the efficiency of the numerical algorithm
for the HAP model.
\subsubsection{Other numerical issues}
Another outstanding numerical issue pertaining to simulating the self-assembly of amphiphilic particles (such as lipid macromolecles) in a
viscous solvent is the collision between amphiphilic particles.
The hydrophobic attraction potential drives the amphiphilic particles to move towards each other so to minimize exposure to the solvent. 
Under such
attraction force the fluid is being squeezed out as particles move toward each other, and the amphiphilic particles can come into physical
contact in finite time. 
Such particle collisions in a dense rigid body suspension is a great challenge and can be a bottleneck in large-scale simulations. 
In our previous work a Leonard-Jones potential is used to keep the
particles from overlapping \cite{Fu2018_SIAM}. 
Such steep steric interaction at
short ranges introduces great numerical stiffness that limit the time step size. 

Yan {\it et al.} proposed
a collision-resolution algorithm within the framework of boundary integral methods to resolve the non-smooth many-body
dynamics due to collisions by formulating the particle collision process as a linear complementarity problem with geometric 
`non-overlapping' constraint \cite{Yan2019}. 
We propose to treat the particle collisions based on Yan {\it et al.}'s work after the first two goals (see above)
are achieved. 
A potential complication in combining the algorithms for particle collision into our 
QBX integral formulation is the computational cost for the QBX to refine meshes on particles in collision.

Once the collision-resolution algorithm is implemented, we will check the computing performance first before we embark on 
incorporating fluctuating hydrodynamics \cite{Bao17,Bao18}, which might be important even at the scales of the amphiphilic 
particle size in some cases. We propose to extend the high-order
time-integration scheme for a stochastic differential equation
\cite{fu2015pre} to
this system.

