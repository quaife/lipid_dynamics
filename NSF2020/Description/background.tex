
\noindent
{\bf {Collaborative Research: Mathematical modeling and coarse-grained simulations of self-assembly of amphiphilic Janus particles in a solvent}}\\
{\it Rolf Ryham (lead PI, Fordham University)}\\
{\it Shidong Jiang (co-PI, NJIT) and Yuan-Nan Young (PI, NJIT)}
\section{Background}
\label{sec:background}
\subsection{Hydrophobic Forces}
\label{sec:hydrophobicforce}

The hydrophobic force is a ubiquitous molecular interaction in biology \cite{Israelachvili1954}.
The word hydrophobic (water fearing) derives from the low solubility of oil (hydrocarbon solute) in water and vice versa. 
It causes hydrophobic moieties to aggregate and cluster,
is responsible for the adhesion between hydrophobic surfaces \cite{Ducker2016}, large contact angles on a 
dewetting surface \cite{Arenas2019,Sandre1999}, accumulation of particles along interfaces \cite{Lee2013,Lee2014}, 
formation of micelles and bilayers \cite{Israelachvili80}, and protein folding and membrane insertion \cite{Kabelka2018}.

Lipids are amphiphilic molecules whose  structure possesses
both hydrophobic and hydrophilic parts. 
The amphiphilic property is what allows lipids to form the membranes and 
compartments of living cells \cite{Israelachvili80}. 
More specifically, a lipid consists of 
an elongated hydrocarbon tail that is hydrophobic, attached to a polar head that is hydrophilic.
To shield the hydrophobic tails from water, lipids self-assemble into micelles and bilayers. 
A micelle is a spherical arrangement of lipids with tails terminating at the micelle center. 
A bilayer consists of two layers of lipids called monolayers, where the lipid tails point 
from the monolayer surface into the bilayer core. 

The mathematical modeling of a biological membrane is a challenging problem in applied mathematics. 
Bilayers are elastic and resist deformations like bending, twisting, and stretching.
Their elastic deformations are well described by the theory of liquid crystals \cite{ANDRIENKO2018520}.
Lipid bilayer membranes can also be fluidic, and the lateral translation of lipids (or any membrane bound
proteins) couples nonlocaly to the motion of the aqueous environment \cite{MerkelSackmannEvans1989,StoneAjdari1998_JFM,OppenheimerDiamant2009_BJ,OppenheimerDiamant2011_PRL}. 
%Finally, the membranes of cellular 
%compartments are constantly merging and pinching as part of intracellular trafficking. 
%Therefore monolayer surfaces undergo discontinuous deformations. 

There are two prevailing approaches in membrane modeling, and each has its advantages
and disadvantages. Molecular dynamics (MD) is to date the only tool capable of resolving granular biological details
The computational cost of MD, however, grows with the sixth power of the sample diameter and 
so simulations are severely limited to 
small system sizes and short time scales \cite{DiCarlo2019}.
The other approach, continuum mechanics, assumes smooth surfaces and can therefore model 
realistic systems over physical times. 
But continuum description of a biological membrane ignores the granularity of lipid molecules, and thus requires some assumptions when a
membrane ruptures or when two membranes fuse \cite{ChKo08}. 
%
%mechanics presumes, rather than predicts, the nucleation of discontinuities. 
%

Recently the PIs developed a mathematical model called the 
hydrophobic attraction potential (HAP) \cite{Fu2018_SIAM}.
This model addresses the major shortcomings of MD and continuum approaches,
and will hopefully offer an alternative modeling methodology that leads to new mathematical ideas 
and is both physically accurate and computationally practical. 

The hydrophobic force arises when polar solvent molecules come in contact with a non-polar substance, such as hydrocarbon or vapor.
In a polar solvent (like water), the dipole-dipole interaction between solvent molecules form a loosely structured hydrogen-bond network where
each solvent molecule shares bonds with neighboring molecules at any given time 
\cite{Israelachvili1954}. In the presence of  a non-polar solvent molecule loses the ability to form hydrogen bonds
in one direction. 
The decrease in the number of hydrogen bonds causes a reorientation, or structural
change, in the surrounding water that is energetically very unfavorable \cite{Bjorneholm2016}.


The substantial free energy for placing hydrophobic substances in contact with water 
is roughly proportional to the surface area of the contact region. 
As a result, hydrocarbon solutes have a large interfacial tension 
and try to minimize their surface area when in water. 
At the microscopic level, however, the hydrophobic force is a long-range, surface interaction. 
This means that two hydrophobic surfaces, separated by
water over some distance, experience an attractive force \cite{Lum1999,Meyer2006,Hammer2010}.
Measurements show that the hydrophobic force
decays exponentially with a decay length like 1 nm 
\cite{Israelachvili1984, Marcelja1977,Christenson2001,Lin2005}. It is also known that the interaction is not pairwise additive, meaning that the force
between any two hydrophobic objects is altered by the presence of a third  object. 

As a summary of its mathematical properties,  the hydrophobic force
is a non-additive, exponentially decaying surface force 
that possesses a separation of length scales. These properties suggest 
a boundary value problem formulation of the hydrophobic force.  
The non-additivity of the hydrophobic force has to do with the fact that there is no superposition
principle for including subdomains in boundary value problems. 
The exponential decay is a property of a second order elliptic partial differential equation (PDE). 
Finally, the separation of scales come from boundary layers, 
where the energy of the boundary layer in the zero-thickness limit corresponds to macroscopic interfacial tension.
Overlapping boundary layers correspond to microscopic hydrophobic attraction, 
and the boundary layer thickness corresponds to the decay length of attraction.

\begin{wrapfigure}[16]{l}{0.22\textwidth}
\centerline{\includegraphics[width=0.22\textwidth]{figures/BG_fig1.pdf}}
\caption{The exterior domain $\Omega$ is the solvent region surrounding a
number of rigid particles $P_1,$ $P_2,$ $P_3,$ \dots 
The surface $S$ is the solvent-particle interface, indicated by the unit
normal $\mathbf{n}.$ }\label{fig:domain}
\end{wrapfigure}
\subsection{Hydrophobic Attraction Potential (HAP)}
\label{sec:HAP}

Based on the physical origin of hydrophobicity,  we have devised a functional, called the hydrophobic attraction potential (HAP), to
model hydrophobic forces \cite{Fu19}. The motivation for HAP concept stemmed from the lead PI's 
earlier work on the problem of energy barriers in membrane fusion \cite{RyKlYaCo16,Chetal16}.
There, the lead PI and collaborators resolved the long-standing issue of accounting for the energy of a fissure surface
during topological transitions, by applying a squared gradient mathematical theory 
for hydrophobic attraction between planar surfaces \cite{Eriksson1989,Lum1999,Menshikov2017,Marcelja1977} to monolayer fissures.
Based on the PI Young's coarse-grained membrane modeling work \cite{Fu2017}, 
the investigators initiated a collaboration to create a gradient theory for arbitrary collections 
of hydrophobic and amphiphilic particles. This new method 
eliminates the costly calculation of water by treating the solvent implicitly, 
and avoids complicated re-meshing schemes of continuum approaches
by utilizing a particle-based representation.

To define HAP, consider a region $\Omega \subset \mathbb{R}^3$  
modeling the solvent surrounding a number of particles (Figure \ref{fig:domain}).
As stated earlier, the HAP must be an energy of the shape of the region
because hydrophobic attraction is non-additive. 
The boundary of the region is the water-particle interface and some parts of this
surface are hydrophobic while others are hydrophilic. 
Monte Carlo simulations 
show that water changes structure at hydrophobic interfaces \cite{Luzar1987, Jonsson2006,Varilly2011}.
Due to rapid fluctuation in the hydrogen bond network,
restructuring at the interface extends into bulk water.
This motivates the following definition for the HAP of the particles:
\begin{equation}
\label{HAP}
  \Phi = \gamma \iiint_{\Omega} \rho |\nabla u|^2 + \rho^{-1}u^2 \,dx \,dy \,dz, 
\end{equation}
where $u$ is the unique solution of screened Laplace equation boundary value problem 
\begin{equation}
  \label{SL}
  -\rho^2 \Delta u + u = 0 \in \Omega,\quad u = f  \mbox{ on } \partial \Omega, \quad u(x) \to 0 \mbox{ as } x \to \infty.
\end{equation}
The scalar function $u(x)$ is called the water activity. Its boundary values $f$ 
define the degree of hydrophobicity of the water-particle interface. The value 
1 represents a completely hydrophobic interface, where water has lost rotational freedom. 
Conversely, the boundary value 0 corresponds to a polar surface where water has the same rotational
freedom as in bulk water. The parameter $\rho$ is the decay length of attraction, 
around 1 nm, and the parameter $\gamma$ is the interfacial tension \cite{Israelachvili1954}.
The parameters $\rho$ and $\gamma$ contain information about the quality
of the solvent \cite{Discher2002}.

From a phenomenological perspective, (\ref{HAP}--\ref{SL}) 
is an appropriate model of hydrophobic attraction. Specifically,
using boundary layer analysis, it is possible 
to show that $\Phi$ converges to a surface energy in the zero-decay length limit \cite{Lee2018,Lin2015,Shibata2004}.
For non-zero decay lengths, solutions of \eqref{SL} yield an attractive interaction, e.g. between hydrophobic parallel disks
separated by water \cite{Eriksson1989}. Also, we have demonstrated with simulation that the forces 
derived from HAP theory are non-additive, and even deviate significantly from 
a pairwise potential \cite{Meyer2006,Fu19}. 
%The hydrophobic force has been implicated in the directed folding of proteins, 
%adhesion between biological membranes, but it is still unanswered as to whether
%the hydrophobic force of the form (\ref{HAP}--\ref{SL}) exists between small molecules. 


\section{Previous Results by the PIs}
\label{sec:results}

In our paper \cite{Fu19}, we developed the model (\ref{HAP}--\ref{SL})
to quantify the macroscopic assembly and mechanics of a lipid bilayer membrane in solvents. 
%We formulated the boundary value problem as a second-kind
%integral equation (SKIE), presented in the Section (). 
%The simulated fluid-particle systems exhibit a variety of multiscale behaviors over both time and length.
%Over short time scales, the numerical results showed self-assembly for model lipid particles. 
%For large system simulations, the particles formed realistic configurations like micelles and bilayers. 
%Our collections showed that these amphiphilic particle bilayers  possessed mechanical properties of a 
%lipid  bilayer  membrane  that  are  consistent  with other results in the literature.   
To define the particle dynamics, we considered the exterior domain $\Omega$
for a finite number of disjoint, rigid and closed particles $P_1$, $P_2$, \dots, $P_n$, each with Lipschitz boundary,
and boundary data $f$ in the Sobolev space $H^1(\Omega).$ 
For any fixed configuration, the functional \eqref{HAP} has a unique minimizer $u$
among all $u - f$ with vanishing trace. This minimizer is a solution to \eqref{SL}.
Conversely, from maximum principles and energy estimates,  any solution of 
\eqref{SL} is the minimizer of \eqref{HAP}, and this supplies $\Phi$ with a well-defined value.

The main analytical tool of our manuscript 
is a formula for the first variation of $\Phi$ \cite{Bandle2015,Schiffer1954,Grinfeld2010}. 
This first variation  is a symmetric, rank-two tensor called the hydrophobic stress;
\begin{equation}
  \label{stress}
\boldsymbol{\sigma}_{\text{hydro}} = \gamma \rho^{-1} u^2 I + 2\rho \gamma (\tfrac{1}{2}|\nabla u|^2I - \nabla u \nabla u^T).
\end{equation}
To obtain \eqref{stress}, we observe that the potential $\Phi$ is a function of the particle position and orientations.
This is because the particle configuration defines the shape of $\Omega$ and the boundary data $f.$ 
Taking the derivative of \eqref{HAP} with respect to particle configurations, and using the boundary value problem \eqref{SL}
in a critical way leads to the surface term \eqref{stress}.  
By integrating the hydrophobic stress over the surface of particle $P_i,$ we obtaining the hydrophobic force and torque on each particle 
\begin{equation}
  \label{forceandtorque}
  \mathbf{F}_i = \iint_{\partial P_i} \boldsymbol{\sigma}_{\text{hydro}} \cdot \mathbf{n} \,dS,\quad
  \mathbf{T}_i = \iint_{\partial P_i} \mathbf{x} \times (\boldsymbol{\sigma}_{\text{hydro}} \cdot \mathbf{n}) \,dS.
\end{equation}
We verified that this system is force and torque free.
To supply viscous dissipation, 
we incorporated the mobility problem in Stokes flow to obtain the rigid body motions; 
\begin{equation}
\label{eq:stokes}
\begin{aligned}
&-\mu \Delta {\bf u} + \nabla p = 0, \quad \text{in}\ \Omega, \qquad 
\nabla \cdot {\bf u} = 0,  \quad \text{in}\ \Omega,\\
&{\bf u}({\bf x}) \to 0 \quad \text{as}\ |{\bf x}|\to \infty,\qquad 
  \mathbf{u}(\mathbf{x})|_{\partial P_i} = \mathbf{v}_i +
\boldsymbol{\omega}_i\times(\mathbf{x} - \mathbf{a}_i),\\
&\int_{\partial P_i}\boldsymbol{\sigma}\cdot {\bf n} dS=-{\bf F}_i, \quad
\int_{\partial P_i}(\mathbf{x}-\mathbf{a}_i)\times (\boldsymbol{\sigma} \cdot \mathbf{n}) dS=-{\bf T}_i.
\end{aligned}
\end{equation}
Here $\mu$ is the fluid viscosity and the first two equations state
that the fluid motion is a
divergence-free Stokes flow; the third equation specifies the fluid
velocity at infinity;
the fourth equation enforces a rigid body motion constraint on each
particle, where
$\mathbf{v}_i$ and $\boldsymbol{\omega}_i$ are unknown translation and
angular velocities
of the $i$-th particle and $\mathbf{a}_i$ is the center of mass of $P_i$;
the last two equations state that the net forces and torques
on each particle are given by the quantities ${\bf F}_i$ and ${\bf T}_i$
from \eqref{forceandtorque}, and $\boldsymbol{\sigma}$ is the fluid shear stress. 

The time integration of particle configurations goes as follows: 
\textbf{(i)} solve the BVP \eqref{SL} for the screened Laplace equation, 
\textbf{(ii)} determine the rigid body forces and torques \eqref{forceandtorque}, 
\textbf{(iii)} solve the Stokes mobility problem \eqref{eq:stokes} for rigid body motion,
\textbf{(iv)} update the particle configuration. 
\S \ref{subsec:specific_aim_2}, addresses the numerical challenges in 
evaluating \eqref{stress} and how these challenges are overcome. 

%%(2YY: 
%We have intentionally ignored  latency in the variational 
%calculations. At issue is that any change in the particle position involves the relaxation time 
%of the hydrogen bond network. To remedy this, we could parametrize a path for the 
%particles configuration as a function of physical time. Then the rate of change
%of hydrophobic interaction includes the surface hydrophobic stress as calculated, but also a body term 
%for time change of activity, assuming we model hydrogen bond relaxation by diffusion. 
%The diffusion time, however, is extremely small since hydrogen bond lifetimes are on the order 
%of $10^{-11}$ s \cite{Israelachvili1954}. This setup is analogous  to that of a Kelvin-Voigt material,
%where viscoelastic deformations exponentially approach the purely elastic deformation.  
%%)


Our simulations used Janus particles to model
lipid amphiphiles. Janus particles are a popular tool in material science and physics
for creating functional materials \cite{Lee2014, Lee2013}. 
This topic is the basis of Specific Aim 3. Janus particles are typically
spherical with a biphasic material label on either hemisphere,  endowing the particle
with a directional order. We considered a two-dimensional system, and represented an 
elongated lipid by elliptical particles with hydrophobic and hydrophilic labels along the ellipse's long axis. 

Under the hydrophobic force, and with an excluded volume to prevent collisions, the 
Janus particles spontaneously merge and realign to form bilayers. This occurs only as a result
of energy minimization and does not require artificial inputs. To our knowledge, this is the first demonstration 
of bilayer self-assembly by a continuum hydrophobic interaction model \cite{Noguchi2001,Farago2003,Brannigan2006,Brooks2009,Wang2013}.  

It is worth emphasizing that the model uses only a few parameters; interfacial tension, decay length, repulsion strength and particle
shape. For example, an elastic modulus for stretching a vesicle derived from micropipette manipulation 
directly calibrate our interfacial tension parameter. 
MD simulators have also made measurements and lately there is better and better agreement with reality. 
But even the simplest coarse grained models based on pair potentials for lipids has many more parameters \cite{Varilly2011,Wang2013} . 

As a proof of concept, our work has already tested for elastic energies for bending, stretching and tilt of the bilayer assembly.
Extracting coefficients from the HAP simulations showed strikingly positive agreement with experimentally determined values  \cite{Fu19}.
Encouraged by these results and the hydrodynamic simulations of the bilayer assembly, the PIs propose to
further extend the HAP model and the numerical schemes (see \S\ref{sec:proposed-work})
to make direct comparison with experiments, the continuum liquid crystal results and MD simulations results, to establish that
the HAP model has the capability and adaptivity to model phenomena across length scales and time scales.
%
%
%The next few years provide the ideal window of opportunity for demonstrating the physical realism 
%of the HAP model. 
%
%
%
%The definitions and calculations for fully three-dimensional bilayer elastic energies are described in greater
%detail in Specific Aim 1. 
%


% 4.1 pN / nm = 4.1 pN nm / nm^2 = 4.1(1e-12)(1e-9) N m/(1e-18) m^2 = 4.1 (1e-3) J/m^2 = 4 mJ/m^2 
% pN/nm = mJ / m^2 and 1 kT / nm = 4 mJ / m^2 so stretching 40 gives 40  
% erg / cm^2 =  (1e-7) J/(1e-4) m^2 =   1e-3 J/m^2 =  mJ/m^2  
% 120 mJ / m^2 
% 10 erg / cm^2 = 10 pN nm / nm^2 = 2 kT / nm^2 
%

%\section{Proposed Research}
%\label{sec:proposed-work}
%The goal of the proposed research is to develop fast,
%high-order-accurate, parallel numerical algorithms for large-scale
%simulations of the collective hydrodynamics of janus particles in a solvent in both two- and three-dimensions.
%First we summarize some basic formulation and preliminary results \cite{Fu19} in \S~\ref{subsec:bie} and \S~\ref{subsec:3dbie}.
%We then describe outstanding numerical issues that we propose to address  in \S~\ref{subsec:proposed_research}.

%\section{Proposed Research}
%\label{sec:proposed-work}
%The goal of the proposed research is to develop fast,
%high-order-accurate, parallel numerical algorithms for large-scale
%simulations of the collective hydrodynamics of  amphiphilic particles in a viscous solvent.
%%
%Based on the integral formulation in \S~\ref{subsec:bie} and \S~\ref{subsec:3dbie}, we have demonstrated that 
%our potential theory approach can efficiently simulate self-assembly of 
%amphiphilic particles into two-dimensional micelles, bilayer membranes, and vesicles \cite{Fu19}.
%%
%While these results show great potentials in simulating the collective hydrodynamics of amphiphilic particles and
%reproducing mechanical properties of their bilayer assembly, 
%several outstanding issues need to be addressed for such approach to be efficiently applied to three-dimensional 
%collective hydrodynamics of amphiphilic particles.
%
%the two-dimensional hydrodynamics of amphiphilic particles 
%
%the two-dimensional results in \cite{Fu19} are in agreement with the 
%
%
%First we summarize some basic formulation and preliminary results \cite{Fu19} in \S~\ref{subsec:bie} and \S~\ref{subsec:3dbie}.
%We then describe outstanding numerical issues that we propose to address  in \S~\ref{subsec:proposed_research}.
% -----------------------------------------------------------------------------
%\subsection{Proposed research: High-order discretization of surface integrals in three dimensions}\label{subsec:proposed_research}
%% -----------------------------------------------------------------------------
%% {{{
%The practical application of integral equation methods requires the
%accurate evaluation of boundary integrals with singular, weakly
%singular or nearly singular kernels.  Historically, these issues have
%been handled either by low-order product integration rules (computed
%semi-analytically), by local modifications of a smooth
%rule~(e.g.~\cite{alpert,kapur,sidi}), by singularity
%subtraction/cancellation (e.g.~\cite{duffy,bruno1,bruno2,davis_1984,graglia_2008,hackbusch_sauter_1994, jarvenpaa_2003,khayat_2005,kress_boundary_1991,schwab_1992, ying_2006}), by kernel
%regularization and asymptotic analysis (e.g.~\cite{beale1,beale2,goodman_1990, haroldson_1998, lowengrub_1993,schwab_1992}), or by the
%construction of special purpose ``generalized Gaussian'' quadrature
%rules (e.g.~\cite{ggq1,ggq2,ggq3}).
%In the complex analytic case, additional methods
%have been developed by \citet{helsing_2008a} for off-surface
%evaluation. It should be noted that in the two-dimensional case,
%several of these alternatives provide extremely effective schemes,
%especially the kernel-splitting method developed by Johan Helsing
%\cite{helsing_integral_2009,helsing_tutorial_2012,helsing_2008a} since
%they all permit local adaptivity and high order accuracy.
%
%The high-order quadrature rules for the evaluation of surface integrals
%in three dimensions are much less developed than the line integrals in two
%dimensions. For example, there are no Gaussian quadratures for integraing
%polynomials on a flat triangle, even though efficient quadratures
%\cite{xiao2010cma,vioreanu2014} have been
%developed recently for such purpose. For weakly singular or singular integrals,
%\cite{bremer2012jcp,bremer2013jcp} constructed high-order quadratures
%for surface integrals on a general triangle, while \cite{gimbutas2013sisc}
%presented a fast algorithm for integrating $1/r$-type singular integrals
%for surfaces that are homeomorphic to a sphere. We would like to propose
%to study the so-called quadrature by expansion (QBX) scheme~\cite{klockner2013jcp,qbx2}
%for the evaluation
%of both singular and near-singular surface integrals encountered in the
%discretization of BIEs in three dimensions. Conceptually, the idea of the QBX
%to evaluate singular, hypersingular and near singular integrals
%on smooth surfaces is more or less straightforward. That is, the surface is discretized
%into smooth triangles and smooth high-order quadratures are applied to evaluate
%the expansion coefficients
%on all source triangles with the QBX expansion center placed at a point off the surface.
%One may then form a suitable expansion (for example, a Taylor expansion) around that
%center and evaluate this expansion back at the target point on the surface (or close
%to the surface in the near singular case). Compared to the competing aforementioned
%quadrature schemes, the QBX quadrature is attractive
%because it offers a clear path for being extended to: \textbf{(1)}
%handle any ambient and source dimensionality, \textbf{(2)} integrate
%any kernel, and thereby be usable for a very large range of PDEs and
%boundary conditions, \textbf{(3)} handle any singularity, including
%hypersingular operators, \textbf{(4)} be usable with any high-order surface
%discretization, \textbf{(5)} generate well-conditioned discrete
%operators to which iterative methods such as GMRES~\cite{gmres} can be
%applied in a black-box fashion, \textbf{(6)} be computationally
%efficient enough to be applied on the fly (without the need to store
%quadrature tables), \textbf{(7)} integrate well with fast algorithms
%such as the Fast Multipole Method. 
%
%In practice, there are still many issues that need to be resolved.
%For example, there are now many variants of QBX including global and local
%QBX~\cite{klockner2013jcp,rachh2017jcp}, the target-specific QBX~\cite{siegel2018jcp},
%kernel-independent QBX~\cite{abtin2018bit}, and
%quadrature by two expansions~\cite{ding2019arxiv}. The coupling of the QBX
%and the FMM may also lead to certain instability issues which may require
%some changes in the fast multipole method~\cite{wala2018jcp}. Similar
%to other quadrature methods, there have been an extensive study on the QBX
%methods in two dimensions, while its three dimension
%version~\cite{wala2019jcp,af2018sisc,siegel2018jcp,wala2019arxiv} has not been
%fully studied and the implementation is even more scarce. We plan to investigate
%the accuracy and the convergence order of the various QBX schemes mentioned above,
%its coupling with the FMM, parallel implementation issues for large-scale
%problems, and the application to our target problems.



