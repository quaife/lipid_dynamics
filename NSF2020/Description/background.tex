\noindent
{\bf Collaborative Research: Mathematical modeling and simulations of
self-assembling amphiphilic particles in a solvent} \\
%{\bf {Collaborative Research: Mathematical modeling and coarse-grained simulations of self-assembly of amphiphilic Janus particles in a solvent}}\\
{\em Rolf Ryham (lead PI, Fordham University), Bryan Quaife (PI,
FSU), and Y.-N. Young (PI, NJIT)}
\section{Background}
\label{sec:background}
The goal of this collaborative proposal is to use mathetical modeling and
numerical simulations to investigate the dynamic self-assembly of
amphiphilic particles interacting with each other via hydrophobic forces
in a solvent. Amphiphilic particles (such as lipid molecules) possess
both hydrophobic and hydrophilic structures. In a viscous solvent they
self-assemble into meso-/macroscopic structures (such as micelles and
bilayers of lipids) to shield their hydrophobic parts from contact with
the solvent molecules (water).
%
%\subsection{Hydrophobic Forces}
%\label{sec:hydrophobicforce}
%
Such self-assembly of amphiphiles via hydrophobic force is ubiquitous in biology and biophysics \cite{Israelachvili1954},
%The hydrophobic force is a ubiquitous molecular interaction in biology \cite{Israelachvili1954}, 
and has been a major source of nonspecific interactions between
nanoparticles in  soft matter
\cite{Sanchez-IglesiasEtAl2012_ACSNano,AltantzisEtAl2013_PSC,XieYangLuEtAl2020_COCIS}. 
%The proposed research aims to provide fundamental understanding of the self-assembly dynamics of amphiphilic particles so to design smart materials of desirable properties by
%tuning the geometry and properties of the amphiphilic particles. 

%The hydrophobic force arises when polar solvent molecules come in contact with a non-polar substance, such as hydrocarbon or vapor.
%In a polar solvent (like water), the dipole-dipole interaction between solvent molecules form a loosely structured hydrogen-bond network where
%each solvent molecule shares bonds with neighboring molecules at any given time 
%\cite{Israelachvili1954}. In the presence of  a non-polar solvent molecule loses the ability to form hydrogen bonds
%in one direction. 
%The decrease in the number of hydrogen bonds causes a reorientation, or structural
%change, in the surrounding water that is energetically very unfavorable \cite{Bjorneholm2016}.


The substantial free energy for placing hydrophobic substances in contact with water 
is roughly proportional to the surface area of the contact region \cite{Bjorneholm2016}.  
%As a result, hydrocarbon solutes have a large interfacial tension 
%and try to minimize their surface area when in water. 
At the microscopic level, the hydrophobic force is a long-range, surface
interaction.  This means that two hydrophobic surfaces, separated by
water over some distance, experience an attractive force
\cite{Lum1999,Meyer2006,Hammer2010}.  Measurements show that the
hydrophobic force decays exponentially with a decay length like 1 nm
\cite{Israelachvili1984, Marcelja1977,Christenson2001,Lin2005}. It is
also known that the interaction is not pairwise additive, meaning that
the force between any two hydrophobic objects is altered by the presence
of a third, hydrophobic or otherwise, object \cite{SilveraBatista1242477}. 

\begin{wrapfigure}[16]{r}{0.22\textwidth}
\centerline{\includegraphics[width=0.22\textwidth]{figures/BG_fig1.pdf}}
\caption{ \footnotesize
A collection $P_1,$ $P_2,$ $P_3,$ \dots of rigid particles. The exterior domain
$\Omega$ represents the solvent.}
  \label{fig:domain}
\end{wrapfigure}
Recently PIs RR and YNY developed a mathematical model called the
hydrophobic attraction potential (HAP) \cite{Fu2018_SIAM} that is based
on the physical origin of hydrophobicity.  This model addresses the
major shortcomings of molecular dynamics (MD) and continuum approaches.
Based on preliminary results (\S\ref{sec:preliminary_work}) the PIs
propose to extend this HAP model to offer an alternative modeling
methodology that leads to new mathematical ideas, and is both physically
accurate and computationally practical.
%
%
%The word hydrophobic (water fearing) derives from the low solubility of oil (hydrocarbon solute) in water and vice versa. 
%It causes hydrophobic moieties to aggregate and cluster,
%is responsible for the adhesion between hydrophobic surfaces \cite{Ducker2016}, large contact angles on a 
%dewetting surface \cite{Arenas2019,Sandre1999}, accumulation of particles along interfaces \cite{Lee2013,Lee2014}, 
%formation of micelles and bilayers \cite{Israelachvili80}, and protein folding and membrane insertion \cite{Kabelka2018}.
%
%Lipids are amphiphilic molecules whose  structure possesses
%both hydrophobic and hydrophilic parts. 
%The amphiphilic property is what allows lipids to form the membranes and 
%compartments of living cells \cite{Israelachvili80}. 
%More specifically, a lipid consists of 
%an elongated hydrocarbon tail that is hydrophobic, attached to a polar head that is hydrophilic.
%To shield the hydrophobic tails from water, lipids self-assemble into micelles and bilayers. 
%A micelle is a spherical arrangement of lipids with tails terminating at the micelle center. 
%A bilayer consists of two layers of lipids called monolayers, where the lipid tails point 
%from the monolayer surface into the bilayer core. 
%
%The mathematical modeling of a biological membrane is a challenging problem in applied mathematics. 
%Bilayers are elastic and resist deformations like bending, twisting, and stretching.
%Their elastic deformations are well described by the theory of liquid crystals \cite{ANDRIENKO2018520}.
%Lipid bilayer membranes can also be fluidic, and the lateral translation of lipids (or any membrane bound
%proteins) couples nonlocaly to the motion of the aqueous environment \cite{MerkelSackmannEvans1989,StoneAjdari1998_JFM,OppenheimerDiamant2009_BJ,OppenheimerDiamant2011_PRL}. 
%%Finally, the membranes of cellular 
%%compartments are constantly merging and pinching as part of intracellular trafficking. 
%%Therefore monolayer surfaces undergo discontinuous deformations. 
%
%There are two prevailing approaches in membrane modeling, and each has its advantages
%and disadvantages. Molecular dynamics (MD) is to date the only tool capable of resolving granular biological details
%The computational cost of MD, however, grows with the sixth power of the sample diameter and 
%so simulations are severely limited to 
%small system sizes and short time scales \cite{DiCarlo2019}.
%The other approach, continuum mechanics, assumes smooth surfaces and can therefore model 
%realistic systems over physical times. 
%But continuum description of a biological membrane ignores the granularity of lipid molecules, and thus requires some assumptions when a
%membrane ruptures or when two membranes fuse \cite{ChKo08}. 
%%
%%mechanics presumes, rather than predicts, the nucleation of discontinuities. 
%%
%
he proposed research aims to provide fundamental understanding of the self-assembly dynamics of amphiphilic particles. These results will facilitate optimal design of smart materials by
tuning the geometry and properties of the amphiphilic particles.


\subsection{Hydrophobic Attraction Potential (HAP)}
\label{sec:HAP}
%Based on the physical origin of hydrophobicity,  we have devised a
%functional, called the hydrophobic attraction potential (HAP), to model
%hydrophobic forces~\cite{Fu2018_SIAM}. 
The motivation for the HAP concept stemmed from PI RR's earlier work on
the problem of energy barriers in membrane
fusion~\cite{RyKlYaCo16,Chetal16}. By applying a squared gradient
mathematical theory for hydrophobic attraction between planar
surfaces~\cite{Eriksson1989,Lum1999,Menshikov2017,Marcelja1977}, PI RR
and collaborators resolved the long-standing issue of accounting for the
energy of a monolayer fissure surface during topological transitions.
Based on PI RR and YNY's coarse-grained membrane modeling
work~\cite{Fu2017}, the investigators initiated a collaboration to
create a gradient theory for arbitrary collections of hydrophobic and
amphiphilic particles. This new method eliminates the costly calculation
of water by treating the solvent implicitly, and avoids complicated
re-meshing schemes of continuum approaches by utilizing a particle-based
representation.

To define HAP, consider a region $\Omega \subset \mathbb{R}^3$  modeling
the solvent surrounding a number of particles (Figure \ref{fig:domain}).
The HAP must be an energy of the shape of the region because hydrophobic
attraction is non-additive. The boundary of the region is the
water-particle interface and some parts of this surface are hydrophobic
while others are hydrophilic. Monte Carlo simulations show that water
changes structure at hydrophobic interfaces~\cite{Luzar1987,
Jonsson2006, Varilly2011}. Due to rapid fluctuation in the hydrogen bond
network, restructuring at the interface extends into bulk water. This
motivates the following definition for the HAP of the particles:
\begin{align}
\label{HAP}
  \Phi = \gamma \iiint_{\Omega} \rho |\nabla u|^2 + \rho^{-1}u^2 \,dx \,dy \,dz, 
\end{align}
where $u$ is the unique solution of the screened Laplace equation boundary value problem 
\begin{equation}
  \label{SL}
  -\rho^2 \Delta u + u = 0, \mbox{ } \mathbf{x} \in \Omega, \qquad
  u = f,  \mbox{ } \mathbf{x} \in \partial \Omega, \qquad 
  u(\mathbf{x}) \to 0 \mbox{ as } |\mathbf{x}| \to \infty.
\end{equation}
The scalar function $u(\mathbf{x})$ is called the water activity. Its
boundary values $f$ define the degree of hydrophobicity of the
water-particle interface. If $f=1$, the interface is hydrophobic and
water has lost rotational freedom. Conversely, if $f=0$, the interface
has the same rotational freedom as in bulk water. The parameter $\rho$
is the decay length of attraction, around 1 nm, and the parameter
$\gamma$ is the interfacial tension~\cite{Israelachvili1954}. Both
parameters contain information about the quality of the
solvent~\cite{Discher2002}.

From a phenomenological perspective, equations~\eqref{HAP}
and~\eqref{SL} form an appropriate model of hydrophobic attraction.
Specifically, using boundary layer analysis, it is possible to show that
$\Phi$ converges to a surface energy in the zero-decay length
limit~\cite{Lee2018, Lin2015, Shibata2004}. For non-zero decay lengths,
solutions of~\eqref{SL} yield an attractive interaction between
hydrophobic parallel disks separated by water~\cite{Eriksson1989}. Also,
we have demonstrated with simulation that the forces derived from HAP
theory are non-additive and significantly deviate from a pairwise
potential~\cite{Meyer2006, Fu2018_SIAM}. 
%The hydrophobic force has been implicated in the directed folding of proteins, 
%adhesion between biological membranes, but it is still unanswered as to whether
%the hydrophobic force of the form (\ref{HAP}--\ref{SL}) exists between small molecules. 

%As a summary of its mathematical properties,  the hydrophobic force
%is a non-additive, exponentially decaying surface force 
%that possesses a separation of length scales. These properties suggest 
%a boundary value problem formulation of the hydrophobic force.  
%The non-additivity of the hydrophobic force has to do with the fact that there is no superposition
%principle for including subdomains in boundary value problems. 
%The exponential decay is a property of a second order elliptic partial differential equation (PDE). 
%Finally, the separation of scales come from boundary layers, 
%where the energy of the boundary layer in the zero-thickness limit corresponds to macroscopic interfacial tension.
%Overlapping boundary layers correspond to microscopic hydrophobic attraction, 
%and the boundary layer thickness corresponds to the decay length of attraction.
%
%
%\section{Previous Results by the PIs}
%\label{sec:results}
%
%%In our paper \cite{Fu2018_SIAM}, 
%PIs RR and YNY developed the HAP model (\ref{HAP}--\ref{SL})
%to quantify the macroscopic assembly and mechanics of a lipid bilayer membrane in solvents \cite{Fu2018_SIAM}.
%%We formulated the boundary value problem as a second-kind
%%integral equation (SKIE), presented in the Section (). 
%%The simulated fluid-particle systems exhibit a variety of multiscale behaviors over both time and length.
%%Over short time scales, the numerical results showed self-assembly for model lipid particles. 
%%For large system simulations, the particles formed realistic configurations like micelles and bilayers. 
%%Our collections showed that these amphiphilic particle bilayers  possessed mechanical properties of a 
%%lipid  bilayer  membrane  that  are  consistent  with other results in the literature.   
To define the particle dynamics, consider the exterior domain $\Omega$
for a finite number of disjoint, rigid and closed particles $P_1$,
$P_2$, \ldots, $P_n$, each with Lipschitz boundary, and boundary data
$f$ in the Sobolev space $H^1(\Omega)$. For any fixed configuration, the
minimizer of the functional~\eqref{HAP} among all $u - f$ with vanishing
trace is the solution of~\eqref{SL}. Conversely, from maximum principles
and energy estimates, the solution of~\eqref{SL} is the minimizer
of~\eqref{HAP}, and this supplies $\Phi$ with a well-defined value.
Taking the derivative of~\eqref{HAP} with respect to particle
configurations, and using the boundary value problem~\eqref{SL}, the
first variation of the hydrophobic potential $\Phi$ yields a symmetric,
rank-two tensor~\cite{Bandle2015, Schiffer1954, Grinfeld2010} called the
hydrophobic stress:
\begin{align}
  \label{stress}
\boldsymbol{\sigma}_{\text{hydro}} = \gamma \rho^{-1} u^2 I + 2\rho \gamma (\tfrac{1}{2}|\nabla u|^2I - \nabla u \nabla u^T).
\end{align}
%
%To obtain \eqref{stress}, we observe that the potential $\Phi$ is a function of the particle position and orientations.
%This is because the particle configuration defines the shape of $\Omega$ and the boundary data $f.$ 
%Taking the derivative of \eqref{HAP} with respect to particle configurations, and using the boundary value problem \eqref{SL}
%in a critical way leads to the surface term \eqref{stress}.  
%
Integrating the hydrophobic stress over the surface of particle $P_i$
reveals the hydrophobic force and torque on each particle 
\begin{align}
  \label{forceandtorque}
  \mathbf{F}_i = \iint_{\partial P_i} \boldsymbol{\sigma}_{\text{hydro}}
  \cdot \boldsymbol{\nu} \,dS,\quad
  \mathbf{T}_i = \iint_{\partial P_i} \mathbf{x} \times
  (\boldsymbol{\sigma}_{\text{hydro}} \cdot \boldsymbol{\nu}) \,dS.
\end{align}
Our previous work verifies that this system is force- and torque-free
and \S\ref{subsec:specific_aim_2} addresses the numerical challenges in
evaluating~\eqref{stress} and how these challenges are overcome. To
supply viscous dissipation, we incorporate the mobility problem in
Stokes flow to obtain the rigid body motions:
\begin{equation}
\label{eq:stokes}
\begin{aligned}
  &-\mu \Delta \mathbf{u} + \nabla p = 0, \quad \mathbf{x} \in \Omega, \qquad 
  \nabla \cdot \mathbf{u} = 0,  \quad \mathbf{x} \in \Omega,\\
  &{\bf u}(\mathbf{x}) \to 0 \quad \text{as}\ |\mathbf{x}|\to \infty,\qquad 
  \mathbf{u}(\mathbf{x})|_{\partial P_i} = \mathbf{v}_i +
\boldsymbol{\omega}_i\times(\mathbf{x} - \mathbf{a}_i),\\
&\int_{\partial P_i}\boldsymbol{\sigma}\cdot {\bf n} dS=-{\bf F}_i, \quad
\int_{\partial P_i}(\mathbf{x}-\mathbf{a}_i)\times (\boldsymbol{\sigma} \cdot \mathbf{n}) dS=-{\bf T}_i.
\end{aligned}
\end{equation}
Here $\mu$ is the fluid viscosity and the first two equations state that
the fluid motion is a divergence-free Stokes flow; the third equation
specifies that the fluid velocity vanishes at infinity; the fourth
equation enforces a rigid body motion on each particle, where
$\mathbf{v}_i$ and $\boldsymbol{\omega}_i$ are unknown translation and
angular velocities of the $i$-th particle and $\mathbf{a}_i$ is the
center of mass of $P_i$; the last two equations state that the net
forces and torques on each particle are $\mathbf{F}_i$ and
$\mathbf{T}_i$ from~\eqref{forceandtorque}, respectively, and
$\boldsymbol{\sigma}$ is the fluid shear stress. The time integration of
particle configurations goes as follows: \textbf{(i)} solve the
BVP~\eqref{SL} for the screened Laplace equation, \textbf{(ii)}
determine the rigid body forces and torques~\eqref{forceandtorque},
\textbf{(iii)} solve the Stokes mobility problem~\eqref{eq:stokes} for
rigid body motion, \textbf{(iv)} update the particle configuration.
Numerical challenges and how these challenges are overcome are described
in \S\ref{subsec:specific_aim_2}.

%%(2YY: 
%We have intentionally ignored  latency in the variational 
%calculations. At issue is that any change in the particle position involves the relaxation time 
%of the hydrogen bond network. To remedy this, we could parametrize a path for the 
%particles configuration as a function of physical time. Then the rate of change
%of hydrophobic interaction includes the surface hydrophobic stress as calculated, but also a body term 
%for time change of activity, assuming we model hydrogen bond relaxation by diffusion. 
%The diffusion time, however, is extremely small since hydrogen bond lifetimes are on the order 
%of $10^{-11}$ s \cite{Israelachvili1954}. This setup is analogous  to that of a Kelvin-Voigt material,
%where viscoelastic deformations exponentially approach the purely elastic deformation.  
%%)

Our simulations use Janus particles to model lipid amphiphiles which are
popular in material science and physics for creating functional
materials~\cite{Lee2014, Lee2013}. Janus particles are typically
spherical with a biphasic material label on either hemisphere, endowing
the particle with a directional order. We considered a two-dimensional
system, and represented an elongated lipid by elliptical particles with
hydrophobic and hydrophilic labels along the ellipse's long axis. Under
the hydrophobic force, and with an excluded volume to prevent
collisions, the Janus particles spontaneously merge and realign to form
bilayers. This occurs only as a result of energy minimization and does
not require artificial inputs. To our knowledge, this is the first
demonstration of bilayer self-assembly by a continuum hydrophobic
interaction model~\cite{Noguchi2001, Farago2003, Brannigan2006,
Brooks2009, Wang2013}.

It is worth emphasizing that the HAP model uses only a few parameters;
interfacial tension, decay length, repulsion strength, and particle
shape. For example, an elastic modulus for stretching a vesicle from
micropipette manipulation directly calibrate our interfacial tension
parameter. This is in direct contrast with pair-potential based
approaches in MD simulations and coarse-grained models where many more
parameters are required~\cite{Varilly2011, Wang2013}.
%MD simulators have also made measurements and lately there is better and better agreement with reality. 
%But even the simplest coarse grained models based on pair potentials for lipids has many more parameters \cite{Varilly2011,Wang2013} . 

As a proof of concept, our work has already tested for elastic energies
for bending, stretching, and tilt of the bilayer assembly. Extracting
coefficients from the HAP simulations shows strikingly positive
agreement with experimentally determined values~\cite{Fu2018_SIAM}.
Encouraged by these results and the hydrodynamic simulations of the
bilayer assembly, the PIs propose to extend the HAP model and the
numerical schemes (see \S\ref{sec:proposed-work}) to make direct
comparison with experiments, the continuum liquid crystal results and MD
simulations results, to establish that the HAP model has the capability
and adaptivity to model phenomena across length scales and time scales.

Over the past decades, researchers have used a number of mathematical
tools to simulate vesicles in a shear flow, including lattice
Boltzmann~\cite{KaouiHartingMisbah2011_PRE}, coarse-grained Brownian
dynamics~\cite{NoguchiTakasu2002_BJ}, phase
field~\cite{DuLiuWang2004_JCP,BibenKassnerMisbah2005_PRE}, level
set~\cite{DoyeuxGuyotChabannesEtAl2013_JCAM}, boundary
integral~\cite{Shravan09,Rahimian15} and immersed boundary
approaches~\cite{KimLai2010_JCP,KimLai2012_PRE,HuLaiSeolEtAl2016_JCP}.
Most of these approaches assume a mathematical surface, whether
implicitly or explicitly, and define an elastic bending energy of the
surface. Researchers have developed a number of numerical methods for
calculating energy minimizing steady equilibrium shapes of lipid bilayer
membranes, vesicles and red blood cells. These approaches range from the
finite element method~\cite{Bartels,Peng13,RyKlYaCo16,Sinha15}, phase
field method~\cite{Du05,QiangDu08,Lowengrub13} to immersed boundary
method~\cite{Hu,Hu13, KimLai2010_JCP}. PI RR and collaborators led in
part the development of phase field functionals of membrane elastic
energy and approaches to coupling membrane elasticity to
fluids~\cite{0951-7715-18-3-016,Du05,DuEuler,QiangDu09}.


%The vesicle obeys fluid
%transport and in turn the fluid balances shear stress with the vesicle's bending force. 




Our HAP approach differs from prior methods in a number of respects.
First, we do not assume a surface. Rather, we more fundamentally
consider a collection of amphiphilic particles.  The collection of
amphiphiles minimize hydrophobic interactions by sequestering
hydrophobic tails in the form of a bilayer, and the particles' excess
free energy gives rise to an elastic bilayer energy. The second
difference lies in the fluid-interface coupling. Here, the associated
mobility problem~\eqref{eq:stokes} is more complicated than dealing with
a stress boundary condition or diffusive surface force, because the
fluid velocity must be that of a rigid body motion at the particle
surfaces. The HAP model also addresses the existence of multiple phases.
We can vary lipid length, spontaneous curvature, and bending rigidity by
introducing different particle shapes and hydrophobic boundary
conditions (Figure~\ref{fig:demixing}). In contrast, continuum theory
deals with multiple phases through additional surface densities that
must then satisfy specialized transport equations~\cite{Lowengrub07,
MikuckiZhou17}. 
%
\begin{wrapfigure}[14]{l}{0.35\textwidth}
\centerline{\includegraphics[width=0.35\textwidth]{figures/PW_fig2.pdf}}
  \vspace{-8pt}
  \caption{\label{fig:demixing} \footnotesize An initial assembly of
  small and large particles spontaneously segregates into two smaller
  bodies.}
\end{wrapfigure}
The greatest strength of using HAP to model lipid bilayer membrane is
the ability to form discontinuities (interfacial singularity) from first
physical principles without any artificial manipulation to rearrangement
the interface. For example, using the HAP model to simulate a vesicle
under a shear flow that is strong enough to cause the vesicle membrane
to rupture, we expect the HAP model to capture the reorganization of
lipid molecules on the scales of membrane thickness ($\sim 5$ nm), which
is a nearly impossible task using phase-field or immerse boundary method
without extreme refinement around the membrane and artificial treatment
of reconnection during the topological change of an interface.

%Some disadvantages are that the hydrophobic interaction does not constrain vesicle volume. 
%Instead, changes in volume are rate limited by inter-particle spacing,
%which may be undesirable in case of a mathematically strict volume constraint.
%% set vesicle volume, as is done in the study of liposomes
%%using auxiliary constraint equations for instance. 
%%Also, the boundary integral method formulation
%%relies on linearity of the Stokes equations. There has been some progress in
%%boundary integral methods for the non-linear Navier-Stokes equations [ref].
%%We point out that some non-Newtonian effects in polymers are a consequence of hydrophobic
%%and steric molecular interactions like the ones presented in this proposal. 





%
%
%The next few years provide the ideal window of opportunity for demonstrating the physical realism 
%of the HAP model. 
%
%
%
%The definitions and calculations for fully three-dimensional bilayer elastic energies are described in greater
%detail in Specific Aim 1. 
%


% 4.1 pN / nm = 4.1 pN nm / nm^2 = 4.1(1e-12)(1e-9) N m/(1e-18) m^2 = 4.1 (1e-3) J/m^2 = 4 mJ/m^2 
% pN/nm = mJ / m^2 and 1 kT / nm = 4 mJ / m^2 so stretching 40 gives 40  
% erg / cm^2 =  (1e-7) J/(1e-4) m^2 =   1e-3 J/m^2 =  mJ/m^2  
% 120 mJ / m^2 
% 10 erg / cm^2 = 10 pN nm / nm^2 = 2 kT / nm^2 
%

%\section{Proposed Research}
%\label{sec:proposed-work}
%The goal of the proposed research is to develop fast,
%high-order-accurate, parallel numerical algorithms for large-scale
%simulations of the collective hydrodynamics of janus particles in a solvent in both two- and three-dimensions.
%First we summarize some basic formulation and preliminary results
%\cite{Fu2018_SIAM} in \S~\ref{subsec:bie} and \S~\ref{subsec:3dbie}.
%We then describe outstanding numerical issues that we propose to address  in \S~\ref{subsec:proposed_research}.

%\section{Proposed Research}
%\label{sec:proposed-work}
%The goal of the proposed research is to develop fast,
%high-order-accurate, parallel numerical algorithms for large-scale
%simulations of the collective hydrodynamics of  amphiphilic particles in a viscous solvent.
%%
%Based on the integral formulation in \S~\ref{subsec:bie} and \S~\ref{subsec:3dbie}, we have demonstrated that 
%our potential theory approach can efficiently simulate self-assembly of 
%amphiphilic particles into two-dimensional micelles, bilayer membranes,
%and vesicles \cite{Fu2018_SIAM}.
%%
%While these results show great potentials in simulating the collective hydrodynamics of amphiphilic particles and
%reproducing mechanical properties of their bilayer assembly, 
%several outstanding issues need to be addressed for such approach to be efficiently applied to three-dimensional 
%collective hydrodynamics of amphiphilic particles.
%
%the two-dimensional hydrodynamics of amphiphilic particles 
%
%the two-dimensional results in \cite{Fu2018_SIAM} are in agreement with the 
%
%
%First we summarize some basic formulation and preliminary results
%\cite{Fu2018_SIAM} in \S~\ref{subsec:bie} and \S~\ref{subsec:3dbie}.
%We then describe outstanding numerical issues that we propose to address  in \S~\ref{subsec:proposed_research}.
% -----------------------------------------------------------------------------
%\subsection{Proposed research: High-order discretization of surface integrals in three dimensions}\label{subsec:proposed_research}
%% -----------------------------------------------------------------------------
%% {{{
%The practical application of integral equation methods requires the
%accurate evaluation of boundary integrals with singular, weakly
%singular or nearly singular kernels.  Historically, these issues have
%been handled either by low-order product integration rules (computed
%semi-analytically), by local modifications of a smooth
%rule~(e.g.~\cite{alpert,kapur,sidi}), by singularity
%subtraction/cancellation (e.g.~\cite{duffy,bruno1,bruno2,davis_1984,graglia_2008,hackbusch_sauter_1994, jarvenpaa_2003,khayat_2005,kress_boundary_1991,schwab_1992, ying_2006}), by kernel
%regularization and asymptotic analysis (e.g.~\cite{beale1,beale2,goodman_1990, haroldson_1998, lowengrub_1993,schwab_1992}), or by the
%construction of special purpose ``generalized Gaussian'' quadrature
%rules (e.g.~\cite{ggq1,ggq2,ggq3}).
%In the complex analytic case, additional methods
%have been developed by \citet{helsing_2008a} for off-surface
%evaluation. It should be noted that in the two-dimensional case,
%several of these alternatives provide extremely effective schemes,
%especially the kernel-splitting method developed by Johan Helsing
%\cite{helsing_integral_2009,helsing_tutorial_2012,helsing_2008a} since
%they all permit local adaptivity and high order accuracy.
%
%The high-order quadrature rules for the evaluation of surface integrals
%in three dimensions are much less developed than the line integrals in two
%dimensions. For example, there are no Gaussian quadratures for integraing
%polynomials on a flat triangle, even though efficient quadratures
%\cite{xiao2010cma,vioreanu2014} have been
%developed recently for such purpose. For weakly singular or singular integrals,
%\cite{bremer2012jcp,bremer2013jcp} constructed high-order quadratures
%for surface integrals on a general triangle, while \cite{gimbutas2013sisc}
%presented a fast algorithm for integrating $1/r$-type singular integrals
%for surfaces that are homeomorphic to a sphere. We would like to propose
%to study the so-called quadrature by expansion (QBX) scheme~\cite{klockner2013jcp,qbx2}
%for the evaluation
%of both singular and near-singular surface integrals encountered in the
%discretization of BIEs in three dimensions. Conceptually, the idea of the QBX
%to evaluate singular, hypersingular and near singular integrals
%on smooth surfaces is more or less straightforward. That is, the surface is discretized
%into smooth triangles and smooth high-order quadratures are applied to evaluate
%the expansion coefficients
%on all source triangles with the QBX expansion center placed at a point off the surface.
%One may then form a suitable expansion (for example, a Taylor expansion) around that
%center and evaluate this expansion back at the target point on the surface (or close
%to the surface in the near singular case). Compared to the competing aforementioned
%quadrature schemes, the QBX quadrature is attractive
%because it offers a clear path for being extended to: \textbf{(1)}
%handle any ambient and source dimensionality, \textbf{(2)} integrate
%any kernel, and thereby be usable for a very large range of PDEs and
%boundary conditions, \textbf{(3)} handle any singularity, including
%hypersingular operators, \textbf{(4)} be usable with any high-order surface
%discretization, \textbf{(5)} generate well-conditioned discrete
%operators to which iterative methods such as GMRES~\cite{gmres} can be
%applied in a black-box fashion, \textbf{(6)} be computationally
%efficient enough to be applied on the fly (without the need to store
%quadrature tables), \textbf{(7)} integrate well with fast algorithms
%such as the Fast Multipole Method. 
%
%In practice, there are still many issues that need to be resolved.
%For example, there are now many variants of QBX including global and local
%QBX~\cite{klockner2013jcp,rachh2017jcp}, the target-specific QBX~\cite{siegel2018jcp},
%kernel-independent QBX~\cite{abtin2018bit}, and
%quadrature by two expansions~\cite{ding2019arxiv}. The coupling of the QBX
%and the FMM may also lead to certain instability issues which may require
%some changes in the fast multipole method~\cite{wala2018jcp}. Similar
%to other quadrature methods, there have been an extensive study on the QBX
%methods in two dimensions, while its three dimension
%version~\cite{wala2019jcp,af2018sisc,siegel2018jcp,wala2019arxiv} has not been
%fully studied and the implementation is even more scarce. We plan to investigate
%the accuracy and the convergence order of the various QBX schemes mentioned above,
%its coupling with the FMM, parallel implementation issues for large-scale
%problems, and the application to our target problems.



