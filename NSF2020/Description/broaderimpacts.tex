\section{Broader Impacts of the Proposed Work}
\label{sec:BroaderImpacts}


The proposed mathematical analysis and modeling will provide a transformative
understanding of the collective dynamics of amphiphilic particles such as 
(1) their self-assembly into micelles and bilayers, and 
(2) the interaction between these building blocks.
Amphiphilic Janus particles have gained increasing popularity for fabrication of smart materials.
The proposed hybrid continuum model, mathematical analysis and numerical algorithms will have transformative impact on
precision design for specific mechanical properties of materials made of amphiphilic nanoparticles.
An important component of this proposal is the interdisciplinary education and training of
both undergraduate and graduate students.  The combination of mathematical modeling, analysis and
scientific computing  in this project provides a compelling
example of the importance of mathematics in biophysics and engineering applications.
The concepts and
methods described here go beyond the context of amphiphilic particles and lipid molecules. They extend
to other problems featuring microscopic phase separation that leads to  formation of mesoscopic domains.
This situation arises, for example, in biological development in systems biology.
The methods developed here have the potential to impact those and other related areas in biomedicine and biotechnology.

\subsection{Educational Impacts}
\label{subsec:Educational_plans}
The proposed research will have immediate impact on 
undergraduate and graduate education. At the undergraduate level,
the project will involve and support undergraduate researcher from Fordham University.
The supported summer researchers will gain valuable first-hand 
experience in inherently interdisciplinary, mathematical fluid modeling and computation.
We will incorporate research into teaching, especially in differential equation  
and programming courses. 

Over the past eight years, PI Ryham has included many undergraduates 
in the execution of research initiatives, coauthoring 
publications. He has a  track record for supporting underrepresented groups,
being called on by the Collegiate Science and Technology Entry Program (CSTEP) to make
opportunities for promising students, for example. He has mentored
two Clare Boothe Luce Scholars, one a U.S. Marine Corp veteran,
and a high school student in the NYU GSTEM program. 

The PIs will capitalize on the synergy between NJIT's doctoral program in Mathematical Sciences 
and Fordham's undergraduate focus in the Department of Mathematics. 
Promising, young undergraduate scientists from the Bronx community 
while find a natural pipeline into graduate studies through working directly 
or indirectly on this project. 
On the graduate-level education, PI Young and co-PI Jiang expect to train and support
a PhD student, Yuexin Liu (who is in her third-year at NJIT) for two more years. Yuexin Liu has past experiences
in the undergraduate training program at NJIT, and will train an undergraduate intern. 
The PIs will foster vertical integration between the senior personnel, their collaborators, postdocs, 
doctoral and Bachelor students, enhancing the
learning environment.
%


%The PIs expect
%to involve undergraduate students in this project through
%their continuing and active involvement in undergraduate advising and research mentoring:
%YNY has mentored undergraduate students as a co-Investigator in CSUMS: Research and Education
%in Computational Mathematics for undergraduates in the Mathematical 
%Sciences at NJIT (funded by NSF) and the lecturer for Capstone Applied
%Mathematics Lab at NJIT (also funded by NSF).
%YNY has track records in involving undergraduate students in research
%that emphasizes both numerical computations and desktop experiments.
%
%such as MATH 340: Advanced Numerical Methods.
%Simple MATLAB codes have been used as teaching tools to show students how to use MATLAB
%to simulate the slender-body equations for an elastic filament in Stokes flow.
%
%In the past year YNY has been mentoring Ufuomaefe Ogbe, a high school student
%from Newark. Ufuomaefe is an African American who 
%worked with PI on a simplified model for vesicles in flow. Based
%on what he learned about modeling and programming, 
%he has applied to the Applied Mathematics/Computer Science programs at both NJIT
%and Rutgers/Newark.

\section{Intellectual Merit}
A central theme of the project is the mathematical development and 
analysis of the physical model for interaction between many amphiphilic 
particles. The model formulates the interaction potential through a
screened Laplace equation boundary value problem possessing 
the physical properties like non-additivity and decay of realistic hydrophobic attraction.
Colloidal systems collectively self-assemble into bilayer
morphologies, and we analyze the elastic properties of these amphiphilic particle ensembles.
This allows us to interpret the Helfrich free energy in terms of hydrophobic interactions and 
specific molecular characteristics. We extend the capabilities of the
boundary integral equation quadrature by expansion method to perform
large particle number simulations in three dimensions with collisions. 
Results from these computations will be used to compare collective  amphiphilic 
against experiment,  and study the optimal design of three-dimensional functional materials. 


\section{Relevant Results from Prior NSF Support}
\noindent {\bf Rolf Ryham}: no prior NSF support.

\noindent
{\bf Yuan-Nan Young}: {\it NSF-DMS-1222550, Mathematical and experimental study of lipid bilayer 
shape and dynamics mediated by surfactants and proteins}, \$212,603, 9/15/2012 - 08/31/2016 (with no-cost extension), PI. 
{\it Intellectual merit:} The focus of this grant is  modeling the interaction between a pure lipid bilayer membrane  (LBM) with surfactant, cholesterol and protein.
%Hence, there is no overlap between these grants and the current proposal.

%YNY and collaborators have studied (1) asymmetry of lipid bilayer due to its interaction
%with proteins and surfactants, (2) the electrohydrodynamics of a LBM under an
%electric field, and (3) the coupling between LBM dynamics and a transmembrane protein,
%such as a mechanosensitive channel. 

\noindent
{\it Broader impacts:} 
One PhD student (Szu-Pei Fu) was
funded to work with YNY, and work has resulted in seven papers
\cite{Nganguia2013_PoF,Nganguia2013_PRE,Young2014_JFM,Young2015_PoF,Nganguia2015_CiCP,Pak2015_PNAS,Fu2015_PRE}.
YNY has been actively involved with promotion of underrepresented students at NJIT. 
The other PhD student (Herve Nganguia) is African. YNY has taught a broad
spectrum of courses in fluid mechanics and applied math modeling.

\noindent
{\bf Shidong Jiang}: {\it NSF DMS-1418918,
  Efficient High-Order Parallel Algorithms for Large-Scale Photonics Simulation},
\$150,000, 08/15/2014 - 06/30/2017, PI.
{\it Intellectual merit:} The goal of the proposed research is to develop
\emph{robust, high-order numerical methods} and \emph{efficient parallel algorithms} to
achieve greater accuracy and larger scale for photonics simulation.

%Jiang, Kloeckner, and their collaborators have constructed
%a stable second kind integral equation formulation for the mode calculation of optical
%waveguides of arbitrary shape; they have quantified the effect of cladding thickness on
%modal confinement loss in photonic waveguides; they have developed high-quality software
%package for solving integral equations in both two and three dimensions.

\noindent
{\it Broader impacts:} 
During the funded period, the PI has published or submitted twelve papers to refereed international journals~\cite{jiang2014sisc,bao2015jcp,jiang2015acom,greengard2015jcp,fu2015pre,gan2016sisc,jiang2016jcp,jiang2016oe,ben2016tpm,jiang2017siammms,jiang2017cicp,lai2018acha},
and presented the research results at many institutions and international
conferences/workshops to disseminate research results. On the education side,
the PI has supervised two students - Shaobo Wang~\cite{jiang2015acom,wang2019jsc}
and Szu-Pei Fu~\cite{fu2015pre,Fu19} (co-advised with Yuan-Nan Young),
with the partial support of the grant.
%Finally, the software package developed by both PIs of this collaborative grant
%has many applications in scientific computing, especially those related to
%the integral equation methods.

\section{Project Management, Collaboration Plan, and Schedules of
  Research Tasks}
\setlength{\parindent}{0pt}
% N.B. wrapfigure is not compatible with \noindent.

\begin{wrapfigure}{r}{0.56\textwidth}
\definecolor{barblue}{RGB}{51,102,254}
\renewcommand\sfdefault{phv}
\renewcommand\mddefault{mc}
\renewcommand\bfdefault{bc}
\sffamily
\begin{ganttchart}[
    canvas/.append style={fill=none, draw=black!5, line width=.75pt},
        x unit =4.5mm,
        y unit chart =\baselineskip,
    hgrid style/.style={draw=black!5, line width=.75pt},
    vgrid={*1{draw=black!5, line width=.75pt}},
    title/.style={draw=none, fill=none},
    title label font=\bfseries\footnotesize,
    title label node/.append style={below=3pt},
    include title in canvas=false,
    bar label font=\mdseries\small\color{black!70},
    %bar label node/.append style={left=2cm},
    bar/.append style={draw=none, fill=barblue},
    bar incomplete/.append style={fill=barblue},
    bar progress label font=\mdseries\footnotesize\color{black!70},
    milestone label font=\mdseries\small\color{black!70},
        milestone left shift =0.9,
        milestone right shift =0.1,
        % Don't draw group bars
        group height =0,
        group peaks height =0,
        group label font =\bfseries\small,
]{1}{12}
\gantttitle[
    title label node/.append style={below left=3pt and -6pt}
]{QUARTERS:\quad1}{1}
\gantttitlelist{2,...,12}{1} \\
\ganttgroup{Tasks\hfill}{1}{12}\\
\ganttbar{Specific Aim 1}{1}{6}\\
\ganttbar{Specific Aim 2}{1}{9}\\
\ganttbar{Specific Aim 3}{5}{12}\\
\ganttbar{Program management}{1}{12}\\
\end{ganttchart}
\caption{Schedule for the proposed work, measured in quarters from the
  beginning of the project.}
\label{fig:schedule}
\end{wrapfigure}
\textbf{Project management}: 
%
The success of the proposed research requires complementary expertise
and collaborative efforts in physics, applied mathematics, algorithms
and computing.  Ryham has been working on mathematical modeling with a strong
analytical background. Young has been working on many areas of computational
fluid dynamics and applications to math biology for many years.
Jiang has been working on integral
equation methods, fast algorithms, and their applications to 
computational electromagnetics and fluid dynamics
for many years.  Their recent collaborative work on
the HAP model in two dimensions has provided 
a solid foundation on the proposed research.

\medskip

\textbf{Collaboration plan}: 
%
The management responsibility of this collaborative research will
reside with the lead PI (Ryham) for this endeavor.  The research work is
structured to meet the tasks discussed in
Sec.~\ref{sec:proposed-work}.
%
The PIs, their postdoc and students will meet frequently in person.
The PIs will share software packages, paper sources, and references on a common
\textsf{Git} repository. The resulting software packages will be posted
on the \textsf{Github} software repository.

\medskip

\textbf{Research Schedule:} The detailed schedule for the proposed
work is shown in Fig.~\ref{fig:schedule}.


