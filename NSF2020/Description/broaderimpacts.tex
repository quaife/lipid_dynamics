\section{Broader Impacts}
\label{sec:BroaderImpacts}

This project aims to advance the mathematical modeling of collective 
dynamics of amphiphilic particles. The simulations use a new, yet intuitive,
approach that can account for important and complex systems that are out of 
reach in computational material science. These complex systems include 
fusion and fission of amphiphilic bilayer membranes and optimal shape design
in metamaterials. The development of three-dimensional 
models describing colloidal systems could be transformative in biomedicine
and material science. The research draws from expertise in scientific 
computing, physics of fluids, and mathematics. The mathematical component 
incorporates leading techniques from geometric analysis and gives deep insight 
into fundamental material science. The project offers undergraduates 
in a socially impactful manner the opportunity to do research and train 
with graduate and postdoctoral personnel. It incorporates research in the 
classroom, and with its combination of mathematical modeling, analysis, 
and scientific computing, the project highlights the importance of 
mathematics and computation to all areas of science and engineering.

%The proposed mathematical analysis, modeling, and numerical algorithms
%will transform our understanding of the collective dynamics of
%amphiphilic particles such as (1) their self-assembly into micelles and
%bilayers, (2) the material properties of such self-assembly, and (3) the
%interaction between these building blocks. In addition to biophysical
%applications, amphiphilic Janus particles have recent popularity in the
%fabrication of smart materials. The proposed work will have a
%transformative impact on precision design for specific mechanical
%properties of materials made of amphiphilic nanoparticles. An important
%component of this proposal is the interdisciplinary education and
%training of both undergraduate students, graduate students, and
%postdoctoral fellows. The combination of mathematical modeling,
%analysis, and scientific computing in this project provides a compelling
%example of the importance of mathematics and computation in biophysics
%and engineering applications. The concepts and methods described here go
%beyond the context of amphiphilic particles and lipid molecules. They
%extend to other problems featuring microscopic phase separation that
%leads to formation of mesoscopic domains. This situation arises, for
%example, in biological development in systems biology. The methods
%developed here have the potential to impact those and other related
%areas in biomedicine and biotechnology.

\subsection{Educational Impacts}
\label{subsec:Educational_plans}
The topics of this proposal have application in robotics, machine
learning, and engineering. To foster training in mathematical sciences,
the project will include undergraduate researchers (URs). The proposal
supports one UR per year from the lead institution, and will include one
or two additional URs from actively supported in-house programs

The URs will work as a team for eight weeks in the summer. They will
collaborate directly with the senior personnel, have tutorials in topics
such as numerical quadrature and integral equations, and train in
mathematical writing and presentation. To maximize vertical integration,
and to the extent possible, the PIs and personnel will travel to New
York for one or two weeks in the summers. Fordham University recently
constructed a seminar/collaboration space specifically for collaborative
research, and the URs will have office space with desks there.

As a requirement, the URs will write quality summary reports and give
presentations at the their respective research symposia. The work will
be considered a success if the URs also participate in a national
conference. The best possible outcome is if URs coauthor a publication.

There are two ways the proposal will meaningfully engage the community.
To be more impactful, students from the Bronx High School of Science (PI
RR) and Newark Science Park High School (PI YNY) will be encouraged to
join the research team. PI BQ will work with undergraduate and high
school students through the Undergraduate Research Opportunity Program
and Young Scholar's Program, as he has done in the past. Secondly, the
PIs will prioritize students who would most benefit from the grant
activity. These include students from underrepresented groups,
and we specifically target students whose socio-economic
background prevents them from participating in out-of-state research
experiences. 

Finally, we will create several modules to include in our courses. These
modules will introduce topics from numerical linear algebra and
optimization, and be based on the ideas of the proposal. Additionally,
we and our UR collaborators will give remote, guest lectures in each
others' courses and undergraduate seminars. 

%The proposed research will have an immediate impact on undergraduate and
%graduate education. At the undergraduate level, the project will include
%and support undergraduate researchers from Fordham University. The
%supported summer researchers will gain valuable first-hand
%interdisciplinary experience in mathematical modeling and computation.
%We will incorporate research into teaching, especially in differential
%equation and programming courses. PI YNY will work on simple modeling of
%self-assembly problems with both undergraduate and high school students
%from Newark Science Park High School. PI YNY has a track record of
%working with local high school students for their research projects
%before they apply for colleges. PI BQ will work with undergraduate and
%high school students through the Undergraduate Research Opportunity
%Program and Young Scholar's Program as he has done in the past.

%Over the past eight years, PI RR has included many undergraduates in the
%execution of research initiatives and coauthoring publications. He has a
%track record for supporting underrepresented groups, being called on by
%the Collegiate Science and Technology Entry Program (CSTEP) to make
%opportunities for promising students, for example. He has mentored two
%Clare Boothe Luce Scholars, one a U.S.~Marine Corp veteran, and a high
%school student in the NYU GSTEM program. 

%The PIs will capitalize on the synergy between both FSU's doctoral
%program in Scientific Computing, NJIT's doctoral program in Mathematical
%Sciences, and Fordham's undergraduate focus in the Department of
%Mathematics. Promising, young undergraduate scientists from the Bronx
%community will find a natural pipeline into graduate studies by working
%directly or indirectly on this project. At the graduate-level, PI YNY
%expects to train and support a PhD student for two more years. PI BQ
%will advise a postdoctoral fellow for two years. The PIs will foster
%vertical integration between the senior personnel, their collaborators,
%postdocs, doctoral, and Bachelor students, enhancing the learning
%environment.

%The PIs expect
%to involve undergraduate students in this project through
%their continuing and active involvement in undergraduate advising and research mentoring:
%YNY has mentored undergraduate students as a co-Investigator in CSUMS: Research and Education
%in Computational Mathematics for undergraduates in the Mathematical 
%Sciences at NJIT (funded by NSF) and the lecturer for Capstone Applied
%Mathematics Lab at NJIT (also funded by NSF).
%YNY has track records in involving undergraduate students in research
%that emphasizes both numerical computations and desktop experiments.
%
%such as MATH 340: Advanced Numerical Methods.
%Simple MATLAB codes have been used as teaching tools to show students how to use MATLAB
%to simulate the slender-body equations for an elastic filament in Stokes flow.
%
%In the past year YNY has been mentoring Ufuomaefe Ogbe, a high school student
%from Newark. Ufuomaefe is an African American who 
%worked with PI on a simplified model for vesicles in flow. Based
%on what he learned about modeling and programming, 
%he has applied to the Applied Mathematics/Computer Science programs at both NJIT
%and Rutgers/Newark.

\section{Intellectual Merit}
The purpose of this research is to reach interesting physical phenomena with 
less computational cost than molecular dynamics, and account for more general
features that continuum theory misses. The main ingredient is defining a 
nonlocal interaction through the solution of an elliptic boundary value problem
that has the phenomenological characteristics of long-range hydrophobic
attraction. It turns out that this minimal model gives rise to rich phenomena
for Janus particle aggregates and correctly predicts elastic properties of bilayer. 
The technical research tasks include quantifying collective properties of 
amphiphilic ensembles, mathematical analysis of continuum elastic energies, 
efficient, high-order numerical algorithms for large-scale simulations, and 
incorporating external fields such as through electric charge. Lastly, the proposal 
extends the results using three-dimensional boundary integral formulations.

%A central theme of the project is the mathematical development and
%analysis of the physical model for interaction between many amphiphilic
%particles. The model formulates the interaction potential through a
%screened Laplace equation boundary value problem possessing the physical
%properties like non-additivity and decay of realistic hydrophobic
%attraction. Colloidal systems collectively self-assemble into bilayer
%morphologies, and we analyze the elastic properties of these amphiphilic
%particle ensembles. This allows us to interpret the Helfrich free energy
%in terms of hydrophobic interactions and specific molecular
%characteristics. We extend the capabilities of the boundary integral
%equation and time stepping methods to stably perform large particle
%number simulations in three dimensions. Results from these computations
%will be used to compare collective amphiphilic against experiments and
%to study the optimal design of three-dimensional functional materials. 

