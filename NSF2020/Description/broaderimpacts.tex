\section{Broader Impacts of the Proposed Work}
\label{sec:BroaderImpacts}

The proposed mathematical analysis and modeling will provide a
transformative understanding of the collective dynamics of amphiphilic
particles such as (1) their self-assembly into micelles and bilayers,
and (2) the interaction between these building blocks. Amphiphilic Janus
particles have gained increasing popularity for fabrication of smart
materials. The proposed hybrid continuum model, mathematical analysis
and numerical algorithms will have transformative impact on precision
design for specific mechanical properties of materials made of
amphiphilic nanoparticles. An important component of this proposal is
the interdisciplinary education and training of both undergraduate
students, graduate students, and postdoctoral fellows. The combination
of mathematical modeling, analysis, and scientific computing in this
project provides a compelling example of the importance of mathematics
in biophysics and engineering applications. The concepts and methods
described here go beyond the context of amphiphilic particles and lipid
molecules. They extend to other problems featuring microscopic phase
separation that leads to formation of mesoscopic domains. This situation
arises, for example, in biological development in systems biology. The
methods developed here have the potential to impact those and other
related areas in biomedicine and biotechnology.

\subsection{Educational Impacts}
\label{subsec:Educational_plans}
The proposed research will have immediate impact on undergraduate and
graduate education. At the undergraduate level, the project will involve
and support undergraduate researcher from Fordham University. The
supported summer researchers will gain valuable first-hand experience in
inherently interdisciplinary, mathematical fluid modeling and
computation. We will incorporate research into teaching, especially in
differential equation and programming courses. PI Quaife will expose
undergraduate and high school students to the reserach through the
Undergraduate Research Opportunity Program (UROP) and Young Scholar's
Program (YSP).

Over the past eight years, PI Ryham has included many undergraduates in
the execution of research initiatives, coauthoring publications. He has
a track record for supporting underrepresented groups, being called on
by the Collegiate Science and Technology Entry Program (CSTEP) to make
opportunities for promising students, for example. He has mentored two
Clare Boothe Luce Scholars, one a U.S. Marine Corp veteran, and a high
school student in the NYU GSTEM program. 

The PIs will capitalize on the synergy between NJIT's doctoral program
in Mathematical Sciences and Fordham's undergraduate focus in the
Department of Mathematics. Promising, young undergraduate scientists
from the Bronx community while find a natural pipeline into graduate
studies through working directly or indirectly on this project. On the
graduate-level education, PI Young and expects to train and
support a PhD student, Yuexin Liu (who is in her third-year at NJIT) for
two more years. Yuexin Liu has past experiences in the undergraduate
training program at NJIT, and will train an undergraduate intern. The
PIs will foster vertical integration between the senior personnel, their
collaborators, postdocs, doctoral and Bachelor students, enhancing the
learning environment.
%


%The PIs expect
%to involve undergraduate students in this project through
%their continuing and active involvement in undergraduate advising and research mentoring:
%YNY has mentored undergraduate students as a co-Investigator in CSUMS: Research and Education
%in Computational Mathematics for undergraduates in the Mathematical 
%Sciences at NJIT (funded by NSF) and the lecturer for Capstone Applied
%Mathematics Lab at NJIT (also funded by NSF).
%YNY has track records in involving undergraduate students in research
%that emphasizes both numerical computations and desktop experiments.
%
%such as MATH 340: Advanced Numerical Methods.
%Simple MATLAB codes have been used as teaching tools to show students how to use MATLAB
%to simulate the slender-body equations for an elastic filament in Stokes flow.
%
%In the past year YNY has been mentoring Ufuomaefe Ogbe, a high school student
%from Newark. Ufuomaefe is an African American who 
%worked with PI on a simplified model for vesicles in flow. Based
%on what he learned about modeling and programming, 
%he has applied to the Applied Mathematics/Computer Science programs at both NJIT
%and Rutgers/Newark.

\section{Intellectual Merit}
A central theme of the project is the mathematical development and
analysis of the physical model for interaction between many amphiphilic
particles. The model formulates the interaction potential through a
screened Laplace equation boundary value problem possessing the physical
properties like non-additivity and decay of realistic hydrophobic
attraction. Colloidal systems collectively self-assemble into bilayer
morphologies, and we analyze the elastic properties of these amphiphilic
particle ensembles. This allows us to interpret the Helfrich free
energy in terms of hydrophobic interactions and specific molecular
characteristics. We extend the capabilities of the boundary integral
equation quadrature by expansion method to perform large particle number
simulations in three dimensions with collisions. Results from these
computations will be used to compare collective amphiphilic against
experiment, and study the optimal design of three-dimensional
functional materials. 



