\documentclass[11pt]{article}
\usepackage{fullpage}
\usepackage{todonotes}

\begin{document}
\topmargin0.25in
%\thispagestyle{empty}
\sloppy

\begin{center}
\Large \bf Data Management Plan
\end{center}
%\vspace{0.2in}

\subsection*{Materials to be produced in the course of the research}
The proposed research will produce data from analysis, numerical
simulations, and theoretical modeling. These data pertain to the
proposed research subjects on self-assembling amphiphilic particles in
solvent. These data will lead to quantitative understanding of the
formation of biophysical structures and smart materials.

\subsection*{The standards to be used for data and meta data format and
content}
Data generated from the proposed research will be shared among the
collaborators in their original data format and content. These data will
be available to the public once they are published in journals or
presented at conferences.

\subsection*{Policies for access and sharing}
The experimental data and numerical codes will be shared among
colleagues in the community, and the numerical MATLAB codes use version
control with the repositories linked on PI Ryham's
(faculty.fordham.edu/rryham), PI Quaife's
(people.sc.fsu.edu/$\sim$bquaife), and PI Young's
(web.njit.edu/$\sim$yyoung) webpages. All sharing and access of data are
allowed only under the consent from the PIs under the policies of
Fordham University, Florida State University (FSU), and New Jersey
Institute of Technology (NJIT). No sharing and access of data will be
allowed if illegal profit is gained.

\subsection*{Policies and provisions for re-use, re-distribution, and
the production of derivatives}
The data can be re-used and/or re-distributed only upon the consent from
the PIs under the policies of Fordham University, FSU, and NJIT. Any
derivatives from the data generated from the proposed research follow
the same rules.

\subsection*{Plans for archiving data, samples and other research
products}
The data will be archived following the policies of Fordham University,
FSU, and NJIT. All samples and research products will be protected by
Fordham University, FSU, and NJIT under the agreement with the National
Science Foundation.

\subsection*{Responsibility for Data Management}
All members of the investigative team including the PIs and graduate
students with access to data will receive instruction in the Responsible
Conduct of Research (Fordham University, FSU, and NJIT). The PIs are
responsible for enforcing the Data Management Plan outlined in this
document and the PIs will perform a quarterly check to make sure data
are properly stored, archived, and maintained. The PIs will conform
closely to the NSF policy on the dissemination and sharing of the
research data and materials created or gathered in the course of the
proposed project. The data acquired and preserved in the context of
this proposal will be further governed by Fordham University, FSU, and
NJIT's policies pertaining to intellectual property, record retention,
and data
management.

\subsection*{Types of data to be produced during the course of the
project}
{\bf PI Ryham}:
The following summarizes the type of numerical data to
be produced:
(i) Matlab codes for running simulations and
processing figures,
(ii) Gnuplot code for creating manuscript figures, 
(iii) C++ code for large-scale simulations, and 
(iv) and ASCII files of simulation outputs. The estimated size of these
data will be under 1 TB for the duration of the project.  

\noindent
{\bf PI Quaife}: The following summarizes the type of numerical data to
be produced: i) Matlab codes and data array structures to perform 2D
numerical simulations, (ii) Fortran and C++ codes and data array
structures to perform 3D numerical simulations, and (iii) binary data
from numerical solutions of amphiphilic particle suspensions. The
estimated size of all the data is approximately 1 TB per year. 

\noindent {\bf PI Young}: The following summarizes the type of
numerical/theoretical data to be produced: i) Mathematica files for the
theoretical modeling work, ii) Matlab codes and data array structures,
and iii) binary data from numerical simulations of the vesicle
osmophoresis. The estimated size of all the data is approximately 200 GB
per year. 
%\todo[inline]{Yuan to update. This is what was in the CBET proposal}

\subsection*{Standards to be used for data format and content}
C++ codes are stored as .cpp files, Fortran codes are stored as .f90
files, Matlab codes are stored as .m files, data array structures are
stored as .mat files, and numerical simulations are stored as .bin
files. Analyzed data will be summarized as graphs regularly and
documented as reports stored in .pdf format 

\subsection*{Methods and policies for providing access and sharing}
Any of the raw data from this project will be made available to other
researchers who may want to replicate or verify the results obtained.
Project data will be made available through university-operated
repository to facilitate the sharing of data. Data will be made
available when the publication(s) containing the data are in print. If
requested, data will be made available for sharing to qualified parties
by the PIs, so long as such a request does not compromise intellectual
property interests, interfere with publication, invade subject privacy,
betray confidentiality, or precede data curation. Data that are shared
will include standards and notations needed to interpret the data,
following commonly accepted practices in the field. Data will be shared
by using cloud-type storage like Dropbox if files are small or USB
key/DVD if files are large. 

\subsection*{Methods for archiving and preserving access to data and
materials}
Original data notebooks will be retained in a secure location in the
PIs' laboratories. All data will be backed up on a secure server
provided by the university and portable hard drives and will be archived
for at least at least three years beyond the award period, as required
by the NSF guidelines.


\end{document}

