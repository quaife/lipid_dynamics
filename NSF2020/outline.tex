
\documentclass[preprint,11pt]{article}
\usepackage[numbers,sort]{natbib}
\usepackage{graphicx}% Include figure files
\usepackage{dcolumn}% Align table columns on decimal point
\usepackage{bm}% bold math
\usepackage{courier}
\usepackage{subfig}
\usepackage{fullpage}
\usepackage{color}
\usepackage{amsmath}
\usepackage{cleveref}
\usepackage{amsfonts}
\usepackage{amssymb}
\usepackage{algorithm}
\usepackage{algpseudocode}
\usepackage{float}
\usepackage{relsize}
\usepackage{epstopdf}
\usepackage{authblk}
\usepackage{indentfirst}
\usepackage{mathtools}
\usepackage{bm}
\usepackage{commath}

% this is used for numbering in alphabets, for example, (a), (b)...etc
\newcommand{\subfigimg}[3][,]{%
  \setbox1=\hbox{\includegraphics[#1]{#3}}% Store image in box
  \leavevmode\rlap{\usebox1}% Print image
  \rlap{\hspace*{10pt}\raisebox{\dimexpr\ht1-2\baselineskip}{#2}}% Print label
  \phantom{\usebox1}% Insert appropriate spcing
}
\usepackage{amsmath,amssymb,amsthm}
\newtheorem{definition}{Definition}
\newtheorem{lemma}{Lemma}
\newtheorem{example}{Example}
\newtheorem{theorem}{Theorem}
\newtheorem{proposition}{Proposition}

\renewcommand{\d}[1]{\mathrm{d}#1}
\title{NSF Applied Math Collaborative Grant}
\author{}
\date{\today}
%\author[1]{Shidong Jiang}
%\author[2]{Szu-Pei P. Fu}
%\affil[1]{New Jersey Institute of Technology, Newark, NJ 07102}
%\affil[2]{Fordham University, Bronx, NY 10458}

\begin{document}
\maketitle
\everymath{\displaystyle}

\section{Introduction}
\subsection{Surface tension and HAP}
\subsection{Boundary Integral Equation Method}
\subsection{PDE and variational }

\section{Mathematical model for self-assembly: mechanical and rheological properties}
\subsection{Clustering behavior as a function of attraction repulsion}
\subsubsection{1D analogy}
\subsubsection{Mathematical heuristics}
\subsubsection{Experimental results}
\subsection{Consequences of symmetry breaking}
\subsubsection{Asymmetric shapes}
\subsubsection{Asymmetrical boundary conditions}
\subsection{Mechanical and rheological properties of particle mats}
\subsubsection{Thermal fluctuation}

\section{Long-range, generalization of surface tension}
\subsection{Boundary layer theory}
\subsection{$\Gamma$-convergence for energies}
\subsection{Conjectured maximal overlap}
\subsection{Curve shortening with HAP}

\section{Statistical mechanical derivation of Helfrich hamiltonian} 

\section{Maxwell stress tensor results}
\subsection{Linearized Debye-H\"uckel theory}
\subsection{Electro-static and related problems}
\subsection{Techniques for extending to 3D}
\subsection{Particle docking}
%%%%%%%%%%%%%%%%%%%%%%%%%%%%%%%%%%

\end{document}


% reaction from Mike Miksis
%
% will not make sense unless we can do many body systems efficiently and accurately
% mutual interaction from building blocks 
%
% We are interested in this many body system that assembles into bilayer membrane or
% into lattice with some functional structure. How do we understand these processes from
% a mathematical or modeling perspective.
% We don't want to do this from the nano scale, but from the micron scale.
%
% First we need to show that this can be done very efficiently. Donner's work argues that
% highly accurate solutions are not a trade-off for efficiency.
%
% We are evolving toward that goal of 3D. We are trying to come up with ways that make it faster.
% Regarding Item 5: we have to include charge and electrokinetics. Bob Kransy has charge distribution on
% surface and calculate potential of protien based on charge. Given charge on amphiphiles, we could in principle
% solve for interaction from potential.
%
% We should include charges (ref to Dimova), and show that the bending modulus changes with surface charge, charge in solution
% In the model, we have these parameters and they correspond to certain things in the experiment that come out of the self-
% assembly process that come out of the loaded situation.
%
% Relationship between the repulsion force and the tank treading phenomenon 
% 4 parameters: shear flow rate, surface tension, screening length and repulsion strength 
%
% Other than motivation, how to fit that into research agenda
% Chiral particles that rotate under magnetic fields, form aggregage, Irvin Willian from Chicago
% Mike Shelly works on continuum model of this system; no finite-size effects
%
% Upload old proposal, reviews. 


