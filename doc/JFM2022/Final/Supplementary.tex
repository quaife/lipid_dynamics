
\thispagestyle{empty}

\newpage
{\Large \bf

  \noindent Supplementary Material\\

  \noindent 
  Two-Dimensional Vesicle Hydrodynamics from Hydrophobic Attraction Potential}\\

\noindent 
Szu-Pei Fu$^{1,*},$ 
Bryan Quaife$^{2},$ 
Rolf Ryham$^{1},$ 
Y.-N. Young$^{3},$
\\


\noindent
$^{1}$Fordham University, Department of Mathematics,  Bronx, NY, USA

\noindent
$^{2}$Department of Scientific Computing, Florida State University, Tallahassee, Florida 32306, USA

\noindent
$^{3}$Department of Mathematical Sciences, New Jersey Institute of Technology, Newark, NJ  07102 USA
\\

\noindent $^*$Corresponding author. Address: Fordham University, Department of Mathematics, 441 E. Fordham Rd, Bronx, NY 10458. email: \text{sfu17@fordham.edu}



\setcounter{page}{1}

\setcounter{figure}{0}
\renewcommand{\thefigure}{S\arabic{figure}}

\setcounter{equation}{0}
\renewcommand{\theequation}{S\arabic{equation}}

\setcounter{section}{0}
\renewcommand{\thesection}{S\arabic{section}}   


%-----ellipse repulsion--------------------

%\Phi_{\mathrm{rep}}


\sloppy
\section{Movie Captions}\mbox{} \\

\noindent
{\bf Movie S1. Relaxation} There are 58 circular particles with radius 0.5 that form a self-enclosed bilayer structure. The simulation result shows the vesicle relaxation and the fluid pressure inside the JP vesicle decreases to 0 after a period of time.
The color field from blue to red shows the magnitude of fluid pressure and the range from 0 to 0.1. All dark blue dots in the domain represent the tracers in fluid that move with respect to calculated fluid motion.
This movies includes a total 5000 time steps where the time step is $\Delta t=0.2$.\\



\noindent
{\bf Movie S2. Single Vesicle Suspended in a Shear Flow} 
We adopt the relaxed configuration and place the JP vesicle in the shear flow. The simulation result shows a tank-treading motion when the shear rate is $\chi=0.005$.
The colored field from dark blue to dark red shows the magnitude of hydrophobic attraction activity and the range is from 0 to 1. All white dots in the domain represent the tracers in fluid that move with respect to calculated fluid motion.
This movies includes a total 20000 time steps where the time step is $\Delta t=0.2$.\\

\noindent
{\bf Movie S3. Inter-leaflet Sliding} 
Using the same setup demonstrated in Movie S2, we show the simulation result for 
inter-monolayer slip by tracking a pair of particles (blue and yellow). After $t=4000$, 
the yellow particle in the outer leaflet surpasses the blue particle in the inner leaflet by the distance about 1 particle diameter.
This movies includes a total 20000 time steps where the time step is $\Delta t=0.2$.\\


\noindent
{\bf Movie S4. Vesicle Rupture} 
The numerical results show that the JP vesicle starts to have rupture phase at the shear rate $\chi=0.0655$. The colored field from dark blue to dark red shows the magnitude of hydrophobic attraction activity and the range is from 0 to 1. All white dots in the domain represent the tracers in fluid that move with respect to calculated fluid motion. This movies includes a total 3000 time steps where the time step is $\Delta t=0.2$.\\


\noindent
{\bf Movie S5. Two Vesicles Suspended in a Shear Flow} 
There are two relaxed JP vesicles suspended in a shear flow. The two centroids are at coordinates $(-25,0)$ and $(25,0)$ in nm. The colored field from dark blue to dark red shows the magnitude of hydrophobic attraction activity and the range is from 0 to 1. All white dots in the domain represent the tracers in fluid that move with respect to calculated fluid motion. This movies includes a total 20000 time steps where the time step is $\Delta t=0.2$.\\


\noindent
{\bf Movie S6. Two Vesicles Suspended in an Extensional Flow} 
There are two relaxed JP vesicles suspended in an extensional flow. The centroid of the left JP vesicle is 0.25 nm above the $x$-axis and the right JP vesicle is 0.25 nm below the $x$-axis. The colored field from dark blue to dark red shows the magnitude of hydrophobic attraction activity and the range is from 0 to 1. All white dots in the domain represent the tracers in fluid that move with respect to calculated fluid motion.
This movies includes a total 6000 time steps where the time step is $\Delta t=0.2$.
%%%\end{document}


