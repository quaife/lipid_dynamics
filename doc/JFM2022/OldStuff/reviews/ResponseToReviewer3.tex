\documentclass[11pt]{article}

\usepackage{hyperref}
\usepackage{fullpage}
\usepackage{color}
\usepackage{newtxmath}
\newcommand{\comment}[1]{{\color{blue} #1}}

\begin{document}

\noindent
Dear Professor Petia Vlahovska,
\\ \\
\noindent
We thank you for handling the manuscript and the Reviewers for carefully
evaluating our work. The revisions address all the comments and
concerns raised by the Reviewers and we believe the result is a
significantly improved manuscript. An itemized list of the specific
changes and responses follows. Changes to the manuscript from Reviewers
1, 2, and 3 are colored in \textcolor{red}{red},
\textcolor{green}{green}, and \textcolor{magenta}{magenta},
respectively. \\ \\
\noindent
Sincerely,
\\
\noindent
Szu-Pei, Bryan Quaife, Rolf Ryham, and Yuan-Nan Young

\section*{Response to Reviewer 3}
\comment{\noindent
The manuscript extends the authors’ previous work on modeling
amphiphilic particles in a viscous solvent to modeling amphiphilic Janus
particles suspended in a viscous background flow. The authors adopted
the GMRES iterative method to solve the screened Laplace equation for
hydrophobic attraction. They investigated the inter-monolayer friction
of vesicles formed by amphiphilic Janus particles and corresponding
membrane ruptures in viscous flows, and simulated the spatial migration
dynamics of a vesicle in a parabolic flow and also the hydrodynamics of
two vesicles in shear and extensional flows.
\\ \\
\noindent
This manuscript presents a systematic study of the hydrodynamics of
vesicles formed by amphiphilic Janus particles in viscous flows. It
provides quantitative analysis of the inter-monolayer friction, membrane
permeability and the migration dynamics of soft suspended vesicles. The
topic on hydrodynamics of complex fluids is interesting, and this
research fits the scope of Journal of Fluid Mechanics. Here, I am in a
favor of its publication at some points if the authors can address the
following concerns in the revised manuscript.
}
\begin{itemize}
\item We thank the Reviewer for their positive feedback
  and suggestions which have helped improve the content and presentation
    of the manuscript.
\end{itemize}

\noindent
\comment{{\bf 1.} In section 2.2, the authors defined a repulsion force with an
empirical formula to prevent particle collisions. However, there is no
benchmark test to show that this artificial repulsion force can generate
correct hydrodynamic lubrication. The authors are suggested to justify
and validate the use of the Imposed Forces with comparison to well-known
lubrication forces between particles.}
\begin{itemize}
  \item Response: The repulsive potential we used to prevent overlap of
    Janus particles is the repulsive Morse potential based on the
    electrostatic repulsive potential between two negatively (or
    positively) charged particles immersed in a medium of low
    electrolyte concentration so that there is a well-defined zeta
    potential on the charged particle (Coakley {\sl et al.}, 1999,
    Biophys. J). This repulsive potential has been approximated by the
    short-range repulsive Lennard-Jones potential (Flormann {\sl et
    al.}, 2017, Sci. Reports), which has been adopted by Quaife and
    Young to study the interaction between two membranes in a viscous
    solvent in Quaife, Veerapaneni and Young, 2019, Phys. Rev. Fluids,
    where Quaife {\sl et al.} showed that the lubrication hydrodynamics
    (draining of fluid between two membrane) is well-captured and the
    scaling of lubrication thin film is recovered. 
  
This is evidence that the lubrication hydrodynamics of the thin film between two particles under the repulsive Morse potential is well captured if the flow velocity can be accurately calculated in the boundary integral simulations when two particles are sufficiently close to each other and the short-range repulsion is the dominant force. 

In this work we used similar numerical techniques to solve the integral equations when some boundaries are quite close to each other. As the repulsive Morse potential works well to prevent overlapping of particles (provided the time step size is sufficiently small), our previous results in Quaife, Veerapaneni and Young, 2019, Phys. Rev. Fluids gives us confidence in the accuracy of the fluid flow in the thin space between the particles.

\item Change: We added added a paragraph at the end of Section 2.2
  describing how the lubrication hydrodynamics in a narrow space between
    boundaries is well-captured within the boundary integral
    formulation. Then, in section 4.1 \textit{Model Parameters}, we
    show quantitatively how the lubrication and artificial repulsion are
    comparable when substituting for our simulation parameters.
\end{itemize}

\noindent
\comment{{\bf 2.} Fig.~3 plots the time evolution of the reduced area in (a) and
the excess length of the bilayer structure in (b). Both show obvious
oscillations with multiple frequencies, which are not as smooth as the
inset of (c) showing one dominant frequency. The authors are suggested
to explain the source of multiple frequencies generated in the time
evolution of $A$ and $\Delta$.}
\begin{itemize}
\item Response: We thank the Reviewer for highlighting this qualitative aspect of
  the curves.

\item Change: We modified the second to last paragraph in Section 4.2.1,
 noting that the frequencies correspond to the tank-treading of the slightly
  polygonal vesicle and individual particles, respectively. 
  Moreover, we scaled the abscissas of Figure 3 to percent, and corrected
  an erroneous scale factor in the time axes. 
\end{itemize}

\noindent
\comment{{\bf 3.} Fig.~6 illustrates the migration process of a JP vesicle in a
parabolic flow, where the moving velocity is 8 nm/ns and the time frame
is 0 to 12 $\mu$s. Given the length scale R = 20 nm, the migration
distance should be $1000 \times R$. However, in the figure, it shows a
migration distance about $4 \times R$. The authors are suggested to
double-check their model parameterization.}
\begin{itemize}
\item Response: Fig.~6 shows 4 snapshots in the time frame from 0 to 12
  $\mu$s along with the trajectory of the centroid.

  \item Change: In the caption of Fig.~6, we now clarify that the
    horizontal axis represents time and that the vertical axis has been
    scaled to make the migration stand out.
  
\end{itemize}

\noindent
\comment{{\bf 4.} In Fig.~10, the significant discontinuities in streamlines
across the surface of vesicles are observed, while in most other cases
the streamlines are smooth across the interface of vesicles. The authors
are suggested to describe such discontinuity and provide the
explanation.}
\begin{itemize}
\item Response: We thank the Reviewer for pointing this out.
  This issue occurred due to a limitation of the software used
  to generate streamlines in post-processing.

\item Change: We improved the software and the corrected streamlines now appear in Fig.~11.
  As expected, the streamlines are continuous while their gradient is
    piecewise continuous
  across the JP vesicle interface.
\end{itemize}

\noindent
\comment{Minor: I read some grammatical errors in the manuscript that
should be corrected, i.e., ``replicate well-know vesicle hydrodynamics
$\ldots$" on page 2, ``$\ldots$ a baseline JP vesicles $\ldots$" on page
8, ``circular vesicles has a Laplace pressure $\ldots$" at Eq.~(4.7).
The authors are suggested to have a thorough read of the manuscript.}
\begin{itemize}
  \item The typo with ``well-know" is corrected, but colored in
    \textcolor{red}{red} since it was pointed out by Reviewer 1 as well.

  \item The manuscript now reads
    \begin{quotation}
      $\ldots$ a baseline JP vesicle $\ldots$
    \end{quotation}

  \item The typo with ``has" instead of ``have" is corrected, but
    colored in \textcolor{red}{red} since it was pointed out by Reviewer
    1.

  \item We have read through the manuscript and done our best to
    remove further grammatical errors.

\end{itemize}


\end{document}
