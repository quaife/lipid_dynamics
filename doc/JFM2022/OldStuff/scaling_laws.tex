\documentclass[12pt,letterpaper, onecolumn]{article}
\usepackage{graphicx}
\usepackage{caption}
\usepackage{epsfig}
\usepackage{subcaption}
\usepackage{sectsty}
\usepackage{amssymb}
\usepackage{amsmath,empheq}


\begin{document}
\title{Nondimensionalization of Lipid Dynamics Model}
\maketitle


Length scale: a half of lipid length $[L]=1.25\ \text{nm}$\\

water viscosity at $20^\circ$C$=293.15^\circ$K: $\mu=1\ \text{cP}= 10^{-12}\ \frac{\text{kg}}{\text{sec}\cdot\text{nm}}$  \\

Line tension $\gamma=\frac{45}{11}$ mJ/nm$=1\ $kT/nm $\approx 4.1$ pN nm\\

set force scale $[F]=\gamma$, we obtain\\

time scale $\displaystyle [T] = \frac{\mu [L]^2}{[F]} = \frac{11\cdot 10^{-12}\ \frac{\text{kg}}{\text{sec}\cdot\text{nm}}\cdot 1.25^2\ \text{nm}^2}
{4.5\cdot10^{-2}\ \frac{\text{kg}\cdot\text{nm}}{\text{sec}^2}} 
\approx 3.82\times10^{-10}\ $sec\\

Energy scale $[E] = [F][L] = 1\cdot1.25\ $kT $= 5.13\ $pN nm\\


dimensionless shear rate $\chi = \dot\gamma [T]$\\


(ref. Finken, Eur. Phys. J. E. {\bf 25}, 2008) 


For $N=50$ vesicle simulations, the initial radius of the vesicle is \\ $R_0=6.75 [L] = 8.4375$ nm, the bending rigidity $\kappa=8.51$ kT $\approx35$ pN nm (SIAM MMS paper)\\

$\chi = \dot\gamma \cdot\frac{\mu R_0^2}{\kappa} = \dot\gamma \cdot\frac{  10^{-12}\frac{\text{ kg}}{\text{sec $\cdot$ nm}} \times 8.4375^3 \text{ nm}^3}{ 35\times 10^{-3} \frac{\text{ kg}\cdot\text{nm}^2}{\text{sec}^2}} = \dot\gamma \cdot 17.16\times10^{-9}$ sec $ = \dot\gamma \cdot 44.92[T]$. \\


(ref. Brandner et al.)
In Figure 7, the shear rates are \\(a) $\dot\gamma = 3.7\times10^7\ s^{-1}$; (b) $\dot\gamma = 1.9\times10^9\ s^{-1}$; (c) $\dot\gamma = 3.7\times10^9\ s^{-1}$\\

By applying our timescale [T], we have all dimensionless shear rates: \\
(a) $\chi = 0.0141$; (b) $\chi = 0.7258$; (c) $\chi = 1.41$.\\
%Moreover, the paper mentions that the time step is 10 fs. This suggests that we can 
%use $\Delta t \approx 2.5\times10^{-5}$\\


If we adopt the scaling law from Finken's paper with $R_0=10$ nm, then we have 
$\chi = \dot\gamma \cdot\frac{\mu R_0^2}{\kappa} \approx \dot\gamma \cdot 75[T]$. 


\section{What kind of velocities can we expect?}
The characteristic viscous forces must be in balance with the characteristic mechanical forces. 
For the kinds of system we deal with, there are two kinds of mechanical force densities:  
\begin{equation}
\text{due to tension } \gamma H \quad  \text{ and } \quad  \text{due to bending } k_B(\Delta_{\Sigma} H + 2H(H^2-K))
\end{equation}
where $\gamma$ is interfacial tension, $k_B$ is bending, $H$ is mean curvature, and $K$ is Gaussian curvature. 
If we are talking about a 10 nm oil droplet in water, then $\gamma = 30$ pN / nm is a typical oil-water tension and $k_B = 0$ kT.  The force due to tension would be like 
\begin{equation}
\gamma H = \gamma \frac{1}{2R} = 30 \frac{ \text{pN} }{\text{nm}} \frac{1}{20 \text{ nm }} = 1.5 \frac{ \text{pN} }{\text{nm}^2}
= O \left(\frac{ \text{pN} }{\text{nm}^2}\right)
\end{equation}
That is $10^6$ Pa, about 10 atmospheres.  If we are talking about a vesicle, then $\gamma = 0$ pN / nm and 
$k_B = 10 \text{ kT } \approx 41 \text{ pN nm}$.  The force due to bending assuming 
$O(1)$ second derivative in curvature would be 
\begin{equation}
k_B \Delta_{\Sigma} H 
=  O\left( \gamma \frac{1}{2R \text{ nm}^2}             \right) 
=  O\left( \frac{ 41 \text{ pN nm}}{2R \text{ nm}^2} \right)
=  O\left( \frac{\text{ pN }}{ \text{ nm}^2} \right)
\end{equation}
Bending and tension are comparable for nm deformations. 
  
The characteristic viscous forces are set by the gradient of the background velocity. 
Consider a shear background shear flow
\begin{equation}
\mathbf{u}_{\infty} = \dot \gamma \mathbf{x} \cdot \mathbf{e}_y \mathbf{e}_x
\end{equation}
With $\mu = 1 \text{ cP} = 1 \text{ pN ns nm}^{-2}$, the characteristic force density is 
\begin{equation}
\mu \nabla \mathbf{u}_{\infty} 
= O\left(\dot \gamma \frac{\text{ pN ns}}{\text{nm}^{2}} \right)
\end{equation}
We see that mechanical forces are comparable to viscous forces when $\dot \gamma = O( \text{ns}^{-1})$.
Thus it is not surprising that our shear rate us so large.

For a parabolic flow, 
\begin{equation}
\mathbf{u}_{\infty} = v_{\max}\left( 1 - \left( \frac{\mathbf{x} \cdot \mathbf{e}_y}{w R_0} \right)^2 \right)\mathbf{e}_x.
\end{equation}
In simulations, $|\mathbf{x}| \sim R_0$ and 
the characteristic force density is 
\begin{equation}
\mu \nabla \mathbf{u}_{\infty} 
= O\left(\frac{\text{ pN ns}}{\text{nm}^{2}}  \hat v_{\max}\frac{ \text{nm}}{ \text{ns}}\frac{2|\mathbf{x}|}{w^2R_0^2}\right)
= O\left(\frac{\hat{v}_{\max}}{5w^2}\frac{\text{ pN}}{\text{nm}^{2}} \right)
\end{equation}
For $w = 1$, say, viscous forces balance mechanical ones when $\hat{v}_{\max}$ order 10.  In the parabolic flow, the 
viscous forces balance the mechanical ones for $v_{\max}$ in the 10s of nm / ns.
\end{document}

























