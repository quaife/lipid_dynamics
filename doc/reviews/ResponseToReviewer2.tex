\documentclass[11pt]{article}

\usepackage{hyperref}
\usepackage{fullpage}
\usepackage{color}
\usepackage{newtxmath}
\newcommand{\comment}[1]{{\color{blue} #1}}

\begin{document}

\noindent
Dear Prof.~Petia Vlahovska,
\\ \\
\noindent
Thank you for handling the manuscript, and we would like to thank the
reviewer for their positive and constructive comments. The attached
manuscript addresses all the reviewer's comments and the result is what
we believe is a stronger manuscript. An itemized list of the changes
addressing the reviewer's comments are below. Changes to the manuscript
from Reviewers 1, 2, and 3 are colored in \textcolor{red}{red},
\textcolor{green}{green}, and \textcolor{magenta}{magenta},
respectively. \\ \\
\noindent
Sincerely,
\\
\noindent
Szu-Pei, Bryan Quaife, Rolf Ryham, and Yuan-Nan Young

\section*{Response to Reviewer 2}
\comment{
\noindent
The authors are introducing a novel numerical approach to study vesicles
under flow conditions in two dimensions. The vesicle is modelled as a
collection of amphiphilic Janus particles (JP) interacting with each
other through a hydrophobic attraction potential (HAP) and a short-range
repulsion force to avoid the overlap between particles. Both the
mobility of the particles and the HAP are recast in the form of an
integral equation of the second kind and solved numerically. The authors
presented some results on the deformation and dynamic of a single
vesicle and a pair of vesicles in different type of external flows.
\\ \\
\noindent
The originality of this work stems from coupling the HAP and Stokes
equations through a boundary integral formulation which I personally
find quite interesting. However, numerical models for vesicles and other
class of soft particles in 2D and 3D are already well established in the
literature. Disregarding the 2D nature of the model, I am not fully
convinced on how this approach outperforms the existing models either in
computational performance or in new applications. The authors should
make an extra effort to highlight this. Another weakness of this model,
if I understood correctly, is that the area of the vesicle is not
conserved beyond the small deformation limit. To the best of my
knowledge, a 2D vesicle should fulfill both constant area and perimeter
constraints. This means that the range of application of the model is
limited to the small deformation limits when used to mimic vesicles
under flow. Finally, this paper lacks a thorough validation of the
numerical results with the existing data in the literature and/or an
original application/result describing a physical phenomenon not covered
by the previous literature. This work is suitable for publication in
this journal but a major revision is needed. Below additional minor
comments to be addressed.}
\begin{itemize}
  \item {\bf Response}: We thank the Reviewer for this positive appraisal of our manuscript.  
The goal of our work, however, is not to provide an alternative vesicle model that outperforms pre-existing continuum models for a lipid bilayer vesicle in the literature. Inspired by de Gennes' suggestion of a thin film from self-assembly of Janus particles, we provide a model for the dynamics of Janus particles that self-assemble into a bilayer (JP vesicle) under various flowing conditions. 
As we showed in the manuscript, there are a lot of similarities between the JP vesicle and the lipid bilayer vesicle reported in the literature.

To highlight both the similarity and difference between our JP vesicle and the lipid bilayer vesicle, we study (1) the hydrodynamics of a JP vesicle in various flow conditions to compare with the corresponding results for a lipid bilayer vesicle,
and (2) we compute the friction between the two JP leaflet, and the permeability of a JP bilayer. We find that the permeability of a JP vesicle allows the total water content enclosed in the JP vesicle to reach an equilibrium value in a flow. We find that the JP vesicle is actually closer to a permeable vesicle that we studied previously, where the 2D vesicle has a constant perimeter but the enclosed area does not stay constant.

In terms of validation of our numerical results of JP vesicle of nano meter size, we have made comparison with the simulations of vesicle of the same size in the literature. For example we compare both the permeability and the inter-leaflet friction coefficient of a JP bilayer membrane with those of a lipid bilayer membrane. We also compare the critical shear rate for rupture between a JP vesicle and a lipid bilayer vesicle. We also compared the hydrodynamics of a JP vesicle and a lipid bilayer vesicle under various flowing conditions. These results are summarized in the original manuscript and re-organized in the revision with better clarity. 

Finally we provide a critical capillary number for rupture of a JP vesicle under a linear shear flow. The rupture of a lipid vesicle by a linear shear flow has been reported in experiments.  Our prediction of a critical capillary number and how it scales with various physical parameters of the individual Janus particle is, to our knowledge, not available in the literature. We hope that our results can motivate experimentalist (such as Charles Schroeder who has studied extreme deformation of a vesicle due to an imposed flow) and other theoreticians to study the dynamics of bilayer structures under extreme flowing conditions.
 
%  \begin{quotation}
%    \noindent
%    Can put verbatim change in this environment.
%  \end{quotation}
\end{itemize}


\noindent
\comment{The abbreviation HAP is not defined in the text. I guess it stands for hydrophobic attraction potential.
}
\begin{itemize}
  \item Response: Thank you for noticing this oversight. 
  \item Change: The definition of HAP has been added at its first occurrence in Section 3.1. 
\end{itemize}

\noindent
\comment{$u$ is used for both the velocity and the solutions of the
Laplace equation. Please use another variable for the scalar function
$u$.}
\begin{itemize}
  \item Response: The velocity ${\bf u}({\bf x})$ introduced in the mobility problem (eqn. (2.1)--(2.3)) is in bold face and of a vector form. The scalar solution function $u$ is the solution of the screened Laplace equation boundary value problem (eqn. (2.7)--(2.9)). Therefore there is no duplication of usage in notation. 
  \item Change: In response to this point, in order to avoid confusions, we changed the scalar function to $\eta$.
\end{itemize}

\noindent
\comment{$\rho$ used in the HAP is not the same as the one used in the
mobility boundary integral equation. One is a decay length of the
attraction potential, while the second is the Euclidean distance between
two JPs.}
\begin{itemize}
  \item Response: We had used $\rho$ to denote the distance between source and
    target points in one of the layer potentials. This has been removed.

  \item Change: We now use $\rho$ to denote the HAP decay length and $\rho_0$ to
    denote the cutoff for the repulsion force.
\end{itemize}

\noindent
\comment{$\nu$ is used on multiple instances, sometimes as a normal
vector (see line 97 for example), and other times as some scalar
function (see line 203).}
\begin{itemize}
  \item Response: We thank the referee for this oversight.
  \item Change: We now reserve $\boldsymbol{\nu}$ for the normal vector and
    $w_i$ denote the scalar what we previously defined to be $\nu_i$.
\end{itemize}

\noindent
\comment{What is the bending modulus here? I guess that it is not a
direct input parameter in this model but rather a quantity that can be
deduced from other input parameters. If I am right then, what are these
parameters? And could you tune them if needed to simulate the effect of
the membrane stiffness? This can be important in some applications such
as rigidity-based microfluidic sorting devices.}
\begin{itemize}
  \item Change 1 
\end{itemize}

\noindent
\comment{To compare the results with the existing literature, I would
suggest introducing the capillary number for vesicles as defined in
\url{https://doi.org/10.1016/j.crhy.2009.10.001}.}
\begin{itemize}
  \item Change 1 
\end{itemize}

\noindent
\comment{For both examples with the linear shear flow and the Poiseuille
flow, the area of the vesicle is not conserved and the relative error
with respect to the stress-free area increases with the increase of the
external stresses. It can reach roughly 7-8\% for the higher shear rate
in the shear flow example. This is quite significant. Do you have a way
to reduce this error to below 1\%? Note that in other numerical models
for vesicles (e.g.~\url{https://doi.org/10.1016/j.jcp.2008.11.036}),
both constant area and perimeter constraints are fulfilled even when the
vesicle is more deformed than the cases reported here.}
\begin{itemize}
  \item Response: We thank the referee for this comment. For impermeable vesicles, the referee is correct about the numerical errors reported in the literature. However, the Janus membrane is permeable and in our simulations we found that the Janus vesicle needs time so the Janus particles in the membrane can re-arrange themselves. During this process of rearrangement the Janus bilayer membrane behaves as a permeable membrane as in Quaife, Gannon and Young (PRF 2021). For a Janus vesicle in a linear shear flow with a shear rate of dimensionless shear $\chi = 0.002$ the enclosed area reaches a nearly constant value after $1$ $\mu$s. We would like to point out that this is not a numerical error, but a consequence of Janus particle hydrodynamics that gives rise to membrane permeability. 
\end{itemize}

\noindent
\comment{In some interesting applications, the fluid encapsulated by the
bilayer membrane has a viscosity which is different from the suspending
fluid. How difficult it is to add this feature to your model?}
\begin{itemize}
  \item Response: Our current numerical method does not support a viscosity contrast in the system 
or the setup similar to the ones in (Veerapaneni {\sl et al.}, {\bf 2011}, Phys. Rev. Lett.),
(Vlahovska and Gracia, {\bf 2007}, Phys. Rev. E), and
(Kaoui and Harting, {\bf 2016}, Rheol Acta). 
As described in the manuscript, the JP vesicle is permeable and the mixture of two different
fluids has to be analyzed within the framework, which would introduce a contact line between fluid 1- JP and fluid 2 -JP boundaries on the Janus particle surface. It is an interesting problem and we plan to study this in the future after we establish some basic knowledge of JP vesicle dynamics in a simpler configuration here.

\end{itemize}


\end{document}
