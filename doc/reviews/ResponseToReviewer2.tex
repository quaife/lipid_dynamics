\documentclass[11pt]{article}

\usepackage{hyperref}
\usepackage{fullpage}
\usepackage{color}
\usepackage{newtxmath}
\newcommand{\comment}[1]{{\color{blue} #1}}

\begin{document}

\noindent
Dear Prof.~Petia Vlahovska,
\\ \\
\noindent
Thank you for handling the manuscript, and we would like to thank the
reviewer for their positive and constructive comments. The attached
manuscript addresses all the reviewer's comments and the result is what
we believe is a stronger manuscript. An itemized list of the changes
addressing the reviewer's comments are below. Changes to the manuscript
from Reviewers 1, 2, and 3 are colored in \textcolor{red}{red},
\textcolor{green}{green}, and \textcolor{magenta}{magenta},
respectively. \\ \\
\noindent
Sincerely,
\\
\noindent
Szu-Pei, Bryan Quaife, Rolf Ryham, and Yuan-Nan Young

\section*{Response to Reviewer 2}
\comment{
\noindent
The authors are introducing a novel numerical approach to study vesicles
under flow conditions in two dimensions. The vesicle is modelled as a
collection of amphiphilic Janus particles (JP) interacting with each
other through a hydrophobic attraction potential (HAP) and a short-range
repulsion force to avoid the overlap between particles. Both the
mobility of the particles and the HAP are recast in the form of an
integral equation of the second kind and solved numerically. The authors
presented some results on the deformation and dynamic of a single
vesicle and a pair of vesicles in different type of external flows.
\\ \\
\noindent
The originality of this work stems from coupling the HAP and Stokes
equations through a boundary integral formulation which I personally
find quite interesting. However, numerical models for vesicles and other
class of soft particles in 2D and 3D are already well established in the
literature. Disregarding the 2D nature of the model, I am not fully
convinced on how this approach outperforms the existing models either in
computational performance or in new applications. The authors should
make an extra effort to highlight this. Another weakness of this model,
if I understood correctly, is that the area of the vesicle is not
conserved beyond the small deformation limit. To the best of my
knowledge, a 2D vesicle should fulfill both constant area and perimeter
constraints. This means that the range of application of the model is
limited to the small deformation limits when used to mimic vesicles
under flow. Finally, this paper lacks a thorough validation of the
numerical results with the existing data in the literature and/or an
original application/result describing a physical phenomenon not covered
by the previous literature. This work is suitable for publication in
this journal but a major revision is needed. Below additional minor
comments to be addressed.}
\begin{itemize}
  \item {\bf Response to Summary}: We thank the Reviewer for this positive appraisal of 
our manuscript. The proposed 2D JP vesicle model has a geometrical property 
of constant perimeter with a finite number of particles. Study of the permeability gives a numerical evidence that the constant area constraint of this enclosed bilayer structure does not apply in this 
work. In response, we performed a number of simulations and provided more parametric study in 
terms of small changes in parameters. 
%  \begin{quotation}
%    \noindent
%    Can put verbatim change in this environment.
%  \end{quotation}
\end{itemize}


\noindent
\comment{The abbreviation HAP is not defined in the text. I guess it stands for hydrophobic attraction potential.
}
\begin{itemize}
  \item Thank you for noticing this oversight. The definition of HAP has
    been added at its first occurrence in Section 3.1. 
\end{itemize}

\noindent
\comment{$u$ is used for both the velocity and the solutions of the
Laplace equation. Please use another variable for the scalar function
$u$.}
\begin{itemize}
  \item Change 1 
\end{itemize}

\noindent
\comment{$\rho$ used in the HAP is not the same as the one used in the
mobility boundary integral equation. One is a decay length of the
attraction potential, while the second is the Euclidean distance between
two JPs.}
\begin{itemize}
  \item We had used $\rho$ to denote the distance between source and
    target points in one of the layer potentials. This has been removed.

  \item We now use $\rho$ to denote the HAP decay length and $\rho_0$ to
    denote the cutoff for the repulsion force.
\end{itemize}

\noindent
\comment{$\nu$ is used on multiple instances, sometimes as a normal
vector (see line 97 for example), and other times as some scalar
function (see line 203).}
\begin{itemize}
  \item We now reserve $\boldsymbol{\nu}$ for the normal vector and
    $w_i$ denote the scalar what we previously defined to be $\nu_i$.
\end{itemize}

\noindent
\comment{What is the bending modulus here? I guess that it is not a
direct input parameter in this model but rather a quantity that can be
deduced from other input parameters. If I am right then, what are these
parameters? And could you tune them if needed to simulate the effect of
the membrane stiffness? This can be important in some applications such
as rigidity-based microfluidic sorting devices.}
\begin{itemize}
  \item Change 1 
\end{itemize}

\noindent
\comment{To compare the results with the existing literature, I would
suggest introducing the capillary number for vesicles as defined in
\url{https://doi.org/10.1016/j.crhy.2009.10.001}.}
\begin{itemize}
  \item Change 1 
\end{itemize}

\noindent
\comment{For both examples with the linear shear flow and the Poiseuille
flow, the area of the vesicle is not conserved and the relative error
with respect to the stress-free area increases with the increase of the
external stresses. It can reach roughly 7-8\% for the higher shear rate
in the shear flow example. This is quite significant. Do you have a way
to reduce this error to below 1\%? Note that in other numerical models
for vesicles (e.g.~\url{https://doi.org/10.1016/j.jcp.2008.11.036}),
both constant area and perimeter constraints are fulfilled even when the
vesicle is more deformed than the cases reported here.}
\begin{itemize}
  \item Change 1 
\end{itemize}

\noindent
\comment{In some interesting applications, the fluid encapsulated by the
bilayer membrane has a viscosity which is different from the suspending
fluid. How difficult it is to add this feature to your model?}
\begin{itemize}
  \item Change 1 
\end{itemize}


\end{document}
