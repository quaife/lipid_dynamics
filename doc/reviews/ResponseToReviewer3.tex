\documentclass[11pt]{article}

\usepackage{hyperref}
\usepackage{fullpage}
\usepackage{color}
\usepackage{newtxmath}
\newcommand{\comment}[1]{{\color{blue} #1}}

\begin{document}

\noindent
Dear Prof.~Petia Vlahovska,
\\ \\
\noindent
Thank you for handling the manuscript, and we would like to thank the
reviewer for their positive and constructive comments. The attached
manuscript addresses all the reviewer's comments and the result is what
we believe is a stronger manuscript. An itemized list of the changes
addressing the reviewer's comments are below. Changes to the manuscript
from Reviewers 1, 2, and 3 are colored in \textcolor{red}{red},
\textcolor{green}{green}, and \textcolor{magenta}{magenta},
respectively. \\ \\
\noindent
Sincerely,
\\
\noindent
Szu-Pei, Bryan Quaife, Rolf Ryham, and Yuan-Nan Young

\section*{Response to Reviewer 3}
\comment{\noindent
The manuscript extends the authors’ previous work on modeling
amphiphilic particles in a viscous solvent to modeling amphiphilic Janus
particles suspended in a viscous background flow. The authors adopted
the GMRES iterative method to solve the screened Laplace equation for
hydrophobic attraction. They investigated the inter-monolayer friction
of vesicles formed by amphiphilic Janus particles and corresponding
membrane ruptures in viscous flows, and simulated the spatial migration
dynamics of a vesicle in a parabolic flow and also the hydrodynamics of
two vesicles in shear and extensional flows.
\\ \\
\noindent
This manuscript presents a systematic study of the hydrodynamics of
vesicles formed by amphiphilic Janus particles in viscous flows. It
provides quantitative analysis of the inter-monolayer friction, membrane
permeability and the migration dynamics of soft suspended vesicles. The
topic on hydrodynamics of complex fluids is interesting, and this
research fits the scope of Journal of Fluid Mechanics. Here, I am in a
favor of its publication at some points if the authors can address the
following concerns in the revised manuscript.
}
\begin{itemize}
  \item We would like to thank the referee for the help with the presentation, which
has been modied as follows.
%  \begin{quotation}
%    \noindent
%    Can put verbatim change in this environment.
%  \end{quotation}
\end{itemize}

\noindent
\comment{{\bf 1.} In section 2.2, the authors defined a repulsion force with an
empirical formula to prevent particle collisions. However, there is no
benchmark test to show that this artificial repulsion force can generate
correct hydrodynamic lubrication. The authors are suggested to justify
and validate the use of the Imposed Forces with comparison to well-known
lubrication forces between particles.}
\begin{itemize}
  \item Response: The repulsive potential we used to prevent overlap of Janus particles is the repulsive Morse potential based on the electrostatic repulsive potential between two negatively (or positively) charged particles immersed in a medium of low electrolyte concentration so that there is a well-defined zeta potential on the charged particle (Coakley {\sl et al.}, 1999, Biophys. J). This repulsive potential has been approximated by the short-range repulsive Lennard-Jones potential (Flormann {\sl et al.}, 2017, Sci. Reports), which has been adopted by Quaife and Young to study the interaction between two membranes in a viscous solvent in Quaife, Veerapaneni and Young, 2019, Phys. Rev. Fluids, where Quaife {\sl et al.} showed that the lubrication hydrodynamics (draining of fluid between two membrane) is well captured and the scaling of lubrication thin film is recovered. 
  
This is evidence that the lubrication hydrodynamics of the thin film between two particles under the repulsive Morse potential is well captured  if the flow velocity can be accurately calculated in the boundary integral simulations when two particles are sufficiently close to each other and the short-range repulsion is the dominant force. 

In this work we used similar numerical techniques to solve the integral equations when some boundaries are quite close to each other.  As the repulsive Morse potential works well to prevent overlapping of particles (provided the time step size is sufficiently small), our previous results in Quaife, Veerapaneni and Young, 2019, Phys. Rev. Fluids gives us confidence in the accuracy of the fluid flow in the thin space between the particles.

\item Change: We added sentences to describe how the lubrication hydrodynamics in a narrow space between boundaries is well-captured within the boundary integral formulation in page 4, starting in line 129.

\end{itemize}

\noindent
\comment{{\bf 2.} Fig.~3 plots the time evolution of the reduced area in (a) and
the excess length of the bilayer structure in (b). Both show obvious
oscillations with multiple frequencies, which are not as smooth as the
inset of (c) showing one dominant frequency. The authors are suggested
to explain the source of multiple frequencies generated in the time
evolution of $A$ and $\Delta$.}
\begin{itemize}
  \item Fig.~3(c) clearly shows that the total length of the JP vesicle has a steady periodic motion 
which is different from the trends shown in panels (a) and (b). We showed the actual time evolution of $A^*$ and $\Delta$ and fitting curves in (a). 
The idea is to show that we can use the fitting data to acquire the numerical expected teminal 
reduced area $A^*_\infty$. Therefore the multiple frequencies during simulations have no correlations between
the output in (c) and (d).
\end{itemize}

\noindent
\comment{{\bf 3.} Fig.~6 illustrates the migration process of a JP vesicle in a
parabolic flow, where the moving velocity is 8 nm/ns and the time frame
is 0 to 12 $\mu$s. Given the length scale R = 20 nm, the migration
distance should be $1000 \times R$. However, in the figure, it shows a
migration distance about $4 \times R$. The authors are suggested to
double-check their model parameterization.}
\begin{itemize}
  \item Fig.~6 shows snapshots for 4 time frames from 0 to 12 $\mu$s and the trajectory of the centriod is demonstrated. Therefore the dashed horizontal line of Fig.~6 represents time in units of $\mu$s and the vertical axis represents the scaled $y$ position of the centroid of the JP vesicle.
\end{itemize}

\noindent
\comment{{\bf 4.} In Fig.~10, the significant discontinuities in streamlines
across the surface of vesicles are observed, while in most other cases
the streamlines are smooth across the interface of vesicles. The authors
are suggested to describe such discontinuity and provide the
explanation.}
\begin{itemize}
  \item We corrected the original Fig.~10 by plotting all continuous streamlines. We thank the Reviewer for pointing this out and this issue occurred due to a mistake made in post-processing
steps. Moreover, we do expect some discontinuous streamlines where it occurs when a strealine flows across the JP vesicle interface.
\end{itemize}

\noindent
\comment{Minor: I read some grammatical errors in the manuscript that
should be corrected, i.e., ``replicate well-know vesicle hydrodynamics
$\ldots$" on page 2, ``$\ldots$ a baseline JP vesicles $\ldots$" on page
8, ``circular vesicles has a Laplace pressure $\ldots$" at Eq.~(4.7).
The authors are suggested to have a thorough read of the manuscript.}
\begin{itemize}
  \item The typo with ``well-know" is corrected, but colored in red
    since it was pointed out by Reviewer 1.

  \item The manuscript now reads
    \begin{quotation}
      $\ldots$ a baseline JP vesicle $\ldots$
    \end{quotation}

  \item The typo with ``has" instead of ``have" is corrected, but
    colored in red since it was pointed out by Reviewer 1.

  \item We have read through the manuscript and made grammatical
    changes.

\end{itemize}


\end{document}
