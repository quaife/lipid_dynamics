\documentclass[11pt]{article}

\usepackage{fullpage}
\usepackage{color}
\usepackage{newtxmath}
\newcommand{\comment}[1]{{\color{blue} #1}}

\begin{document}

\noindent
Dear Prof.~Petia Vlahovska,
\\ \\
\noindent
Thank you for handling the manuscript, and we would like to thank the
reviewer for their positive and constructive comments. The attached
manuscript addresses all the reviewer's comments and the result is what
we believe is a stronger manuscript. An itemized list of the changes
addressing the reviewer's comments are below. Changes to the manuscript
from Reviewers 1, 2, and 3 are colored in \textcolor{red}{red},
\textcolor{green}{green}, and \textcolor{magenta}{magenta},
respectively. \\ \\
\noindent
Sincerely,
\\
\noindent
Szu-Pei, Bryan Quaife, Rolf Ryham, and Yuan-Nan Young

\section*{Response to Reviewer 1}
\noindent
\comment{Overall, the paper is original and interesting. The derivation
of the hydrophobic force without singular integrals is smart and well
written. The characterization of the effective properties of the
coarse-grained vesicles is also of interest and carefully conducted.
However, in many instances the manuscript looks more like a catalogue of
the different behaviors JP vesicles can exhibit in linear flows, rather
than a comprehensive study. The authors describe a few interesting
phenomena but only partially address the exciting physical questions
related to them. The paper lacks quantitative comparisons with continuum
models, parametric studies and detailed explanations to meet the
standards of JFM. In addition, the manuscript seems to have been written
in haste: a figure with no apparent connection to the content of the
paper has been inserted without being referenced or discussed in the
text (see below), some parts need some serious proofreading and
additional clarifications.}
\begin{itemize}
  \item {\bf Response to Summary}: 	We thank the Reviewer for the assessmnet of our manuscript submission. In response
to the comments, we have undertaken a major revision in including parametric studies and quantitative comparisons between models. We perforemed a number of simulations 
in studies of inter-monolayer friction coefficients and an additional numerical investigation 
of permeabilities. We have ensured that all figures are conneted with the corresponding content. 
  \begin{quotation}
%    \noindent
%    Can put verbatim change in this environment.
  \end{quotation}
\end{itemize}

\noindent
\comment{There is no parametric study of the effect of the particle size
on the effective properties of the vesicles (interlayer friction,
stretching modulus, permeability,etc$\ldots$). The authors defer this
part for future work in the conclusions, but since the effect of the
particle size is important, as evidenced in Section 4.3, it should be
explored and explained more thoroughly in the manuscript, otherwise the
research would be too incremental. The effect of particle shape seems
indeed more complex and out of the scope of the current paper.}
\begin{itemize}
  \item This comment encouraged us to have more convergence studies in the choices of model parameters. We made a worthwhile effort to perform a large number of simulations. 
We revised Table 1 and added Table 2 to account for effective properties of vesicles with different size of particles. We agree with the Reviewer that the particle/vesicle shape is complex with the setup therefore we adopted small changes in particle sizes and reported all the reults.





\end{itemize}

\noindent
\comment{The membrane rupture situation, briefly illustrated in Section
4.2.3, is not investigated at all. Is there a dimensionless number
related to this phenomenon with a critical value above which rupture
happens? (e.g.~quantifying the competition between shear and
attraction)}

\noindent
\comment{I understand that the results of Section 4.4 are mostly
qualitative and that the goal is to show the visual agreement between
the coarse-grained and continuum vesicle models, but it would be
insightful to provide quantitative comparisons to measure how close they
are to each other.}
\begin{itemize}
  \item We agree with the Reviewer that we briefly illustrated the capability of simulating membrane rupture via the proposed model and numerical algorithm in the first submission of the manuscript. We further performed a series of study to find the critical shear rate for the appearance of membrane rupture and the corresopnding movie is added as a supplementary material. We also included the description in the section of membrane rupture.
\end{itemize}

\noindent
\comment{The problem is not nondimensionalized with the characteristic
scales of the system but with ns, nm and $\text{pN} \cdot
\text{nm}^{-2}$. This (unusual) nondimensionalization makes it hard to
compare physical quantities and to define order parameters that would
explain/characterize some phenomena observed in the simulations
(e.g.~the membrane rupture).}

\begin{itemize}
  \item We thank the Reviewer to point this out. There is a significant difference between 
the proposed model in the manuscript and other continuum models in literatures. For example, refer to \emph{Rahimian, et al., J. Comput. Phys., 2011}, the 
provided time scale is $\tau = \mu R_0^3/\kappa_B$ where $\mu$ the fluid viscosity and $\kappa_B$ the bending modulus. $R_0$ is the initial radius of the reference vesicle. Since the bending modulus does not play any roles in the proposed model, we then scaled time unit with a realistic characteristic time ns. With the use of the detailed scaling laws described in section 4.1, we aim to have comparable
numerical results against the experiments. Moreover, if we attempt to calculate the characterstic 
time scale using characteristic length and energy scales, it will also result a vaule of order ns. 
Since the system involves multiple length scales, diameter of the particle $l_0$ and size of the JP
vesicle $R_0$, we decided to use ns to perform nondimensionalization for simplicity.
\end{itemize}

\noindent
\comment{For the sake of clarity, at the end of Section 3 the authors
should summarize the exact set of equations they use and the unknowns
they solve for.}
\begin{itemize}
  \item We followed the suggestion given by the Reviewer that a summary about what to solve throughout the entire algorithm is provided at the end of Section 3.
\end{itemize}

\noindent
\comment{General comment for all the figures with colormaps: it would be
good to use colorblind friendly colormaps for colorblind people and for
black and white printing.}
\begin{itemize}
  \item We thank the Review for having this meaningful comment and we agree that changing all the figure 
to colorblind friendly colormap is insightful. We would love to make this change once the manuscript is accepted.
\end{itemize}

\noindent
\comment{Figure 1: the arc length $L$ does not appear in panel (c).}
\begin{itemize}
  \item We deleted the sentence ``$L$ is the arc length''.
\end{itemize}

\noindent
\comment{Line 50: typo: ``well-known"}
\begin{itemize}
  \item This typo is corrected.
\end{itemize}

\noindent
\comment{Page 3: the first two paragraphs seem to be a bit disconnected
from the rest of the introduction. Maybe the authors should rewrite that
part so that there is a transition from the physical problem into
consideration to the various numerical methods used in the literature to
simulate it.}
\begin{itemize}
  \item Change 1 
\end{itemize}

\noindent
\comment{Line 70: ``we maintain contact-free suspensions with a
relatively weak non-stiff repulsive force". What do you mean by
relatively weak repulsive force ? Weak compared to what? (see additional
question on this hereinafter).}
\begin{itemize}
  \item Change 1 
\end{itemize}

\noindent
\comment{Equation (2.4) is not necessary as the no-slip boundary
condition is standard and it is redundant with (2.5).}
\begin{itemize}
  \item We deleted the redundant equation (2.5) mentioned by the Reviewer.
\end{itemize}

\noindent
\comment{Section 2.2: for clarity, define the variable u as the
hydrophobic attraction potential in the first paragraph.}

\noindent
\comment{The equations of the continuous time evolution of the particle
positions and orientations should be written in Section 2 to close the
problem.}
\begin{itemize}
  \item Section 2.2 introduces the imposed forces and torques in hydrophobic attraction and repulstion that are not used to obtain particle directions directly. Following the next comment by 
the Reviewer, we agreed that Section 2.3 should be moved to the end of Section 3 and we complete the change.
\end{itemize}

\noindent
\comment{I think the time-discretization Section 2.3 should go at the
end of Section 3 since this is the one devoted to the numerical
discretization of the governing equations. Also, the sentence ``particle
collisions are avoided even when using a relatively large time step" is
very vague.  What do you mean by ``relatively large time step"? Compared
to what time scale?}
\begin{itemize}
  \item We thank the Review for the comment and we moved the original Section 2.3 {\it Time Marching} at the end of Section 3 combining the change based on the comment previously addressed about what to solve for in the proposed algorithm. 
As described in Section 2.2, the repulsion force with a constant strength coefficient $M$ is chosen 
to be a large number in order to avoid particle collisions. If $M$ is not large enough, the hydrophobic attraction will dominate the total force and the dynamics may cause particle collisions with the use of a larger time step. Therefore, with an empirical study to the relationship between 
choices of M and $\gamma$, the parameter set in this work allows the time step to be as large as
$O(10^{-1})$ with no error. Comparing the time step used in literatures, for instance, in Bystricky {\sl et al.}, Int. J. Numer. Methods Fluids), $\Delta t\sim O(10^{-2})$.
\end{itemize}

\noindent
\comment{Section 3: the acronym HAP is not defined in the text.}
\begin{itemize}
  \item Thank you for noticing this oversight. The definition of HAP has
    been added at its first occurrence in Section 3.1. 
  \item This change is in \textcolor{green} since it was also pointed
    out by Reviewer 2.
\end{itemize}

\noindent
\comment{Equation 3.1: use parenthesis to show that the normal
derivative only applies to the Laplace’s Green function $K_0$ and not
the density $\sigma$.}
\begin{itemize}
  \item Parenthesis have been added to equation (3.1) 

  \item Equations (3.2) and (3.11) had the same notational issue, and
    they have both been corrected.
\end{itemize}

\noindent
\comment{For clarity, use contraction products ``$\cdot$" when a second
order tensor multiplies a vector to avoid confusion (e.g.~${\bf S} \cdot
{\bf F}$ instead of ${\bf SF}$) in eqs.~(3.4), (3.5), (3.7) and so on
$\ldots$}
\begin{itemize}
  \item This notation has been adjusted throughout Section 3.
\end{itemize}

\noindent
\comment{In practice, how do you compute the derivatives $d\sigma /ds$,
$dv_i/ds$ and $dv_i/d\boldsymbol{\nu}$ in (3.13)?}
\begin{itemize}
  \item Since the only derivatives that are required are functions
    defined only on the boundary of the Janus particles, they are
    computed with spectrally-accurate Fourier differentiation.

  \item We have added the following sentence to the start of Section 3
    \begin{quotation}
      All derivatives are computed with spectrally-accurate Fourier
      differentiation.
    \end{quotation}
\end{itemize}

\noindent
\comment{Line 256: you set the strength of the repulsive force $M = 4k_B
T$. In your simulations there is no thermal fluctuations, so why using
$k_B T$ as a reference energy if it is no there in the simulations?
There should be another energy scale that you could use to calibrate
$M$, right?}
\begin{itemize}
  \item Change 1 
\end{itemize}

\noindent
\comment{Related question: how much does the interparticle distance
depend on on the strength and range of this artificial repulsive
potential? What is the impact on the vesicle shape and effective
properties?}
\begin{itemize}
  \item Change 1 
\end{itemize}

\noindent
\comment{Figure 3a,b,d: what does the term in parenthesis ``(1)" means
on the label of the ordinate axis?}
\begin{itemize}
  \item The term (1) means the magnitude of the corresponding $y$-axis is of order 1.
\end{itemize}

\noindent
\comment{Line 302: the authors mention that tank-threading can only be
observed for shear rate as high as $10^{6}\text{s}^{-1}$. Can such a
high value be generated in experiments?}
\begin{itemize}
  \item A shear rate of order $10^{6}\text{s}^{-1}$ is too high to be generated in experiments.
Since the size of the JP vesicle in this work is much smaller than experimental use vesicles, 
a very high shear rate of the flow is needed to acquire the tank-treading motion.
\end{itemize}

\noindent
\comment{Eq.~4.4, typo: $v$ should be $u$.}
\begin{itemize}
  \item This typo is fixed.
\end{itemize}

\noindent
\comment{Figure 6: the yellow particle is not visible when printing in
black and white.}
\begin{itemize}
  \item Change 1 
\end{itemize}

\noindent
\comment{Line 388, typo: ``has" should be ``have".}
\begin{itemize}
  \item This typo is corrected.
\end{itemize}

\noindent
\comment{Figure 7a is never referred to and its content is never
discussed in the text.}
\begin{itemize}
  \item Thank you for noticing this missing connection. We added a description connected to Figure 7a.
\end{itemize}

\noindent
\comment{Line 393: The reference to panel ``Figure 7c" seems to be a
typo. Same for ``Figures 7c and 7d" on line 394, I think it should be
``Figures 7b and 7c".}
\begin{itemize}
  \item This typo is corrected.
\end{itemize}

\noindent
\comment{Side question: Is it possible to change the vesicle stretching
modulus and hydraulic permeability independently or are these two
properties linked to each other and how?}
\begin{itemize}
  \item We added a parametric study in new Table 2 and the trend provides an idea about 
how to change vesicle stretching modulus and hydraulic permeability by adjusting the length related parameters or force related parameters. We think that it is possible to have better understanding about the link between all parameters and these physical quantities after more 
convergence study.
\end{itemize}


\end{document}
