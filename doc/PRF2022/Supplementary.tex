
\thispagestyle{empty}

\newpage
{\Large \bf

  \noindent Supplementary Material\\

  \noindent 
  Collective behavior of Janus Particles  Suspended in a Viscous Fluid}\\

\noindent 
Szu-Pei Fu$^{1,*},$ 
Rolf Ryham$^{2},$ 
Bryan Quaife$^{3},$ 
Y.-N. Young$^{4},$
\\

\noindent
$^{1}$Department of Mathematics, Trinity College, Hartford, Connecticut 06106, USA

\noindent
$^{2}$Department of Mathematics, Fordham University,  Bronx, NY, USA

\noindent
$^{3}$Department of Scientific Computing, Florida State University, Tallahassee, Florida 32306, USA

\noindent
$^{4}$Department of Mathematical Sciences, New Jersey Institute of Technology, Newark, NJ  07102 USA
\\

\noindent $^*$Corresponding author. Address: Department of Mathematics, Trinity College, 
300 Summit Street, Hartford, CT 06106. email: \text{peter.fu@trincoll.edu}



\setcounter{page}{1}

\setcounter{figure}{0}
\renewcommand{\thefigure}{S\arabic{figure}}

\setcounter{equation}{0}
\renewcommand{\theequation}{S\arabic{equation}}

\setcounter{section}{0}
\renewcommand{\thesection}{S\arabic{section}}   


%-----ellipse repulsion--------------------

%\Phi_{\mathrm{rep}}


\sloppy
\section{Movie Captions}\mbox{} \\

\noindent
{\bf Movie S1. Relaxation} 
There are 60 circular particles with radius 0.5 that are initially confined in a square box.
The simulation results show the relaxation with the use of all three boundary conditions.
The color on the boundary from blue to red shows the magnitude of the boundary data and 
the range is from $\min g(\xx)$ to $\max g(\xx)$. All final configurations are adopted in 
simulations with hydrodynamic flows. The time step of all simulations is $\Delta t=0.2$.\\


\noindent
{\bf Movie S2. Structures in a Shear Flow without Ruptures} 
We adopt the relaxed configurations and place the JP structures in the shear flow. 
In this movie, there is no apparent structural rupture in each case. For the choices of the shear rate, we pick: $\dot\gamma = 0.05$ for the vesicle, $\dot\gamma = 0.05$ for the bilayer, 
$\dot\gamma = 0.05$ for the multi-lamellar, and $\dot\gamma = 0.1$ for the striated configurations.
The time step of all simulations is $\Delta t=0.2$.\\


\noindent
{\bf Movie S3. Structures in a Taylor-Green Flow without Ruptures} 
We adopt the relaxed configurations and place the JP structures in the Taylor-Green flow. 
In this movie, there is no apparent structural rupture in each case. For the choices of the flow strength, we pick: $V_0 = 0.1$ for the vesicle, $V_0 = 0.1$ for the bilayer, 
$V_0 = 0.1$ for the multi-lamellar, and $V_0 = 0.2$ for the striated configurations.
The time step of all simulations is $\Delta t=0.2$.\\


\noindent
{\bf Movie S4. Structures in a Shear Flow with Ruptures} 
We adopt the relaxed configurations and place the JP structures in the shear flow. 
In this movie, some clear structural ruptures occur in each case. For the choices of the shear rate, we pick: $\dot\gamma = 0.075$ for the vesicle, $\dot\gamma = 0.1$ for the bilayer, 
$\dot\gamma = 0.15$ for the multi-lamellar, and $\dot\gamma = 0.15$ for the striated configurations.
The time step of all simulations is $\Delta t=0.2$.
In order to observe the structure behaviors at later time, we stabilize the frame by tracking the center of mass position of all JP. \\


\noindent
{\bf Movie S5. Structures in a Taylor-Green Flow with Ruptures} 
We adopt the relaxed configurations and place the JP structures in the Taylor-Green flow. 
In this movie, some clear structural ruptures occur in each case. For the choices of the flow strength, we pick: $V_0 = 0.2$ for the vesicle, $V_0 = 0.2$ for the bilayer, 
$V_0 = 0.2$ for the multi-lamellar, and $V_0 = 0.3$ for the striated configurations.
The time step of all simulations is $\Delta t=0.2$.\\

