\documentclass[11pt]{article}

\usepackage{hyperref}
\usepackage{fullpage}
\usepackage{color}
\usepackage{newtxmath}
\newcommand{\comment}[1]{{\color{blue} #1}}

\begin{document}

\noindent Dear Prof. Cecile Cottin-Bizonne,\\

\noindent Thank you for handling the manuscript.
We thank the Reviewer for their constructive feedback.
A point-by-point list of changes to the manuscript 
addressing the Reviewer's comments appears below.\\

\noindent As a summary, we have undertaken a major revision of the manuscript
that includes further detailed simulation data and analyses,
reorganized figures, and a new ``Results and Discussion'' section
written from scratch.  These changes address all the reviewers'
comments and we believe the result is a stronger manuscript.\\


\noindent
Sincerely,
\\
\noindent
Szu-Pei Fu, Rolf Ryham, Bryan Quaife, and Yuan-Nan Young

\section*{Response to Reviewer 2}

\noindent Changes appear in \textcolor{magenta}{magenta}.
(Other changes specific to Reviewers 1 appear in \textcolor{red}{red}.)\\ \\

\comment{
\noindent
I report here on the manuscript "Effects of tunable hydrophobicity on the
collective hydrodynamics of Janus particles under flows" by Szu-Pei Fu et al.
submitted for publication to Physical Review Fluids. The authors have conducted
numerical hydrodynamic simulations in 2D on the self-assembly and dynamics of
nanoscale Janus particles under simple shear flow and Taylor-Green flow using
the method of discretized boundary integrals to describe the motion of the
particles immersed in 2D Stokes flow. The Janus character of the individual
particles is realized by tuning the local hydrophobicity/hydrophilicity. The
authors have previously shown that this system forms vesicle-like structures and
investigated such structures in simple shear flow (reference [18] in the paper).
The authors now continue this work by studying different model systems by tuning
the local hydrophobicity to model (i) bilayer-forming amphiphilic particles,
(ii) formation of multilamella structures, and (iii) formation of block-chain
structures, all pre-assembled under quiescent flow conditions. These equilibrium
structures are then studied under flow of varying shear rate. The different
structures are characterized by (i) a deformation order parameter, (ii) an
orientation order parameter, and (iii) the free energy. The authors study the
dynamics of the structures under flow, when mapping to physical scales, on the
order of micro-seconds, for nanometer-sized particles and shear rates on the
order of smaller than nanoseconds$^{-1}$. The effect of shear on the conformations
is analyzed in detail, and the results are discussed with respect to related
previous literature.\\ \\
In principle I think the paper meets the criteria to be considered as an article
for Physical Review Fluids in terms of originality, quality, and presentation.
However, I am not yet convinced about the potential applicability to real
systems, as raised below:}

\begin{itemize}
  \item {\bf Response}: We thank the Reviewer for this positive appraisal of 
our manuscript. 
\end{itemize}

%  \begin{quotation}
%    \noindent
%    Can put verbatim change in this environment.
%  \end{quotation}



\noindent
\comment{1) The authors study the nano-hydrodynamics around the JPs very carefully by
modeling the flow as a continuous 2D Stokes flow. While I understand that
modeling the dynamics in 2D instead of 3D is fair, i.e. their results should in
principle be extensible to 3D, my major concern is the absence of thermal
fluctuations. They are expected to be very relevant on the nanometer scale,
regarding self-assembly, dynamics and stability of the JP structures.
}

\begin{itemize}
\item {\bf Response:}
  To motivate excluding thermal fluctuations in our system,
  the new Section V.D. 
  analyzes that fluctuation Fourier spectrum 
  of a planar membrane corresponding to the Type I JP bilayer phase
  (starting on page 18).
  As the Reviewer points out, our 2D system applies
  to a transversely invariant phase with depth denoted by $L_y$.
  Equation (29) shows that the expected fluctutation is small
  relative to particle size, meaning that thermal fluctuations can
  be safely ignored when 2D system corresponds to a material that 
  is more than a few particles deep.  The elastic modulus $\kappa$ for
  Type II and Type III phases is still greater than for Type I.

  We point out though, that because the wave vector $\mathbf{k}$ is inversely
  proportional to the characteristic phase-sample size,
  fluctuation will be large provided the $L_x$ or $L_y$ is sufficiently large,
  as in the case of micron-sized vesicles.
\end{itemize}

\noindent
\comment{2) The shear rates studied are below nanoseconds$^{-1}$. What is the relevance of
such small shear rates in real (biological) systems? Furthermore, in presence of
noise, I would expect that the JP diffusive timescale is much smaller then the
shear flow time scale, and hence effectively the associated Peclet number, i.e.
importance of flow compared to noise, is negligible in this system?

Before I can recommend publication, the authors should clarify these points, in
particular justify why noise can be neglected, and why such small shear rates
are relevant.}

\begin{itemize}
\item {\bf Response:}
  In the beginning  of the new ``Results and Discussion'' Section V. (page 12),
  we now derive the range of shear rates 
  based on the capillary number $C_a$.  The derivation shows that shear forces dominate
  interfacial forces when $\dot \gamma$ exceeds
  10$^{-2}$ ns $^{-1} = 10^6$ s$^{-1}$.  Following the derivation,
  we include references to molecular dynamics simulations that also
  consider nm system sizes like ours, along with references to
  larger experimental systems were shear rates are more moderate,
  on the order of $10^3$ s$^{-1}$.  We end by calculating the
  P\'eclet number $\mathrm{Pe} = 60$, showing that
  noise is negligible compared to advection in our system.  

  Although the range of simulation shear rates is unchanged in our
  revision, we thank the Reviewer as their comment has clarified
  two important points: why we chose the range of shear
  rates that we did and how neglecting noise and
  thermal fluctuation can be justified in our system.

  Reviewer 1 pointed out that most of the simulation runs
  corresponded to breakage in JP phases. The new Figure 3
  now includes simulation runs for shear rates ranging
  from $0$ ns$^{-1}$ to $0.1$ ns$^{-1}$ in shear flow
  and from $0$ ns$^{-1}$ to $0.05$ ns$^{-1}$ in TG flow.
  These additional data allowed us to plot pre-rupture
  shear stress against shear rate in the new Figure 5
  revealing the rheological properties of the distinct
  JP phases. 
\end{itemize}



\end{document}
