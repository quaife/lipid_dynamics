\documentclass[11pt]{article}

\usepackage{hyperref}
\usepackage{fullpage}
\usepackage{color}
\usepackage{newtxmath}
\newcommand{\comment}[1]{{\color{blue} #1}}

\begin{document}

\noindent
Dear Prof.~Cecile Cottin-Bizonne,
\\ \\
\noindent
Thank you for handling the manuscript, and we would like to thank the
reviewer for their positive and constructive comments. The attached
manuscript addresses all the reviewer's comments and the result is what
we believe is a stronger manuscript. An itemized list of the changes
addressing the reviewer's comments are below. Changes to the manuscript
from Reviewers 1 and 2 are colored in \textcolor{red}{red} and 
\textcolor{magenta}{magenta}, respectively. \\ \\
\noindent
Sincerely,
\\
\noindent
Szu-Pei Fu, Rolf Ryham, Bryan Quaife, and Yuan-Nan Young

\section*{Response to Reviewer 2}
\comment{
\noindent
I report here on the manuscript "Effects of tunable hydrophobicity on the
collective hydrodynamics of Janus particles under flows" by Szu-Pei Fu et al.
submitted for publication to Physical Review Fluids. The authors have conducted
numerical hydrodynamic simulations in 2D on the self-assembly and dynamics of
nanoscale Janus particles under simple shear flow and Taylor-Green flow using
the method of discretized boundary integrals to describe the motion of the
particles immersed in 2D Stokes flow. The Janus character of the individual
particles is realized by tuning the local hydrophobicity/hydrophilicity. The
authors have previously shown that this system forms vesicle-like structures and
investigated such structures in simple shear flow (reference [18] in the paper).
The authors now continue this work by studying different model systems by tuning
the local hydrophobicity to model (i) bilayer-forming amphiphilic particles,
(ii) formation of multilamella structures, and (iii) formation of block-chain
structures, all pre-assembled under quiescent flow conditions. These equilibrium
structures are then studied under flow of varying shear rate. The different
structures are characterized by (i) a deformation order parameter, (ii) an
orientation order parameter, and (iii) the free energy. The authors study the
dynamics of the structures under flow, when mapping to physical scales, on the
order of micro-seconds, for nanometer-sized particles and shear rates on the
order of smaller than nanoseconds$^{-1}$. The effect of shear on the conformations
is analyzed in detail, and the results are discussed with respect to related
previous literature.\\ \\
In principle I think the paper meets the criteria to be considered as an article
for Physical Review Fluids in terms of originality, quality, and presentation.
However, I am not yet convinced about the potential applicability to real
systems, as raised below:}

\begin{itemize}
  \item {\bf Response to Summary}: We thank the Reviewer for this positive appraisal of 
our manuscript. 
\end{itemize}

%  \begin{quotation}
%    \noindent
%    Can put verbatim change in this environment.
%  \end{quotation}



\noindent
\comment{1) The authors study the nano-hydrodynamics around the JPs very carefully by
modeling the flow as a continuous 2D Stokes flow. While I understand that
modeling the dynamics in 2D instead of 3D is fair, i.e. their results should in
principle be extensible to 3D, my major concern is the absence of thermal
fluctuations. They are expected to be very relevant on the nanometer scale,
regarding self-assembly, dynamics and stability of the JP structures.
}

\begin{itemize}
  \item Response:  
  
  \item Change:  
  
\end{itemize}

\noindent
\comment{2) The shear rates studied are below nanoseconds$^{-1}$. What is the relevance of
such small shear rates in real (biological) systems? Furthermore, in presence of
noise, I would expect that the JP diffusive timescale is much smaller then the
shear flow time scale, and hence effectively the associated Peclet number, i.e.
importance of flow compared to noise, is negligible in this system?}

\begin{itemize}
  \item Response:  
  
  \item Change: 
  
\end{itemize}

\noindent
\comment{Before I can recommend publication, the authors should clarify these points, in
particular justify why noise can be neglected, and why such small shear rates
are relevant.}

\begin{itemize}
  \item Response: 
\end{itemize}


\end{document}
