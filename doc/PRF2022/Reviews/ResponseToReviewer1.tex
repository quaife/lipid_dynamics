\documentclass[11pt]{article}

\usepackage{fullpage}
\usepackage{color}
\usepackage{newtxmath}
\newcommand{\comment}[1]{{\color{blue} #1}}

\begin{document}

\noindent
Dear Prof.~Cecile Cottin-Bizonne,
\\ \\
\noindent
Thank you for handling the manuscript, and we would like to thank the
reviewer for their positive and constructive comments. The attached
manuscript addresses all the reviewer's comments and the result is what
we believe is a stronger manuscript. An itemized list of the changes
addressing the reviewer's comments are below. 
%Changes to the manuscript from Reviewers 1 and 2 are colored in
%\textcolor{red}{red} and \textcolor{magenta}{magenta}, respectively. 
\\ \\
\noindent
Sincerely,
\\
\noindent
Szu-Pei Fu, Rolf Ryham, Bryan Quaife, and Yuan-Nan Young

\section*{Response to Reviewer 1}
\noindent
\comment{The manuscript discusses the self-assembly of amphiphilic Janus
particles and their dynamics in various flow conditions. Different
morphologies are recovered for different boundary conditions which are
developed by varying the hydrophobicity on Janus particle surface.
These various boundary conditions give rise to unilamellar,
multilamellar and striated structures. Finally, these structures are
subjected to linear shear flow and Taylor-Green flow. The subject of
study is interesting to soft matter community.}

\begin{itemize}
  \item {\bf Response to Summary}: We thank the Reviewer for the
    positive assessment of our manuscript. We have addressed each of the
    reviewer's comments which are summarized below.
%  \begin{quotation}
%    \noindent
%    Can put verbatim change in this environment.
%  \end{quotation}
\end{itemize}


\noindent
\comment{My main criticism is the following: 1. Unilamellar,
multilamellar and striated structured are subjected to various flow
conditions, with increasing shear rate or strength of the flow (for TG
case). For all structures in both flows, the main result is that these
structures break eventually either for increased shear rate or increased
flow rate. But this is expected. What do all these results mean? I feel
that a constructive message is missing.}

\begin{itemize}
  \item Response: 
  
  \item Change:
  
\end{itemize}


\noindent
\comment{2. And the discussion section is not written well. Please write it
properly.}

\begin{itemize}
  \item Response:  
  
  \item Change: 
  
\end{itemize}


\noindent
\comment{Minor comments: 1. Page 18: The section ends with 
the statement “In previous work, Fu
et al. [18] determined the permeability constant of particle-based
vesicles in the context of shear background flow [53, 54]. “. I find
the statement abrupt and not related to the earlier text.}

\begin{itemize}
  \item Response: 
  
\item Change: 


\end{itemize}

\noindent
\comment{2. Fig. 8: Please describe what the arrows mean in (a), (b), (c) and
(d).}

\begin{itemize}
  \item Response: 
  
  \item Change:
  
\end{itemize}



\end{document}
