\documentclass[11pt]{article}

\usepackage{fullpage}
\usepackage{color}
\usepackage{newtxmath}
\newcommand{\comment}[1]{{\color{blue} #1}}

\begin{document}

\noindent
Dear Prof.~C\'ecile Cottin-Bizonne,
\\ \\
\noindent 
We thank the Reviewer for their constructive feedback.
A point-by-point list of changes to the manuscript 
addressing the Reviewer's comments appears below.\\

\noindent As a summary, we have undertaken a major revision of the manuscript
that includes further detailed simulation data and analyses,
reorganized figures, and a new ``Results and Discussion'' section
written from scratch.  These changes address all the Reviewer's
comments and we believe the result is a significantly stronger manuscript.
\\ \\
\noindent
Sincerely,
\\
\noindent
Szu-Pei Fu, Rolf Ryham, Bryan Quaife, and Yuan-Nan Young

\section*{Response to Reviewer 1}

\noindent Changes appear in \textcolor{red}{red}.
(Other changes specific to Reviewer 2 appear in \textcolor{magenta}{magenta}.)\\ \\

\noindent
\comment{The manuscript discusses the self-assembly of amphiphilic Janus
particles and their dynamics in various flow conditions. Different
morphologies are recovered for different boundary conditions which are
developed by varying the hydrophobicity on Janus particle surface.
These various boundary conditions give rise to unilamellar,
multilamellar and striated structures. Finally, these structures are
subjected to linear shear flow and Taylor-Green flow. The subject of
study is interesting to soft matter community.}

\begin{itemize}
  \item {\bf Response to Summary}: We thank the Reviewer for their
    positive assessment of our manuscript.  
%  \begin{quotation}
%    \noindent
%    Can put verbatim change in this environment.
%  \end{quotation}
\end{itemize}


\noindent
\comment{My main criticism is the following: 1. Unilamellar,
multilamellar and striated structured are subjected to various flow
conditions, with increasing shear rate or strength of the flow (for TG
case). For all structures in both flows, the main result is that these
structures break eventually either for increased shear rate or increased
flow rate. But this is expected. What do all these results mean? I feel
that a constructive message is missing.}

\begin{itemize}
  \item[] {\bf Response:}
      The Results Section V of our first submission
    analyzed simulation data on a case-by-case basis. 
    This made it unnecessarily difficult for the reader to
    make meaningful connections and we agree with the Reviewer:
    a constructive message was missing.  Moreover,
    the large-shear-rate 
    simulation runs for the most part corresponded to breakage in
    JP phases.

    To address these deficiencies,
    we performed further simulations 
    for moderate shear rates ranging
    from $0$ ns$^{-1}$ to $0.1$ ns$^{-1}$ in shear flow
    and from $0$ ns$^{-1}$ to $0.05$ ns$^{-1}$ in TG flow.
    These additional data allowed us to examine 
    material properties in greater detail: the new
    \begin{itemize}
      \item[$\bullet$] Figure 3 summarizes the strain and alignment data
        under \emph{all} flow and JP-type conditions.
      \item[$\bullet$] Figure 4 characterizes the change in free and binding energies.
      \item[$\bullet$] Figure 5 plots pre-rupture
        shear stress against shear rate revealing the rheological properties of
        the three distinct JP phases.    
    \end{itemize}
    A constructive message---identification a critical shear rate,
    relative binding strengths as a function of JP type,
    and effective viscosities and presence of shear thinning---is
    now formulated thanks to exploring the full range of
    shear rates.

    Finally, we normalized the TG flow rate (previously $V_0$)
    so that we can compare the shear and TG flow in terms of the same
    shear rate $\dot \gamma$.  See page 11, equation (24) and page 12,
    top paragraph.
\end{itemize}

\noindent
\comment{2. And the discussion section is not written well. Please write it
properly.}
    \begin{itemize}
    \item[] {\bf Response:}
      The address this,
      we rewrote the Results section from scratch
    to what is now Section V ``Results and Discussion''.
    We highlighted the section heading on page 11 to indicate
    that the entire section is rewritten.

    Starting on page 11, this section introduces the background flows,
    characterizes the system's capillary and P\'eclet numbers.
    As described in the previous response, Section V examines the data
    as a whole and is divided into subsections A. Strain and alignment,
    B. Free energy, and C. Rheology.

    The raw data, which was previously shown in the
    main text, has now been placed in Supplementary Material
    and appropriately captioned.  We also combined
    Figures 2 and 3 into a simpler, more readable Figure 2.    

    Finally, we rewrote the previous Discussion Section VI from scratch
    into Section V, subsection D. Here,
    we perform a dimensional analysis of the critical shear rate $\dot \gamma_*$
    for rupture, show how fluid-structure interactions
    give rise to non-monotonic dependence, justify why
    fluctuation can been omitted from the simulations, and
    discuss the rheological data in the context of 
    rheologies for classical and active suspensions.

    We feel that these changes have significantly improved the readability
    of the manuscript.  As an added advantage, the main text of the
    revised manuscript is 21 pages, whereas the first submission's was
    25 pages.  
    \end{itemize}

\noindent
\comment{Minor comments: 1. Page 18: The section ends with 
the statement “In previous work, Fu
et al. [18] determined the permeability constant of particle-based
vesicles in the context of shear background flow [53, 54]. “. I find
the statement abrupt and not related to the earlier text.}

\begin{itemize}
  \item[] {\bf Response:}
    We have omitted this sentence and rewritten the discussion
    for vesicles in TG flow.  See second middle of page 18.   
\end{itemize}

\noindent
\comment{2. Fig. 8: Please describe what the arrows mean in (a), (b), (c) and
(d).}

\begin{itemize}
\item[] {\bf Response:}
  In our first submission, these arrows showed the clockwise or counterclockwise
  vesicle motion in TG flow.  These arrows have been omitted in the new Figure 6,
  and the motion is instead described in the middle of page 18. 
\end{itemize}



\end{document}
